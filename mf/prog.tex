%% LyX 1.6.5 created this file.  For more info, see http://www.lyx.org/.
%% Do not edit unless you really know what you are doing.
\documentclass[12pt,a4paper,english]{article}
\usepackage[T1]{fontenc}
\usepackage[latin9]{inputenc}
\pagestyle{empty}

\makeatletter
%%%%%%%%%%%%%%%%%%%%%%%%%%%%%% User specified LaTeX commands.

%%%%%%%%%%%%%%%%%%%%%%%%%%%%%%%%%%%%%%%%%%%%%%%%%%%%%%%%%%%%%%%%%%%%%%%%%%%%%%%%%%%%%%%%%%%%%%%%%%%%%%%%%%%%%%%%%%%%%%%%%%%%%%%%%%%%%%%%%%%%%%%%%%%%%%%%%%%%%%%%%%%%%%%%%%%%%%%%%%%%%%%%%%%%%%%%%%%%%%%%%%%%%%%%%%%%%%%%%%%%%%%%%%%%%%%%%%%%%%%%%%%%%%%%%%%%
%TCIDATA{OutputFilter=LATEX.DLL}
%TCIDATA{Version=5.50.0.2953}
%TCIDATA{<META NAME="SaveForMode" CONTENT="1">}
%TCIDATA{BibliographyScheme=Manual}
%TCIDATA{LastRevised=Friday, April 09, 2010 09:59:02}
%TCIDATA{<META NAME="GraphicsSave" CONTENT="32">}
\topmargin-0.7cm\headheight0cm\headsep0cm\textheight26cm\parindent0cm



\makeatother

\usepackage{babel}

\begin{document}
\textsc{\small universit� della svizzera italiana}\hfill{}\textsc{\small facolt�
di scienze economiche}\\
 \textsc{\small dr. claudio ortelli}\hfill{}\textsc{\small semestre
estivo 2011}{\small \par}

\vspace{0.4cm}


\begin{center}
\textbf{\large Metodi Quantitativi\\[0pt] Profilo Finanziario\\[0pt]
\vspace{0.2cm}
 (seconda parte)} 
\par\end{center}

\vspace{0.1cm}


\vspace{-0.3cm}
 $\hrulefill$

\vspace{0.5cm}

\begin{enumerate}
\item \textsc{Fonti e raccolta di dati finanziari}

\begin{enumerate}
\item [1.1] Fornitori di dati finanziari
\item [1.2] Codici d'identificazione
\item [1.3] Datastream, classi di strumenti finanziari e tipologie di
dati 
\end{enumerate}
\item \textsc{B. Chapter 7: Portfolio Models - Introduction}

\begin{enumerate}
\item [2.1] Notazione
\item [2.2] Rendimenti percentuali e logaritmici: definizione e propriet�

\begin{enumerate}
\item [2.2.1] Il caso con dividendi 
\end{enumerate}
\item [2.3] Varianza, covarianza e coefficiente di correlazione
\item [2.4] Valore atteso e varianza del rendimento di un portafoglio

\begin{enumerate}
\item [2.4.1] Caso con due soli titoli
\item [2.4.2] Caso generale 
\end{enumerate}
\item [2.5] Portafoglio efficiente: concetti generali e definizione 
\end{enumerate}
\item \textsc{Introduzione al linguaggio di programmazione R}

\begin{enumerate}
\item [3.1] Premessa
\item [3.2] Installazione
\item [3.3] Strutture semplici e strutture di dati
\item [3.4] Grafici
\item [3.5] Comandi if, for, while, repeat
\item [3.6] Funzioni e programmazione da Excel 
\end{enumerate}
\item \textsc{B. Chapter 8: Calculating the Variance-Covariance Matrix}

\begin{enumerate}
\item [4.1] In Excel
\item [4.2] In R
\item [4.3] The Single-Index Model 
\end{enumerate}
\newpage{}

\item \textsc{B. Chapter 10: Calculating Efficient Portfolios When are No
Short-Sale Restrictions}

\begin{enumerate}
\item [5.1] Notazione e definizioni preliminari
\item [5.2] La matematica della frontiera efficiente
\item [5.3] Implementazione con dati reali 
\end{enumerate}
\item \textsc{B. Chapter 15: The Lognormal Distribution}

\begin{enumerate}
\item [6.1] Introduzione: il cammino dei prezzi azionari
\item [6.2] La distribuzione lognormale
\item [6.3] Simulazione di variabili aleatorie
\item [6.4] The Geometric Random Walk with Drift 
\end{enumerate}
\item \textsc{B. Chapter 12: Value at Risk (VaR)}

\begin{enumerate}
\item [7.1] Definizione
\item [7.2] Approssimazione tramite la distribuzione Normale
\item [7.3] Critiche alla distribuzione Normale
\item [7.4] La metodologia di simulazione storica
\item [7.5] Osservazioni finali 
\end{enumerate}
\end{enumerate}
\vspace{1.5cm}


\underline{Testi di riferimento}
\begin{itemize}
\item Benninga, Simon, Financial Modeling, MIT Press, 2008


(BUL A 658.150285 BEN FIN) \end{itemize}

\end{document}
