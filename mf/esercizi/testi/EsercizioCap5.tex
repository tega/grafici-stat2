
\documentclass[a4paper,12pt]{article}
%%%%%%%%%%%%%%%%%%%%%%%%%%%%%%%%%%%%%%%%%%%%%%%%%%%%%%%%%%%%%%%%%%%%%%%%%%%%%%%%%%%%%%%%%%%%%%%%%%%%%%%%%%%%%%%%%%%%%%%%%%%%%%%%%%%%%%%%%%%%%%%%%%%%%%%%%%%%%%%%%%%%%%%%%%%%%%%%%%%%%%%%%%%%%%%%%%%%%%%%%%%%%%%%%%%%%%%%%%%%%%%%%%%%%%%%%%%%%%%%%%%%%%%%%%%%
\usepackage{amsmath}
\usepackage[italian]{babel}


\topmargin-0.7cm
\headheight0cm
\headsep0cm
\textheight24cm
\pagestyle{empty}
\parindent0cm
\begin{document}


{\small \textsc{universit\`{a} della svizzera italiana}}\hfill {\small 
\textsc{facolt\`{a} di scienze economiche}}\newline
{\small \textsc{Dr. Claudio Ortelli}}\hfill {\small \textsc{semestre estivo
2010}}

\vspace{0.2cm}

\begin{center}
\textbf{{\large Metodi quantitativi per economia finanziaria \\[0pt]
\vspace{0.5cm} Capitolo 5: Esercizio supplementare}}
\end{center}

\vspace{0.1cm}

\hrulefill

\vspace{0.3cm}


\begin{enumerate}
\item Utilizzando i medesimi dati dell'Esercizio 6, \textquotedblleft The single-Index Model\textquotedblright, calcolate:

\begin{enumerate}
\item Le costanti $A,~B,~C,~D$, ed i vettori $h$ e $g$.

\item La frontiera.

\item I pesi del $mvp$. Lo si rappresenti nel grafico della frontiera.

\item Nel grafico della frontiera si inseriscano le due rette degli asintoti.

\item Dato il portafoglio frontiera $p$ di rendimento atteso 2\%, calcolate
la struttura dei pesi e la varianza di $zc(p)$. Fate il grafico della retta
tangente alla frontiera in $p$ e verificate che passa nel punto $(0,E(r_{zc(p)}))$.
\end{enumerate}
\end{enumerate}

\end{document}
