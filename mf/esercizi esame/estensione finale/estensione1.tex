
\documentclass[a4paper,12pt]{article}
%%%%%%%%%%%%%%%%%%%%%%%%%%%%%%%%%%%%%%%%%%%%%%%%%%%%%%%%%%%%%%%%%%%%%%%%%%%%%%%%%%%%%%%%%%%%%%%%%%%%%%%%%%%%%%%%%%%%%%%%%%%%
%TCIDATA{OutputFilter=LATEX.DLL}
%TCIDATA{Version=4.00.0.2312}
%TCIDATA{LastRevised=Friday, July 08, 2005 08:57:18}
%TCIDATA{<META NAME="GraphicsSave" CONTENT="32">}

\input{tcilatex}

\begin{document}


{\Large Esercizio 1}

\bigskip

Sono date due azioni $A$ e $B$ i cui prezzi alla data $t$ sono notati con $%
S_{t}^{A}$ rispettivamente $S_{t}^{B}$. Per quanto riguarda la dinamica dei
prezzi azionari si assume per entrambe le azioni un modello di tipo random
walk geometrico con drift, cio\'{e}

\begin{eqnarray}
\ln S_{t+\Delta t}^{A} &=&\ln S_{t}^{A}+\mu ^{A}\Delta t+\sigma ^{A}\sqrt{%
\Delta t}\ \varepsilon _{t+\Delta t}^{A}  \label{eq1} \\
\ln S_{t+\Delta t}^{B} &=&\ln S_{t}^{B}+\mu ^{B}\Delta t+\sigma ^{B}\sqrt{%
\Delta t}\ \varepsilon _{t+\Delta t}^{B}  \label{eq2}
\end{eqnarray}

Riguardo alle propriet\`a stocastiche dei termini d'errore vale quanto segue:

\begin{itemize}
\item $\{\varepsilon_{t}^A\}$ i.i.d. $N(0,1)$.

\item $\{\varepsilon_{t}^B\}$ i.i.d. $N(0,1)$.

\item $cov(\varepsilon_{t}^A,\varepsilon_{t}^B)=\rho \ \forall \ t$, con $%
-1\leq \rho \leq 1$.

\item $cov(\varepsilon_{t}^A,\varepsilon_{s}^B)=0, \ t \neq s$.
\end{itemize}

Notiamo che poich\'{e} la varianza degli errori \`{e} pari ad uno, $\rho $
corrisponde al coefficiente di correlazione. Da esso dipende la covarianza
fra i rendimenti di $A$ e $B$.

\begin{enumerate}
\item La messa in piega del modello

Riscriviamo le due equazioni precedenti in forma matriciale nel modo
seguente: 
\begin{equation}
\ln S_{t+\Delta t}=\ln S_{t}+\Delta t\,\mu +\sqrt{\Delta t}D\ \varepsilon
_{t+\Delta t}  \label{eq3}
\end{equation}

\begin{enumerate}
\item Date dimensione e contenuto di ogni simbolo.

\item Calcolate la matrice di covarianza di $\varepsilon _{t+\Delta t}$,
notata $\Omega $.
\end{enumerate}

\item Vettore dei rendimenti logaritmici

Il vettore dei rendimenti logaritmici, notato $r_{t+\Delta t}$, \`{e}
definito come 
\begin{equation}
r_{t+\Delta t}:=\ln S_{t+\Delta t}-\ln S_{t}=\Delta t\,\mu +\sqrt{\Delta t}%
D\ \varepsilon _{t+\Delta t}.  \label{eq4}
\end{equation}%
Notate che poich\'{e} $\ln S_{t+\Delta t}$ \`{e} un vettore, non ha pi\`{u}
senso scrivere $\ln S_{t+\Delta t}/S_{t}$! $r_{t+\Delta t}$ corrisponde come
nel caso univariato al rendimento logaritmico riferito ad un arco temporale
di lunghezza\footnote{%
Ricordiamo che la scelta di $\Delta t$ dipende dall'unit\`{a} di misura di $t
$. Se $t$ fosse misurato in anni e $\Delta t$ rappresentasse un giorno,
allora $\Delta t=1/252\;$\ (in un anno ci sono approssimativamente $21\times
12=252$ giorni lavorativi), mentre se $\Delta t$ rappresentasse un mese,
allora $\Delta t=1/12$. Se invece scegliessimo quale unit\`{a} di misura di $%
t$ il giorno, allora un $\Delta t$ di un mese sarebbe pari a $\Delta t=21$.} 
$\Delta t$.

\begin{enumerate}
\item Si dia il significato di $\mu $ e di $D$.
\end{enumerate}

\item La matrice di varianza covarianza

\begin{enumerate}
\item Utilizzando le propriet\`{a} di una trasformazione lineare di un
vettore aleatorio calcolate valore atteso e matrice delle covarianze di $%
r_{t+\Delta t}$.

\item Dimostrate che valore atteso e varianza del rendimento su $n\,\Delta t$
periodi, $\ln S_{t+n\Delta t}-\ln S_{t}$, \`{e} pari a $n\,\Delta t\,\mu $
rispettivamente $n\,\Delta tD\Omega D$. Date esplicitamente il contenuto di $%
n\,\Delta tD\Omega D$ in funzione di $\Delta t,$\ $\sigma ^{A},$\ $\sigma
^{B}$ e $\rho $.

\item Avete stimato la matrice di covarianza utilizzando dati giornalieri
dalla quale estraete $\widehat{cov}(r^{A},r^{B})=1.4$. Date la covarianza
fra i rendimenti mensili.
\end{enumerate}
\end{enumerate}

\end{document}
