
\documentclass[a4paper,12pt]{article}
%%%%%%%%%%%%%%%%%%%%%%%%%%%%%%%%%%%%%%%%%%%%%%%%%%%%%%%%%%%%%%%%%%%%%%%%%%%%%%%%%%%%%%%%%%%%%%%%%%%%%%%%%%%%%%%%%%%%%%%%%%%%
%TCIDATA{OutputFilter=LATEX.DLL}
%TCIDATA{Version=4.00.0.2312}
%TCIDATA{LastRevised=Monday, July 04, 2005 15:51:57}
%TCIDATA{<META NAME="GraphicsSave" CONTENT="32">}

\input{tcilatex}

\begin{document}


{\Large Esercizio 2}

\bigskip

E' dato un portafoglio con valuta di riferimento chf. Il vostro universo
titoli comprende $UBS$, $BASF$ ed un conto corrente in euro. Il portafoglio 
\`{e} costituito da una posizione azionaria in chf, diciamo $1^{\prime }000$
chf in $UBS$ ed una posizione cash di $1^{\prime }000$ Euro ad interesse 0.
Il prezzo di $UBS$ oggi \`{e} di chf $100.-$, quello di $BASF$ (in euro) 
\`{e} di $55.-$ ed infine il cambio euro franco \`{e} pari a $1.55$.

\begin{enumerate}
\item Quanti sono i fattori di rischio di questo portafoglio e perch\'{e}?

\item Calcolate il valore del portafoglio in chf e le posizioni percentuali
investite nei due strumenti.

\item Avete a disposizione la seguente matrice di covarianza, stimata
utilizzando i rendimenti logaritmici giornalieri e poi \emph{annualizzata}:%
\[
\Sigma =\left[ 
\begin{array}{cccc}
& UBS & BASF & EUR \\ 
UBS & 0.02643 & 0.00966 & 0.00057 \\ 
BASF &  & 0.02486 & 0.00018 \\ 
EUR &  &  & 0.00081%
\end{array}%
\right] 
\]%
Allo stesso modo il vettore dei rendimenti logaritmici attesi \`{e} dato da%
\[
\mu =\left[ 
\begin{array}{cc}
UBS & 5\% \\ 
BASF & 8\% \\ 
EUR & 1\%%
\end{array}%
\right] .
\]%
Si noti che per quanto riguarda $BASF$ si tratta di rendimenti in valuta
locale, cio\'{e} in \emph{euro}!

\begin{enumerate}
\item Calcolate il vettore dei rendimenti logaritmici attesi in moneta di
riferimento, cio\'{e} in chf.

\item Calcolate la matrice delle covarianze dei rendimenti logaritmici in
chf.

\item Calcolate il vettore delle volatilit\`{a} annue.
\end{enumerate}

Supponiamo ora un random walk geometrico con drift quale modello generatore
dei rendimenti ed un orizzonte temporale per il calcolo de VaR pari ad 1
mese. 

\item Calcolate il rendimento atteso e la varianza a un mese del portafoglio
attuale.

\item Calcolate il VaR a 1 mese del portafoglio attuale ad un livello di
confidenza del $5\%$.

\item Avete a disposizione i sequenti rendimenti logaritmici storici \emph{%
giornalieri} e desiderate calcolare il VaR:%
\[
\begin{array}{cccc}
& UBS & BASF & EUR \\ 
t=1 & 0.009 & 0.022 & 0.001 \\ 
t=2 & 0.011 & -0.012 & 0.002 \\ 
t=3 & -0.002 & -0.003 & 0.015%
\end{array}%
\]

\begin{enumerate}
\item Eseguite un solo campionamento utilizzando la tecnica di simulazione
storica scegliendo una qualsiasi delle tre date.

\item Secondo quale distribuzione discreta devono essere campionate le date?
\end{enumerate}
\end{enumerate}

\end{document}
