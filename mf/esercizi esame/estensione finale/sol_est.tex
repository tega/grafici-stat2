
\documentclass[a4paper,12pt]{article}


\begin{document}


\noindent{\Large Soluzione esercizio 2}

\bigskip

E' dato un portafoglio con valuta di riferimento chf. Il vostro universo
titoli comprende $UBS$, $BASF$ ed un conto corrente in euro. Il portafoglio
\`{e} costituito da una posizione azionaria in chf, diciamo $1^{\prime }000$
chf in $UBS$ ed una posizione cash di $1^{\prime }000$ Euro ad interesse 0.
Il prezzo di $UBS$ oggi \`{e} di chf $100.-$, quello di $BASF$ (in euro)
\`{e} di $55.-$ ed infine il cambio euro franco \`{e} pari a $1.55$.

\begin{enumerate}
\item  Quanti sono i fattori di rischio di questo portafoglio e perch\'{e}?

I fattori di rischio presenti al momento nel portafoglio sono solo 2. In
particolare, la posizione cash essendo in euro \`{e} soggetta al rischio di
cambio.

\item  Calcolate il valore del portafoglio in chf e le posizioni percentuali
investite nei tre strumenti.

$V_{0}=1000+1000\ast 1.55=2550$ chf.

\[
w=\left[ 
\begin{array}{cc}
UBS & 1000/2550 \\ 
BASF & 0 \\ 
EUR & 1550/2550
\end{array}
\right] =\left[ 
\begin{array}{cc}
UBS & 39\% \\ 
BASF & 0 \\ 
EUR & 61\%
\end{array}
\right] 
\]

\item  Avete a disposizione la seguente matrice di covarianza, stimata
utilizzando i rendimenti logaritmici giornalieri e poi \emph{annualizzata}:
\[
\Sigma =\left[ 
\begin{array}{cccc}
& UBS & BASF & EUR \\ 
UBS & 0.02643 & 0.00966 & 0.00057 \\ 
BASF &  & 0.02486 & 0.00018 \\ 
EUR &  &  & 0.00081
\end{array}
\right] 
\]
Allo stesso modo il vettore dei rendimenti logaritmici attesi \`{e} dato da
\[
\mu =\left[ 
\begin{array}{cc}
UBS & 5\% \\ 
BASF & 8\% \\ 
EUR & 1\%
\end{array}
\right] .
\]
Si noti che per quanto riguarda $BASF$ si tratta di rendimenti in valuta
locale, cio\'{e} in \emph{euro}!

\begin{enumerate}
\item  Calcolate il vettore dei rendimenti logaritmici attesi in moneta di
riferimento, cio\'{e} in chf.

Dalla relazione 
\[
P_{BASF}^{chf}=P_{BASF}^{eur}\ast e
\]
si ricava semplicemente applicando la definizione di rendimento logaritmico
e la propriet\`{a} della funzione $\ln $ che

\begin{equation}
r_{BASF}^{chf}=r_{BASF}^{eur}+r_e.  \label{eq1}
\end{equation}
Avremo quindi che 
\[
\mu ^{chf}=\left[ 
\begin{array}{cc}
UBS & 5\% \\ 
BASF & 8\%+1\% \\ 
EUR & 1\%
\end{array}
\right] =\left[ 
\begin{array}{cc}
UBS & 5\% \\ 
BASF & 9\% \\ 
EUR & 1\%
\end{array}
\right] .
\]

\item  Calcolate la matrice delle covarianze dei rendimenti logaritmici in
chf.

Dalla (\ref{eq1}) abbiamo:
\begin{eqnarray*}
V(r_{BASF}^{chf}) &=&V(r_{BASF}^{eur})+V(r_e)+2Cov(r_{BASF}^{eur},r_e) \\
&=&0.02486+0.00081+0.00018  \nonumber \\
&=&0.02585  \nonumber \\
Cov(r_{BASF}^{chf},r_e) &=&Cov(r_{BASF}^{eur},r_e)+Cov(r_e,r_e)  \nonumber \\
&=&Cov(r_{BASF}^{eur},r_e)+V(r_e) \\
&=&0.00018+0.00081  \nonumber \\
&=&0.00099  \nonumber \\
Cov(r_{BASF}^{chf},r_{UBS}) &=&Cov(r_{BASF}^{eur},r_{UBS})+Cov(r_e,r_{UBS}) \\
&=&0.00966+0.00057  \nonumber \\
&=&0.01023  \nonumber
\end{eqnarray*}

La matrice di covarianza in chf \`{e} pertanto uguale a
\[
\Sigma ^{chf}=\left[ 
\begin{array}{cccc}
& UBS & BASF & EUR \\ 
UBS & 0.02643 & 0.01023 & 0.00057 \\ 
BASF &  & 0.02585 & 0.00099 \\ 
EUR &  &  & 0.00081
\end{array}
\right] 
\]

\item  Calcolate il vettore delle volatilit\`{a} annue.

\[
\sigma ^{chf}=\sqrt{diag(\Sigma ^{chf})}=\left[ 
\begin{array}{cc}
UBS & 16.25\% \\ 
BASF & 16.07\% \\ 
EUR & 2.84\%
\end{array}
\right] 
\]
\end{enumerate}

Supponiamo ora un random walk geometrico con drift quale modello generatore
dei rendimenti ed un orizzonte temporale per il calcolo de VaR pari ad 1
mese.

\item  Calcolate il rendimento atteso e la varianza a un mese del
portafoglio attuale.

Vedete calcolo foglio excel allegato per il calcolo del rendimento atteso e
la varianza.

\item  Calcolate il VaR a 1 mese del portafoglio attuale ad un livello di
confidenza del $5\%$.

Il $VaR(5\%,1m)$ sar\`{a} dato da meno il 5\%-quantile di una variabile
aleatoria $N(\mu _{1m},\sigma _{1m}^{2})$ cio\`{e}
\begin{eqnarray*}
VaR(5\%,1Mese) &=&-\left( \mu _{1mese}+\sigma _{1mese}\ast q_{5\%}\right)  \\
&=&2.49\% 
\end{eqnarray*}
(per il calcolo si veda foglio excel)

Quindi con una probabilit\`a del 5\% sull'arco del prossimo mese il portafoglio potr\`a
perdere 2.49\% (63.50 chf) o pi\`u. 
\item  Avete a disposizione i sequenti rendimenti logaritmici storici \emph{%
giornalieri} e desiderate calcolare il VaR:
\[
\begin{array}{cccc}
& UBS & BASF & EUR \\ 
t=1 & 0.009 & 0.022 & 0.001 \\ 
t=2 & 0.011 & -0.012 & 0.002 \\ 
t=3 & -0.002 & -0.003 & 0.015
\end{array}
\]

\begin{enumerate}
\item  Eseguite un solo campionamento utilizzando la tecnica di simulazione
storica scegliendo una qualsiasi delle tre date.

Abbiamo a disposizione dati giornalieri dai quali abbiamo stimato il
rendimento atteso e la varianza (le covarianze non ci interessano in questo
caso) utilizzando il modello random walk con drift che nella sua generalit%
\`{a} \`{e} scritto come
\begin{equation}
r_{t+\Delta t}=\Delta t\mu +\sigma \sqrt{\Delta t}\varepsilon _{t+\Delta
t},\; con \varepsilon _{t+\Delta t}\sim N(0,1)
\end{equation}
Scegliendo quindi $t=giorni$ e $\Delta t=1$ otteniamo
\begin{equation}
r_{t+1}=\mu _{1g}+\sigma _{1g}\varepsilon _{t+1}  \label{eq2}
\end{equation}
e risolvendo rispetto a $\varepsilon _{t+1}$ si ottiene
\[
\varepsilon _{t+1}=\frac{r_{t+1}-\mu _{1g}}{\sigma _{1g}}.
\]
Avendo a disposizione una stima di $\mu $ e $\sigma $, diciamo $\widehat{\mu 
}$ e $\widehat{\sigma }$, possiamo calcolare l'errore stimato $\widehat{%
\varepsilon }_{t+1}$:
\[
\widehat{\varepsilon }_{t+1}=\frac{r_{t+1}-\widehat{\mu }_{1g}}{\widehat{%
\sigma }_{1g}}.
\]
Possiamo in questo modo, utilizzando i rendimenti giornalieri ed i parametri
stimati $\widehat{\mu }_{1g}$ e $\widehat{\sigma }_{1g}$, ottenere una serie
di innovazioni $\widehat{\varepsilon }_{t},\;t=1,\ldots N$ estratte da una $%
N(0,1)$. Queste realizzazioni verranno utilizzate nel modesimo modello con $%
t=mesi$. Infatti, sappiamo che se scegliamo $t=mesi$ e $\Delta t=1$, avremo
l'analogo dell'equazione (\ref{eq2}), e cio\`{e}
\begin{equation}
r_{t+1}=\mu _{1m}+\sigma _{1m}\varepsilon _{t+1},\;  %
\varepsilon _{t+1}\sim N(0,1)  \label{eq3}
\end{equation}
Ora, scalando propriamente il rendimento atteso e la varianza stimati
possiamo simulare rendimenti mensili utilizzando l'equazione (\ref{eq3}) e
le innovazioni $\widehat{\varepsilon }_{t}$ stimate su dati giornalieri. La
procedura \`{e} la seguente:

\begin{enumerate}
\item  Campioniamo casualmente un numero compreso fra $1$ ed $N$, nel nostro caso $N=3$.
 Supponiamo d'aver ottenuto 2.

\item  Calcoliamo una simulazione mensile di $r_{1m}$, notata $\widetilde{r}%
_{1m}$ utilizzando la  (\ref{eq3}), cio\`{e}

\[
\widetilde{r}_{1m}=\widehat{\mu }_{1m}+\widehat{\sigma }_{1m}\widehat{%
\varepsilon }_{2}
\]

\item  Conoscendo il valore attuale dello strumento $P_{0}$ e sapendo che $%
r_{1m}=\ln \left( P_{1m}/P_{0}\right) $ otteniamo che
\begin{equation}
\widetilde{P}_{1m}=P_{0}\exp (\widetilde{r}_{1m})  \label{eq4}
\end{equation}
Questa tuttavia non \`{e} la formula finale. Partendo dalla (\ref{eq4})
possiamo scrivere $\widetilde{P}_{1m}$ in funzione di $\widehat{\mu }_{1m},\;%
\widehat{\sigma }_{1m},\;\mu _{1g},\;\sigma _{1g}\;$ed il rendimento
giornaliero $r_{2}:$%
\begin{eqnarray}
\widetilde{P}_{1m} &=&P_{0}\exp (\widehat{\mu }_{1m}+\widehat{\sigma }_{1m}%
\widehat{\varepsilon }_{2})=P_{0}\exp \left( \widehat{\mu }_{1m}+\widehat{%
\sigma }_{1m}\frac{r_{2}-\widehat{\mu }_{1g}}{\widehat{\sigma }_{1g}}\right) 
\nonumber \\
&=&P_{0}\exp \left( \widehat{\mu }_{1m}+\frac{\widehat{\sigma }_{1m}}{%
\widehat{\sigma }_{1g}}\left( r_{2}-\widehat{\mu }_{1g}\right) \right)  
\nonumber \\
&=&P_{0}\exp \left( \widehat{\mu }_{1m}+\sqrt{21}\left( r_{2}-\widehat{\mu }%
_{1g}\right) \right)   \nonumber \\
&=&P_{0}\exp \left( \widehat{\mu }_{1m}-\sqrt{21}\widehat{\mu }_{1g}+\sqrt{21%
}r_{2}\right)   \nonumber \\
&=&P_{0}\exp \left( \sqrt{21}\widehat{\mu }_{1g}(\sqrt{21}-1)+\sqrt{21}%
r_{2}\right)   \label{eq5}
\end{eqnarray}
L'ultima equazione, equazione (\ref{eq5}) \`{e} la formula finale. Da questa
formula si nota che al lato pratico non \`{e} necessario calcolare $\widehat{%
\varepsilon }_{2}$ per eseguire la simulazione storica. Noi l'abbiamo
calcolato a fini didattici, cio\`{e} per meglio comprendere l'ideona
sottostante e in ultima analisi perch\'{e} e come si giunge all'equazione (%
\ref{eq5}). Come vedete, non \`{e} necessario calcolare le covarianze fra
rendimenti di strumenti diversi. Tuttavia sar\`{a} obbligatorio campionare
per tutti gli strumenti rendimenti dello stesso istante, nel nostro esempio 
cio\`{e} con $t=2.$
Un'ultima osservazione: su orizzonti brevi, cio\`{e} 1 giorno, 1 o 2
settimane \`{e} possibile porre $\widehat{\mu }_{1g}=0$ cos\`{i} da ottenere
la semplice espressione
\[
\widetilde{P}_{1m}=P_{0}\exp \left( \sqrt{21}r_{t}\right) .
\]

\item  Avendo ricalcolato tutti i prezzi dei fattori di rischio rivalutate
il portafoglio cos\`{i} da ottenere un valore simulato $\widetilde{V}_{1m}$.

\item  Ripetete il tutto 1000 volte ed utilizzate la distribuzione empirica
delle simulazioni per calcolare il quantile.
\end{enumerate}

\item  Secondo quale distribuzione discreta devono essere campionate le date?

Le date devono essere estratte casualmente campionando in maniera uniforme.
\end{enumerate}
Tornando all'esercizio avremo quindi: 
\end{enumerate}

\end{document}
