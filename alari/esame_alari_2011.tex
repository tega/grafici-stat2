%% LyX 2.0.0rc3 created this file.  For more info, see http://www.lyx.org/.
%% Do not edit unless you really know what you are doing.
\documentclass[11pt,a4paper,english]{article}
\usepackage[T1]{fontenc}
\usepackage[latin9]{inputenc}
\usepackage{amsmath}
\usepackage{amssymb}

\makeatletter

%%%%%%%%%%%%%%%%%%%%%%%%%%%%%% LyX specific LaTeX commands.
\special{papersize=\the\paperwidth,\the\paperheight}


%%%%%%%%%%%%%%%%%%%%%%%%%%%%%% User specified LaTeX commands.


%%%%%%%%%%%%%%%%%%%%%%%%%%%%%%%%%%%%%%%%%%%%%%%%%%%%%%%%%%%%%%%%%%%%%%%%%%%%%%%%%%%%%%%%%%%%%%%%%%%%%%%%%%%%%%%%%%%%%%%%%%%%%%%%%%%%%%%%%%%%%%%%%%%%%%%%%%%%%%%%%%%%%%%%%%%%%%%%%%%%%%%%%%%%%%%%%%%%%%%%%%%%%%%%%%%%%%%%%%%%%%%%%%%%%%%%%%%%%%%%%%%%%%%%%%%%
\@ifundefined{definecolor}
 {\usepackage{color}}{}
\usepackage{graphics}\usepackage{hypernat}\usepackage{lscape}

\setcounter{MaxMatrixCols}{10}
%TCIDATA{OutputFilter=LATEX.DLL}
%TCIDATA{Version=5.00.0.2606}
%TCIDATA{<META NAME="SaveForMode" CONTENT="1">}
%TCIDATA{BibliographyScheme=BibTeX}
%TCIDATA{LastRevised=Wednesday, October 14, 2009 09:57:04}
%TCIDATA{<META NAME="GraphicsSave" CONTENT="32">}
%TCIDATA{Language=American English}
%TCIDATA{CSTFile=article.cst}

\newtheorem{theorem}{Theorem}
\newtheorem{acknowledgement}[theorem]{Acknowledgement}
\newtheorem{algorithm}[theorem]{Algorithm}
\newtheorem{axiom}[theorem]{Axiom}
\newtheorem{case}[theorem]{Case}
\newtheorem{claim}[theorem]{Claim}
\newtheorem{conclusion}[theorem]{Conclusion}
\newtheorem{condition}[theorem]{Condition}
\newtheorem{conjecture}[theorem]{Conjecture}
\newtheorem{corollary}[theorem]{Corollary}
\newtheorem{criterion}[theorem]{Criterion}
\newtheorem{definition}[theorem]{Definition}
\newtheorem{example}[theorem]{Example}
\newtheorem{exercise}[theorem]{Exercise}
\newtheorem{lemma}[theorem]{Lemma}
\newtheorem{notation}[theorem]{Notation}
\newtheorem{problem}[theorem]{Problem}
\newtheorem{proposition}[theorem]{Proposition}
\newtheorem{remark}[theorem]{Remark}
\newtheorem{solution}[theorem]{Solution}
\newtheorem{summary}[theorem]{Summary}
%\numberwithin{equation}{section} 
\RequirePackage[OT1]{fontenc}
\RequirePackage[colorlinks]{hyperref} \RequirePackage{hypernat}
\RequirePackage[colorlinks]{hyperref} \setlength{\textwidth}{15.5cm}
\setlength{\textheight}{21.5cm} \setlength{\oddsidemargin}{0.1cm}
\setlength{\topmargin}{1mm}
\newenvironment{proof}[1][Proof]{\textbf{#1.}}{\rule{0.5em}{0.5em}}
\linespread{1.3}
\def\Red{\special{color cmyk 0 1. 1. 0}}
\def\MagentaSpecial{\special{color cmyk 0 1. 0.5 0.12}}
\def\White{\special{color cmyk 0 0 0 0}}
\def\Cyan{\special{color cmyk 1. 0 0 0}}
\def\ProcessBlue{\special{color cmyk 0.96 0 0 0}}
\def\SkyBlue{\special{color cmyk 0.62 0 0.12 0}}
\def\Turquoise{\special{color cmyk 0.85 0 0.20 0}}
\def\TealBlue{\special{color cmyk 0.86 0 0.34 0.02}}
\def\Black{\special{color cmyk 0 0 0 1.}}
\def\Violet{\special{color cmyk 0.79 0.88 0 0}}
\def\RoyalPurple{\special{color cmyk 0.75 0.90 0 0}}
\def\BlueViolet{\special{color cmyk 0.86 0.91 0 0.04}}
\def\Periwinkle{\special{color cmyk 0.57 0.55 0 0}}
\def\CadetBlue{\special{color cmyk 0.62 0.57 0.23 0}}
\def\CornflowerBlue{\special{color cmyk 0.65 0.13 0 0}}
\def\MidnightBlue{\special{color cmyk 0.98 0.13 0 0.43}}
\def\NavyBlue{\special{color cmyk 0.94 0.54 0 0}}
\def\RoyalBlue{\special{color cmyk 1. 0.50 0 0}}
\def\Blue{\special{color cmyk 1. 1. 0 0}}
\def\Green{\special{color cmyk 0.8 0 0.25 0.5}}
\def\Greennew{\special{color cmyk 0.7 0 0.8 0.3}}
\def\red#1{\Red#1\Black}
\newcommand{\blue}[1]{\Blue#1\Black}
\newcommand{\magentas}[1]{\MagentaSpecial#1\Black}

\makeatother

\usepackage{babel}
\begin{document}
\noindent \textsc{\small university of lugano}\hfill{}\textsc{\small faculty
of informatics}\\
\textsc{\small Prof. Claudio Ortelli}{\small \hfill{}}\textsc{\small SS
2011}{\small \par}

\noindent \textsc{\small Prof. Davide La Vecchia}{\small \hfill{}}\vspace{0.2cm}


\begin{center}
\textbf{\large Statistics}{\large{} }
\par\end{center}{\large \par}

\vspace{0.1cm}


ALaRI\ Exam \hfill{}27 June 2011

\vspace{-0.3cm}
 \hrulefill{}

\vspace{0.3cm}


\hspace{5cm}

\hspace{5cm}$\bullet$ Duration: 2 hours and 30 minutes

\hspace{5cm}$\bullet$ Open book exam

\hspace{5cm}$\bullet$ Solve all exercises

\vspace{0.3cm}



\section{Exercise 1 (20 pts)}
\begin{itemize}
\item [1.1] (\textbf{10 pts}) \textbf{Binomial Distribution:} Let $X$
be a random variable having a Binomial distribution, with parameters
$n$ and $p$.


(i) For $n=10$ and $p=0.2$, compute, $\mu_{X}:=E(X)$, $\sigma_{X}=\sqrt{Var(X)}$
and $P(X<\mu_{X}-2\sigma_{X})$;


(ii) Let us keep $n=10$ and let $p$ be any real in $[0,1]$. Find
the value of $p$ such that $\sigma_{X}^{2}$ is maximized. 

\item [1.2] (\textbf{9 pts}) \textbf{Poisson Distribution:} Let $X$ be
a random variable having a Poisson distribution such that $P(X=0)=P(X=1)$.


(i) Find $\mu_{X}:=E(X)$ and $\sigma_{X}=\sqrt{Var(X)}$; (ii) Find
an upper bound for $P(|X-\mu_{X}|\geq2\sigma_{X})$ and comment briefly
the result.

\item [1.3] (\textbf{1 pts}) \textbf{Gamma/Chi-square Distribution:} Let
$X$ be a random variable having a Gamma distribution $\Gamma(\alpha;\lambda)$,
for $\alpha=2$ and $\lambda=2$. Compute $P(X<11.1433)$. \textbf{Remark}:
\textit{In fact, a random variable having a $\Gamma$-distribution
$\Gamma(\alpha;\lambda)$, with $\lambda=2$ has a $\chi^{2}$-distribution,
with $2\alpha$ degrees-of-freedom. Thus, apply the table of the $\chi^{2}$
to solve the exercise.}
\end{itemize}

\section{Exercise 2 (80 pts)}

Let $X$ be a positive random variable, measuring the time (in years)
that a student at ALaRI needs in order to pass the exam of Statistics.
Let us assume that $X$ has an exponential distribution: 
\begin{equation}
f(x;\lambda)=\lambda\exp^{-\lambda x},\label{exp}
\end{equation}
 for $\lambda\in\mathbb{R}^{+}$ and $x\geq0$.
\begin{itemize}
\item [2.1] \textbf{(5 pts)} Compute, as a function of $\lambda$:

\begin{itemize}
\item [2.1.1] (i) $P(X>1.2)$;


(ii) $P(1.5\leq X\leq3)$;


(iii) $P(1.5<X<3)$;


(iv) $P(X\leq0.5)$;


(v) $P(X=3.55)$;


(vi) $P(X\geq2.5\vert X>2)$;

\item [2.1.2] Find $x_{\alpha}$ such that $F_{X}(x_{\alpha})=\alpha$,
for any given value of $\alpha\in[0,1]$.
\end{itemize}
\item [2.2] \textbf{(10 pts)} Estimation:

\begin{itemize}
\item [2.2.1] Compute the Maximum Likelihood Estimator (MLE) of $\lambda$
and $\frac{1}{\lambda}$;
\item [2.2.2] Derive analytically the expression for $Var(X)$. Then,
provide the MLE for $Var(X)$. Justify your answer.
\item [2.2.3] Provide the MLE of the tail area: $P(X>10)$. Justify your
answer.
\end{itemize}
\item [2.3] \textbf{(35 pts)} Now let us consider the random variable
\begin{equation}
S_{n}:=\sum_{i=1}^{n}X_{i},\label{Sn}
\end{equation}
 which is defined as the sum of $n$ i.i.d. random variables, each
having an exponential density as in Eq.(\ref{exp}), for $n$ fix.

\begin{itemize}
\item [2.3.1] What is the exact distribution (expressed as a function
of $\lambda$) of $S_{n}$? (Hint: apply the Laplace transform of
$X_{i}$).
\item [2.3.2] Provide the MLE of $P(S_{n}\geq s)$, for $s>0$. Justify
your answer.
\item [2.3.3] Using the results in 2.3.1, explain carefully how one can
define a test with level $\alpha=5\%$, for: 
\begin{eqnarray}
H_{0}:\lambda=\lambda_{0} & vs & H_{1}:\lambda>\lambda_{0}.\label{test}
\end{eqnarray}
 We assume that the value $\lambda_{0}$ is larger than zero. (Hint:
find the distribution of $S_{n}$ under $H_{0}$).
\item [2.3.4] Assume that we are given a sample of $x_{1},x_{2},...,x_{n}$
observations of $X$. For $n=50$, we obtain $s=\sum_{i=1}^{50}x_{i}=100$.
According to the test defined in the previous point 2.3.3, do you
accept the null hypothesis $H_{0}$ in (\ref{test}), for $\lambda_{0}=0.5$?
\end{itemize}
\item [2.4] \textbf{(30 pts)} Now let us consider the new random variable:
\begin{equation}
\bar{X}_{n}:=\frac{1}{n}S_{n}.\label{Xn}
\end{equation}


\begin{itemize}
\item [2.4.1] Provide the expression for $E(\bar{X}_{n})$ and $Var(\bar{X}_{n})$;
\item [2.4.2] Consider also the random variable: 
\begin{equation}
Z_{n}:=\frac{\bar{X}_{n}-E(\bar{X}_{n})}{\sqrt{Var(\bar{X}_{n})}}.
\end{equation}
 (i) What is the distribution of $Z_{n}$, for $n$ large?


(ii) Define $Y:=Z_{n}^{2}$. What is the distribution of $Y$, when
$n$ is large?


(iii) Explain how one can compute: $P(Z_{n}>z)$ and $P(Y>z^{2})$,
for $z\in\mathbb{R}^{+}$ and $n$ large.

\item [2.4.3] For $n=1000$, provide an approximation to the distribution
(expressed as a function of $\lambda$) of $S_{n}=n\bar{X}_{n}$;
see Eq.(\ref{Xn}). Compare this result to the exact distribution
of $S_{n}$, as derived in the question 2.3.1, and comment this finding.
\end{itemize}
\end{itemize}

\section{Exercise 3 (40 pts)}

Suppose we have a random sample $X_{1}$, ...,$X_{n}$ from an exponential
distribution with unknown parameter $\lambda$, i.e. $X_{i}$ are
$i.i.d.$ $\sim$ $Exp(\lambda)$. Suppose we want to estimate $\frac{1}{\lambda}$.
Let us define $M_{n}=\min(X_{1},...,X_{n})$.
\begin{itemize}
\item [3.1] \textbf{(8 pts)} Show that the estimator $T_{1}=\bar{X}_{n}=(X_{1}+...+X_{n})/n$
is an unbiased estimator of $1/\lambda$.
\item [3.2] \textbf{(8 pts)} Derive and explain in details the following
formula: $P(M_{n}\leq x)=1-\prod_{i=1}^{n}\left(1-P(X_{i}\leq x)\right)$.
\item [3.3] \textbf{(8 pts)} Show that $M_{n}\sim Exp(n\lambda$). Give
detailed explanation.
\item [3.4] \textbf{(8 pts)} Show that $T_{2}=nM_{n}$ is an unbiased
estimator of $1/\lambda$.
\item [3.5] \textbf{(8 pts)} Which of the estimators $T_{1}$ and $T_{2}$
would you choose for estimating the mean $1/\lambda$? Substantiate
your answer. 
\end{itemize}

\section{Exercise 4 (30 pts)}

Someone is proposing two unbiased estimators $U$ and $V$, with the
same variance $Var(U)=Var(V)$. It therefore appears that we would
not prefer one estimator over the other. However, we could go for
a third estimator, namely $W=(U+V)/2$.
\begin{itemize}
\item [4.1] \textbf{(5 pts)} Show that $W$ is unbiased.
\end{itemize}
To judge the quality of $W$ we want to compute its variance. Lacking
information on the joint probability distribution of $U$ and $V$,
this is impossible. However, we should prefer $W$ in any case! To
see this, 
\begin{itemize}
\item [4.2] \textbf{(15 pts)} Show by means of the variance-of-the-sum
rule that the relative efficiency of $U$ with respect to $W$, is
equal to 
\[
\frac{Var\left((U+V)/2\right)}{Var(U)}=\frac{1}{2}+\frac{1}{2}\rho(U,V).
\]
Here $\rho(U,V)=\frac{Cov(U,V)}{\sqrt{Var(U)\, Var(V)}}$ is the correlation
coefficient.
\item [4.3] \textbf{(10 pts)} Why does this result imply that we should
use $W$ instead of $U$ (or $V$ )? 
\end{itemize}

\section{Exercise 5 (30 pts)}

We model the delivery time $y$ of a given web service as a linear
regression model (without intercept) where the deterministic variable
$x$ denotes the size in megabytes of the input:

\[
y_{i}=\theta x_{i}+\epsilon_{i}\quad\mbox{ for }i=1,2,...,n.
\]
As usual, the $\epsilon_{i}$ here are independent random variables
with $E[\epsilon_{i}]=0$, and $Var(\epsilon_{i})=\sigma^{2}$ . We
consider three estimators for the slope $\theta$ of the line $y=\theta x$: 
\begin{enumerate}
\item the least squares estimator 
\[
T_{1}=\frac{\sum_{i=1}^{n}x_{i}y_{i}}{\sum_{i=1}^{n}x_{i}^{2}};
\]

\item the average slope estimator 
\[
T_{2}=\frac{1}{n}\sum_{i=1}^{n}\frac{y_{i}}{x_{i}};
\]
 
\item the slope of the averages estimator 
\[
T_{3}=\frac{\sum_{i=1}^{n}y_{i}}{\sum_{i=1}^{n}x_{i}}.
\]
 \end{enumerate}
\begin{itemize}
\item [5.1] \textbf{(8 pts)} Show that all estimators are linear estimators,
i.e. they can be written as 
\[
\sum_{i=1}^{n}a_{i}y_{i}.
\]
Determine the weights $a_{i}$ for all three estimators.
\item [5.2] \textbf{(7 pts)} Which one of the three estimators are unbiased
estimators of $\theta$? Show the calculations.
\item [5.3] \textbf{(10 pts)} Compute the variance of each estimator.
\item [5.4] \textbf{(5 pts)} Which estimator is to be preferred? Why?\end{itemize}

\end{document}
