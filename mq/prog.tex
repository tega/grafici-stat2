%% LyX 2.1.2 created this file.  For more info, see http://www.lyx.org/.
%% Do not edit unless you really know what you are doing.
\documentclass[12pt,a4paper,english]{article}
\usepackage[T1]{fontenc}
\usepackage[latin9]{inputenc}
\pagestyle{empty}

\makeatletter

%%%%%%%%%%%%%%%%%%%%%%%%%%%%%% LyX specific LaTeX commands.
\special{papersize=\the\paperwidth,\the\paperheight}


%%%%%%%%%%%%%%%%%%%%%%%%%%%%%% User specified LaTeX commands.

%%%%%%%%%%%%%%%%%%%%%%%%%%%%%%%%%%%%%%%%%%%%%%%%%%%%%%%%%%%%%%%%%%%%%%%%%%%%%%%%%%%%%%%%%%%%%%%%%%%%%%%%%%%%%%%%%%%%%%%%%%%%%%%%%%%%%%%%%%%%%%%%%%%%%%%%%%%%%%%%%%%%%%%%%%%%%%%%%%%%%%%%%%%%%%%%%%%%%%%%%%%%%%%%%%%%%%%%%%%%%%%%%%%%%%%%%%%%%%%%%%%%%%%%%%%%
%TCIDATA{OutputFilter=LATEX.DLL}
%TCIDATA{Version=5.50.0.2953}
%TCIDATA{<META NAME="SaveForMode" CONTENT="1">}
%TCIDATA{BibliographyScheme=Manual}
%TCIDATA{LastRevised=Wednesday, February 17, 2010 14:24:35}
%TCIDATA{<META NAME="GraphicsSave" CONTENT="32">}
\topmargin-0.7cm\headheight0cm\headsep0cm\textheight26cm\parindent0cm

\makeatother

\usepackage{babel}
\begin{document}
\textsc{\small{}universit� della svizzera italiana}\hfill{}\textsc{\small{}facolt�
di scienze economiche}\\
 \textsc{\small{}claudio ortelli}\hfill{}\textsc{\small{}semestre
autunnale}{\small \par}

\vspace{0.4cm}


\begin{center}
\textbf{\large{ECONOMETRIA}}
\par\end{center}

\vspace{0.1cm}

\begin{itemize}
\item Assistente: Mirco Rubin 
\end{itemize}
\begin{center}
\vspace{0.4cm}

\par\end{center}

\begin{center}
\textbf{\large{LA STIMA DEI MODELLI ECONOMETRICI}} \\[0pt]
\textbf{\large{(al di l� della regressione classica)}} 
\par\end{center}

\vspace{1cm}

\underline{Complementi alla regressione classica:}
\begin{enumerate}
\item [1] La regressione scomposta 
\end{enumerate}

\noindent \underline{Parte prima: La regressione generalizzata}

\begin{enumerate}
\item [2] Il modello generale
\begin{enumerate}
\item [2.1] Introduzione
\begin{itemize}
\item Breve ritorno sulla regressione classica
\item Modelli econometrici che non soddisfano le ipotesi classiche 
\end{itemize}
\item [2.2] Il modello e le ipotesi
\item [2.3] Quando la matrice delle varianze-covarianze � conosciuta:
i minimi quadrati generalizzati
\begin{itemize}
\item Gli stimatori e le loro propriet�
\item L'induzione statistica
\item La previsione 
\end{itemize}
\item [2.4] Le conseguenze dell'applicazione dei minimi quadrati ordinari

\begin{itemize}
\item Correttezza e perdita di efficienza
\item Stima convergente della varianza dello stimatore dei minimi quadrati
ordinari
\item Casi in cui i minimi quadrati ordinari ed i minimi quadrati generalizzati
sono equivalenti 
\end{itemize}
\item [2.5] Quando la matrice delle varianze-covarianze dipende da parametri
sconosciuti

\begin{itemize}
\item I minimi quadrati generalizzati a due tappe
\item Il massimo di verosimiglianza\newpage{} 
\end{itemize}
\end{enumerate}
\item [3] L'autocorrelazione
\begin{enumerate}
\item [3.1] Il modello markoviano di primo ordine
\begin{itemize}
\item Ipotesi e propriet�
\item Il test dell'autocorrelazione
\item I metodi di stima
\item La previsione 
\end{itemize}
\item [3.2] Altre forme d'autocorrelazione 
\end{enumerate}
\end{enumerate}
\bigskip{}

\underline{Parte seconda: La regressione stocastica}

\begin{enumerate}
\item [4] Elementi di analisi asintotica
\begin{enumerate}
\item [4.1] Convergenza in probabilit�
\begin{itemize}
\item Definizione e propriet�
\item Convergenza in media quadratica
\item Convergenza dei momenti del campione 
\end{itemize}
\item [4.2] Convergenza in distribuzione
\begin{itemize}
\item Un esempio
\item Teorema del limite centrale 
\end{itemize}
\item [4.3] Analisi asintotica della regressione classica 
\end{enumerate}
\item [5] Il modello di regressione stocastica
\begin{enumerate}
\item [5.1] Il modello e le ipotesi
\item [5.2] Quando i regressori e gli errori sono indipendenti
\item [5.3] Quando sono correlati
\begin{itemize}
\item La non convergenza degli stimatori dei minimi quadrati
\item Il metodo delle variabili strumentali
\item La scelta degli strumenti 
\end{itemize}
\end{enumerate}
\end{enumerate}
\bigskip{}


\underline{Parte terza: Modelli Dinamici}

\begin{enumerate}
\item [6] Variabili endogene ritardate
\begin{enumerate}
\item [6.1] Esempi economici
\begin{enumerate}
\item [6.1.1] La formazione di abitudini: effetti immediati ed effetti
di lungo periodo
\item [6.1.2] Il modello d'aggiustamento parziale
\item [6.1.3] Il modello ad aspettative adattive 
\end{enumerate}
\item [6.2.] La stima dei modelli a variabili endogene ritardate
\begin{enumerate}
\item [6.2.1] Errori indipendenti
\item [6.2.2] Quando gli errori sono autocorrelati
\item [6.2.3] Il test di autocorrelazione
\item [6.2.4] Stima convergente ed asintoticamente efficiente 
\end{enumerate}
\end{enumerate}
\item [7] Modelli con ritardi distribuiti
\begin{enumerate}
\item [7.1] Specificazione e propriet�. Moltiplicatori d'impatto, moltiplicatori
dinamici, ritardo medio
\begin{enumerate}
\item [7.1.1] Specificazione 
\end{enumerate}
\item [7.2] Distribuzione dei ritardi finita
\begin{enumerate}
\item [7.2.1] Il caso generale
\item [7.2.2] Distribuzione di Almon 
\end{enumerate}
\item [7.3] Distribuzione dei ritardi infinita
\begin{enumerate}
\item [7.3.1] Il modello
\item [7.3.2] Distribuzione razionale dei ritardi: idea introduttiva
\item [7.3.3] L'operatore di ritardo
\item [7.3.4] Il modello dinamico di Jorgenson 
\end{enumerate}
\end{enumerate}
\item [8] Serie macroeconomiche non stazionarie
\begin{enumerate}
\item [8.1] Stazionariet� e non stazionariet�
\begin{enumerate}
\item [8.1.1] Stazionariet�
\item [8.1.2] Non stazionariet�
\item [8.1.3] Una riparametrizzazione del modello dinamico classico (Jorgenson)
\item [8.1.4] Introduzione alla cointegrazione 
\end{enumerate}
\end{enumerate}
\end{enumerate}
\vspace{1.5cm}


\underline{Testi di riferimento}
\begin{itemize}
\item BALTAGI, Badi H., Econometrics, Springer, 1998
\item GREENE, W.H., Econometric Analysis, 3.rd Ed., Prentice Hall, 1997
\item HAYASHI, F., Econometrics, Princeton University Press, 2000
\item MADDALA, G.S., Introduction to Econometrics, 2.nd Ed., Prentice Hall,
1992 \end{itemize}

\end{document}
