
\documentclass[a4paper]{report}
%%%%%%%%%%%%%%%%%%%%%%%%%%%%%%%%%%%%%%%%%%%%%%%%%%%%%%%%%%%%%%%%%%%%%%%%%%%%%%%%%%%%%%%%%%%%%%%%%%%%%%%%%%%%%%%%%%%%%%%%%%%%%%%%%%%%%%%%%%%%%%%%%%%%%%%%%%%%%%%%%%%%%%%%%%%%%%%%%%%%%%%%%%%%%%%%%%%%%%%%%%%%%%%%%%%%%%%%%%%%%%%%%%%%%%%%%%%%%%%%%%%%%%%%%%%%
\usepackage{amssymb}
\usepackage[dvips]{graphicx}
\usepackage[italian]{babel}
\usepackage{amsmath}
\usepackage{amsthm}
\usepackage{multicol}

\setcounter{MaxMatrixCols}{10}
%TCIDATA{OutputFilter=Latex.dll}
%TCIDATA{Version=5.50.0.2953}
%TCIDATA{<META NAME="SaveForMode" CONTENT="1">}
%TCIDATA{BibliographyScheme=Manual}
%TCIDATA{LastRevised=Wednesday, February 17, 2010 14:36:14}
%TCIDATA{<META NAME="GraphicsSave" CONTENT="32">}

\newcounter{ese}
\newcounter{def}
\theoremstyle{remark} 
\newtheorem{esempio}[ese]{Esempio}
\newtheorem{definizione}[def]{Definizione}
\newtheorem{remark}{Osservazione}
\newcounter{theor}
\newtheorem{theorem}[theor]{Teorema}

\begin{document}

\title{Metodi Quantitativi\\
Appunti del corso}
\author{}
\date{Semestre Estivo\\
2010}
\maketitle
\tableofcontents

\part{\protect\bigskip Complementi al modello classico di regressione}

\chapter{La regressione scomposta}

\section{Il problema}

Nel quadro della regressione classica%
\begin{equation*}
y=X\beta +\varepsilon
\end{equation*}%
con $n$ osservazioni e $K$ variabili esplicative scomponiamo le variabili
esplicative ed i loro coefficienti in 2 gruppi:

\begin{itemize}
\item la matrice delle variabili esplicative%
\begin{equation*}
X=\left[ 
\begin{tabular}{l|l}
$\underset{\left( n\times K_{1}\right) }{X_{1}}$ & $\underset{\left( n\times
K_{2}\right) }{X_{2}}$%
\end{tabular}%
\right]
\end{equation*}

\item il vettore dei coefficienti%
\begin{equation*}
\beta =\left[ 
\begin{array}{c}
\beta _{1} \\ 
\beta _{2}%
\end{array}%
\right] 
\begin{array}{c}
\left( K_{1}\times 1\right) \\ 
\left( K_{2}\times 1\right)%
\end{array}%
\end{equation*}
\end{itemize}

\noindent Naturalmente, lo stimatore di $\beta $ \`{e} lo stimatore abituale
dei m.q.%
\begin{equation*}
\widehat{\beta }=\left( X^{\prime }X\right) ^{-1}X^{\prime }y.
\end{equation*}%
Supponiamo di essere interessati a $\beta _{2}$ (vettore $K_{2}\times 1)$.
Evidentemente $\widehat{\beta }_{2}$ \`{e} la seconda parte del vettore $%
\widehat{\beta }$. Come esprimerlo analiticamente?

\section{Lo stimatore}

Sappiamo che $\widehat{\beta }$ \`{e} la soluzione del sistema%
\begin{equation*}
X^{\prime }X\ \widehat{\beta }=X^{\prime }y.
\end{equation*}%
In forma scomposta possiamo riscrivere il precedente sistema come%
\begin{eqnarray*}
\left[ 
\begin{array}{c}
X_{1}^{\prime } \\ 
X_{2}^{\prime }%
\end{array}%
\right] \left[ 
\begin{array}{cc}
X_{1} & X_{2}%
\end{array}%
\right] \left[ 
\begin{array}{c}
\widehat{\beta }_{1} \\ 
\widehat{\beta }_{2}%
\end{array}%
\right] &=&\left[ 
\begin{array}{c}
X_{1}^{\prime } \\ 
X_{2}^{\prime }%
\end{array}%
\right] y \\
\left[ 
\begin{array}{cc}
X_{1}^{\prime }X_{1} & X_{1}^{\prime }X_{2} \\ 
X_{2}^{\prime }X_{1} & X_{2}^{\prime }X_{2}%
\end{array}%
\right] \left[ 
\begin{array}{c}
\widehat{\beta }_{1} \\ 
\widehat{\beta }_{2}%
\end{array}%
\right] &=&\left[ 
\begin{array}{c}
X_{1}^{\prime }y \\ 
X_{2}^{\prime }y%
\end{array}%
\right]
\end{eqnarray*}%
\begin{eqnarray}
X_{1}^{\prime }X_{1}\widehat{\beta }_{1}+X_{1}^{\prime }X_{2}\widehat{\beta }%
_{2} &=&X_{1}^{\prime }y  \label{sistema 1} \\
X_{2}^{\prime }X_{1}\widehat{\beta }_{1}+X_{2}^{\prime }X_{2}\widehat{\beta }%
_{2} &=&X_{2}^{\prime }y  \label{sistema 2}
\end{eqnarray}%
Poich\'{e} per ipotesi $rg(X)=K\Leftrightarrow $ le $K$ colonne di $X$ sono
linearmente indipendenti, anche le $K_{1}$ colonne di $X_{1}$ devono essere
necessariamente linearmenti indipendenti. Ci\`{o} significa che la matrice
quadrata%
\begin{equation*}
\underset{\left( K_{1}\times K_{1}\right) }{X_{1}^{\prime }X_{1}}
\end{equation*}%
\`{e} non singolare (invertibile). Dalla (\ref{sistema 1}) risolvendo per $%
\widehat{\beta }_{1}$ otteniamo:%
\begin{equation}
\widehat{\beta }_{1}=\left( X_{1}^{\prime }X_{1}\right) ^{-1}X_{1}^{\prime
}\left( y-X_{2}\widehat{\beta }_{2}\right)  \label{sistema 3}
\end{equation}%
Inserendo la (\ref{sistema 3}) nella (\ref{sistema 2}):%
\begin{equation*}
X_{2}^{\prime }X_{1}\left( X_{1}^{\prime }X_{1}\right) ^{-1}X_{1}^{\prime
}\left( y-X_{2}\widehat{\beta }_{2}\right) +X_{2}^{\prime }X_{2}\widehat{%
\beta }_{2}=X_{2}^{\prime }y
\end{equation*}%
\begin{eqnarray}
\left( X_{2}^{\prime }X_{2}-X_{2}^{\prime }X_{1}\left( X_{1}^{\prime
}X_{1}\right) ^{-1}X_{1}^{\prime }X_{2}\right) \widehat{\beta }_{2}
&=&X_{2}^{\prime }y-X_{2}^{\prime }X_{1}\left( X_{1}^{\prime }X_{1}\right)
^{-1}X_{1}^{\prime }y  \notag \\
X_{2}^{\prime }\left( I-X_{1}\left( X_{1}^{\prime }X_{1}\right)
^{-1}X_{1}^{\prime }\right) X_{2}\widehat{\beta }_{2} &=&X_{2}^{\prime
}\left( I-X_{1}\left( X_{1}^{\prime }X_{1}\right) ^{-1}X_{1}^{\prime
}\right) y  \notag \\
X_{2}^{\prime }\ M_{1}\ X_{2}\widehat{\beta }_{2} &=&X_{2}^{\prime }\ M_{1}\
y.  \label{sistema 3bis}
\end{eqnarray}%
Osservazioni sulla matrice $M_{1}$

\begin{itemize}
\item $M_{1}$ \`{e} della stessa forma e propriet\`{a} che la matrice%
\begin{equation*}
M=I-X\left( X^{\prime }X\right) ^{-1}X^{\prime }
\end{equation*}

\item soltanto che essa \`{e} formata a partire da $X_{1}$ e non da $X$.
Come per $M$ si dimostra facilmente che%
\begin{equation*}
\begin{tabular}{l}
$\bullet \ M_{1}^{\prime }=M_{1}$ e $M_{1}M_{1}=M_{1}\text{ (\`{e}
simmetrica e idempotente)}$ \\ 
$\bullet \ rg(M_{1})=tr(M_{1})=n-K_{1}$ \\ 
$\bullet \ M_{1}X_{1}=0$%
\end{tabular}%
\end{equation*}
\end{itemize}

\noindent Quando la matrice $X$ \`{e} di rango $K$, la matrice%
\begin{equation*}
\underset{\left( K_{2}\times K_{2}\right) }{X_{2}^{\prime }M_{1}X_{2}}
\end{equation*}%
\`{e} non singolare\footnote{%
Dimostrazione:\ Supponiamo che $rg(X_{2}^{\prime }M_{1}X_{2})<K_{2}$. Questo
significa che il nucleo di $X_{2}^{\prime }M_{1}X_{2}$ non \`{e} vuoto e cio%
\`{e} che esiste un vettore $z\in \mathbb{R}^{K_{2}}$ diverso dal vettore
nullo tale per cui 
\begin{equation*}
X_{2}^{\prime }M_{1}X_{2}z=0.
\end{equation*}%
Premoltiplicando la seguente uguaglianza per $y^{\ast \prime }$ otteniamo
una, utilizzando la propriet\`{a} di idempotenza di $M_{1}$, la seguente
forma quadratica%
\begin{eqnarray*}
\underset{\widetilde{z}^{\prime }}{\underbrace{z^{\prime }X_{2}^{\prime
}M_{1}}}\underset{\widetilde{z}}{\ \underbrace{M_{1}X_{2}z}} &=&0 \\
\widetilde{z}^{\prime }\widetilde{z} &=&0
\end{eqnarray*}%
Ma $\widetilde{z}^{\prime }\widetilde{z}=0\Leftrightarrow \widetilde{z}=0$,
da cui deduciamo che $M_{1}X_{2}z=0$, ovvero%
\begin{eqnarray*}
X_{2}z-X_{1}\left( X_{1}^{\prime }X_{1}\right) ^{-1}X_{1}^{\prime }X_{2}z
&=&0 \\
\left[ 
\begin{array}{cc}
X_{1} & X_{2}%
\end{array}%
\right] \underset{y^{\ast }\neq 0}{\underbrace{\left[ 
\begin{array}{c}
-\left( X_{1}^{\prime }X_{1}\right) ^{-1}X_{1}^{\prime }X_{2}z \\ 
z%
\end{array}%
\right] }} &=&0 \\
X\ y^{\ast } &=&0
\end{eqnarray*}%
$y^{\ast }\in \mathbb{R}^{K}$ \`{e} diverso dal vettore nullo poich\'{e} $%
z\neq 0$. Tuttavia $X\ y^{\ast }=0$ \`{e} in contraddizione con l'ipotesi
che $Rg(X)=K$.
\par
{}} (lo deve essere necessariamente altrimenti il sistema non ammetterebbe
una soluzione unica e ci\`{o} sarebbe in contraddizione con $\widehat{\beta }%
=\left( X^{\prime }X\right) ^{-1}X^{\prime }y$).

\noindent Quindi, tornando all'equazione (\ref{sistema 3bis}) abbiamo%
\begin{eqnarray}
X_{2}^{\prime }M_{1}X_{2}\widehat{\beta }_{2} &=&X_{2}^{\prime }M_{1}y. 
\notag \\
\widehat{\beta }_{2} &=&\left( X_{2}^{\prime }M_{1}X_{2}\right)
^{-1}X_{2}^{\prime }M_{1}y  \label{sistema 4}
\end{eqnarray}%
Inoltre:%
\begin{eqnarray*}
V(\widehat{\beta }_{2}) &=&\left( X_{2}^{\prime }M_{1}X_{2}\right)
^{-1}X_{2}^{\prime }M_{1}\underset{\sigma ^{2}I}{\underbrace{V(y)}}%
M_{1}X_{2}\left( X_{2}^{\prime }M_{1}X_{2}\right) ^{-1} \\
&=&\sigma ^{2}\left( X_{2}^{\prime }M_{1}X_{2}\right) ^{-1}
\end{eqnarray*}%
Si tratta quest'ultimo del blocco in posizione $(2,2)$ della matrice $\left(
X^{\prime }X\right) ^{-1}$. Abbiamo di fatto "solo" riscritto $V(\widehat{%
\beta }_{2})$ in funzione di $X_{2}$ e $M_{1}$%
\begin{equation*}
V(\widehat{\beta })=\sigma ^{2}\left( X^{\prime }X\right) ^{-1}=\left[ 
\begin{array}{cc}
V(\widehat{\beta }_{1}) & \ldots \\ 
\ldots & V(\widehat{\beta }_{2})%
\end{array}%
\right] .
\end{equation*}%
Punto sulla situazione:

\begin{itemize}
\item \`{E} possibile riscrivere il modello di regressione%
\begin{equation*}
y=X\beta +\varepsilon
\end{equation*}

a blocchi rispetto alle variabili esplicative di nostro interesse:%
\begin{equation*}
y=X_{1}\beta _{1}+X_{2}\beta _{2}+\varepsilon
\end{equation*}

\item \`{E} possibile esprimere il coefficiente $\widehat{\beta }_{2}$ dei
m.q.o. utilizzando $X_{2}$ e la matrice $M_{1}$ come%
\begin{equation*}
\widehat{\beta }_{2}=\left( X_{2}^{\prime }M_{1}X_{2}\right)
^{-1}X_{2}^{\prime }M_{1}y
\end{equation*}
\end{itemize}

\noindent Osservazione: abbiamo solo esplicitato (riscritto) $\widehat{\beta 
}_{2}$ in altro modo!

\section{Il modello ridotto}

Lo stimatore dei m.q. dei coefficienti di nostro interesse, $\widehat{\beta }%
_{2}$, pu\`{o} essere ottenuto direttamente utilizzando i m.q. su un modello
ridotto (o trasformato). Come?

\begin{enumerate}
\item Si modifica\ (trasforma) il modello di partenza%
\begin{equation*}
y=X_{1}\beta _{1}+X_{2}\beta _{2}+\varepsilon
\end{equation*}

premoltiplicando per $M_{1}$%
\begin{eqnarray}
M_{1}y &=&\underset{=0}{\underbrace{M_{1}X_{1}}}\beta _{1}+M_{1}X_{2}\beta
_{2}+M_{1}\varepsilon  \notag \\
\underset{\left( n\times n\right) \left( n\times 1\right) }{M_{1}y} &=&%
\underset{\left( n\times n\right) \left( n\times K_{2}\right) }{M_{1}X_{2}}%
\beta _{2}+M_{1}\varepsilon  \notag \\
\widetilde{y} &=&\widetilde{X}_{2}\beta _{2}+\widetilde{\varepsilon }
\label{sistema 5}
\end{eqnarray}

\item Si applicano i minimi quadrati sul modello trasformato%
\begin{eqnarray*}
\widehat{\beta }_{2} &=&\left( \widetilde{X}_{2}^{\prime }\widetilde{X}%
_{2}\right) ^{-1}\widetilde{X}_{2}^{\prime }\widetilde{y} \\
&=&\left( X_{2}^{\prime }M_{1}M_{1}X_{2}\right) ^{-1}X_{2}^{\prime
}M_{1}M_{1}y \\
&=&\left( X_{2}^{\prime }M_{1}X_{2}\right) ^{-1}X_{2}^{\prime }M_{1}y
\end{eqnarray*}%
ottenendo cos\`{\i} esattamente la formula (\ref{sistema 4}) precedente per $%
\widehat{\beta }_{2}$.
\end{enumerate}

\noindent Analizziamo i residui stimati di questa regressione:%
\begin{equation*}
\widehat{\widetilde{\varepsilon }}=\widetilde{y}-\widetilde{X}_{2}\widehat{%
\beta }_{2}.
\end{equation*}%
Questi residui sono identici ai residui%
\begin{equation*}
\widehat{\varepsilon }=y-X\widehat{\beta }
\end{equation*}%
della regressione del modello di partenza $y=X\beta +\varepsilon $. Infatti,%
\begin{eqnarray*}
\widehat{\varepsilon } &=&y-X\widehat{\beta }=y-X_{1}\widehat{\beta }%
_{1}-X_{2}\widehat{\beta }_{2} \\
&=&y-X_{1}\underset{=\widehat{\beta }_{1}\text{, cf. formula (\ref{sistema 3}%
)}}{\underbrace{\left( X_{1}^{\prime }X_{1}\right) ^{-1}X_{1}^{\prime
}\left( y-X_{2}\widehat{\beta }_{2}\right) }}-X_{2}\widehat{\beta }_{2} \\
&=&\left( I-X_{1}\left( X_{1}^{\prime }X_{1}\right) ^{-1}X_{1}^{\prime
}\right) \left( y-X_{2}\widehat{\beta }_{2}\right) \\
&=&M_{1}\left( y-X_{2}\widehat{\beta }_{2}\right) =\widetilde{y}-\widetilde{X%
}_{2}\widehat{\beta }_{2}=\widehat{\widetilde{\varepsilon }}
\end{eqnarray*}%
Dall'uguaglianza degli errori stimati deduciamo che $\widetilde{SS}=\widehat{%
\widetilde{\varepsilon }}^{\prime }\widehat{\widetilde{\varepsilon }}$ e $SS=%
\widehat{\varepsilon }^{\prime }\widehat{\varepsilon }$ sono pure identici.

\bigskip

\noindent Attenzione, consiglio pratico: se si stima direttamente il modello
trasformato (\ref{sistema 5}) al computer, utilizzando quindi $\widetilde{y}$
quale variabile dipendente e $\widetilde{X}_{2}$ quale matrice delle
variabili esplicative, occorre modificare la matrice stimata di
varianza-covarianza di $\widehat{\beta }_{2}$ in quanto il programma non pu%
\`{o} sapere che $K_{1}$ gradi di libert\`{a} sono stati persi a causa della
trasformazione $M_{1}$ (che \`{e} di rango $n-K_{1}$). Il programma calcoler%
\`{a} 
\begin{equation*}
\widetilde{SS}=\widehat{\widetilde{\varepsilon }}^{\prime }\widehat{%
\widetilde{\varepsilon }}
\end{equation*}%
che come notato corrisponde al giusto $SS$,%
\begin{equation*}
\widehat{\beta }_{2}=\left( X_{2}^{\prime }M_{1}X_{2}\right)
^{-1}X_{2}^{\prime }M_{1}y
\end{equation*}%
che corrisponde al giusto $\widehat{\beta }_{2}$. Ma per la stima di $\sigma
^{2}$ il programma dar\`{a}%
\begin{equation*}
\widehat{\sigma }_{falso}^{2}=\frac{\widetilde{SS}}{n-K_{2}}
\end{equation*}%
mentre il vero (giusto) stimatore \`{e}:%
\begin{equation*}
\widehat{\sigma }_{vero}^{2}=\frac{SS}{n-K}=\frac{\widetilde{SS}}{%
n-K_{1}-K_{2}}.
\end{equation*}%
Ne segue che occorre aggiustare l'output del programma in maniera tale da
correggere i gradi di libert\`{a}%
\begin{equation*}
\widehat{\sigma }^{2}=\widehat{\sigma }_{vero}^{2}=\frac{\widetilde{SS}}{%
n-K_{1}-K_{2}}=\frac{n-K_{2}}{n-K}\widehat{\sigma }_{falso}^{2}.
\end{equation*}%
In maniera analoga,%
\begin{equation*}
\widehat{V}_{falso}(\widehat{\beta }_{2})=\widehat{\sigma }%
_{falso}^{2}\left( X_{2}^{\prime }M_{1}X_{2}\right) ^{-1}
\end{equation*}%
andr\`{a} corretto con%
\begin{eqnarray*}
\widehat{V}(\widehat{\beta }_{2}) &=&\widehat{\sigma }_{vero}^{2}\left(
X_{2}^{\prime }M_{1}X_{2}\right) ^{-1} \\
&=&\frac{n-K_{2}}{n-K}\widehat{\sigma }_{falso}^{2}\left( X_{2}^{\prime
}M_{1}X_{2}\right) ^{-1} \\
&=&\frac{n-K_{2}}{n-K}\widehat{V}_{falso}(\widehat{\beta }_{2})
\end{eqnarray*}

\noindent Regola: i risultati per $\widetilde{SS}$ e $\widehat{\beta }_{2}$
del modello trasformato sono Ok, quelli per $\widehat{\sigma }^{2}$ e $%
\widehat{V}(\widehat{\beta }_{2})$ vanno corretti!

\section{Interpretazione della trasformazione}

\begin{equation*}
\widetilde{y}=M_{1}y=y-\underset{\widehat{\beta }_{1}\text{ della
regressione }y=X_{1}\beta _{1}\ +\ \text{errore}}{X_{1}\underbrace{\left(
X_{1}^{\prime }X_{1}\right) ^{-1}X_{1}^{\prime }y}}
\end{equation*}%
Ora $\left( X_{1}^{\prime }X_{1}\right) ^{-1}X_{1}^{\prime }y$ \`{e} lo
stimatore dei minimi quadrati del modello parziale (incompleto) $%
y=X_{1}\beta _{1}+$ errore.

\begin{itemize}
\item Si ha dunque che $\widetilde{y}$ \`{e} l'errore stimato del modello
parziale, o in altre parole, $y$ "epurato" dall'influenza parziale di $X_{1}$%
:%
\begin{equation*}
y-\text{ "la parte di }y\text{ che \`{e} spiegabile da }X_{1}\text{".}
\end{equation*}

\item La stessa interpretazione vale per ogni colonna di $X_{2}$:%
\begin{equation*}
M_{1}X_{2}=\left[ 
\begin{tabular}{l|l|l|l}
$M_{1}\left( X_{2}\right) _{\cdot ,1}$ & $M_{1}\left( X_{2}\right) _{\cdot
,2}$ & $\ldots $ & $M_{1}\left( X_{2}\right) _{\cdot ,K_{2}}$%
\end{tabular}%
\right]
\end{equation*}%
Ad ogni variabile esplicativa in $X_{2}$ (ad ogni sua colonna) tolgo la
parte "spiegata" da $X_{1}$. $\Rightarrow $ nel modello trasformato, la
variabile endogena $\widetilde{y}$ e i regressori $\widetilde{X}_{2}$ sono
ottenuti depurando le variabili originali $y$ e $X_{2}$ dalla parte spiegata
da $X_{1}$.
\end{itemize}

\noindent Domanda: Quando lo stimatore di $\beta _{1}$ sul modello parziale
(incompleto) ci d\`{a} il giusto stimatore $\widehat{\beta }_{1}$?

\bigskip

\noindent Risposta:

\begin{enumerate}
\item[a)] Abbiamo precedentemente osservato che il giusto stimatore pu\`{o}
essere scritto (cf. (\ref{sistema 3}))%
\begin{eqnarray*}
\widehat{\beta }_{1} &=&\left( X_{1}^{\prime }X_{1}\right)
^{-1}X_{1}^{\prime }\left( y-X_{2}\widehat{\beta }_{2}\right) \\
&=&\underset{\text{Stimatore di }\widehat{\beta }_{1}\text{\ sul modello
parziale}}{\underbrace{\left( X_{1}^{\prime }X_{1}\right) ^{-1}X_{1}^{\prime
}y}}-\ \ \ \underset{\text{Fattore d'aggiustamento}}{\underbrace{\left(
X_{1}^{\prime }X_{1}\right) ^{-1}X_{1}^{\prime }X_{2}\widehat{\beta }_{2}}}
\end{eqnarray*}

\item[b)] Quando le colonne di $X_{1}$ sono ortogonali a quelle di $X_{2}$
(e quindi $X_{1}^{\prime }X_{2}=0$) i due stimatori sono identici. Infatti 
\begin{eqnarray*}
\widehat{\beta } &=&\left( X^{\prime }X\right) ^{-1}X^{\prime }y=\underset{%
\text{Matrice blocco diag. se }X_{1}^{\prime }X_{2}=0}{\underbrace{\left[ 
\begin{array}{cc}
X_{1}^{\prime }X_{1} & X_{1}^{\prime }X_{2} \\ 
X_{2}^{\prime }X_{1} & X_{2}^{\prime }X_{2}%
\end{array}%
\right] ^{-1}}}\left[ 
\begin{array}{c}
X_{1}^{\prime }y \\ 
X_{2}^{\prime }y%
\end{array}%
\right] \\
&=&\left[ 
\begin{array}{c}
\left( X_{1}^{\prime }X_{1}\right) ^{-1}X_{1}^{\prime }y \\ 
\left( X_{2}^{\prime }X_{2}\right) ^{-1}X_{2}^{\prime }y%
\end{array}%
\right] \leftarrow \text{ stimatori del modello parziale.}
\end{eqnarray*}
\end{enumerate}

\begin{esempio}
Modello con costante

\begin{equation*}
y=X_{1}\widehat{\beta }_{1}+X_{2}\widehat{\beta }_{2}+\varepsilon
\end{equation*}%
dove $K_{1}=1$ e $X_{1}=S$, il vettore somma d'ordine $n$.%
\begin{equation*}
\widetilde{y}=\widetilde{X}_{2}\widehat{\beta }_{2}+\widetilde{\varepsilon }
\end{equation*}%
con 
\begin{eqnarray*}
\widetilde{y} &=&M_{1}y \\
M_{1} &=&I-S(S^{\prime }S)^{-1}S^{\prime }=I-\frac{1}{n}SS^{\prime } \\
M_{1}y &=&y-\frac{1}{n}SS^{\prime }y=y-S\overline{y}=\left[ 
\begin{array}{c}
y_{1} \\ 
y_{2} \\ 
\vdots \\ 
y_{n}%
\end{array}%
\right] -\left[ 
\begin{array}{c}
\overline{y} \\ 
\overline{y} \\ 
\vdots \\ 
\overline{y}%
\end{array}%
\right] =\left[ 
\begin{array}{c}
y_{1}-\overline{y} \\ 
y_{2}-\overline{y} \\ 
\vdots \\ 
y_{n}-\overline{y}%
\end{array}%
\right] .
\end{eqnarray*}%
Si tratta semplicemente del vettore degli scarti alla media. Il modello di
partenza%
\begin{equation*}
y_{i}=\beta _{1}+\beta _{2}x_{2i}+\ldots +\beta _{K}x_{Ki}+\varepsilon _{i}
\end{equation*}%
pu\`{o} essere stimato sul modello trasformato%
\begin{equation*}
y_{i}-\overline{y}=\beta _{2}\left( x_{2i}-\overline{x}_{2}\right) +\beta
_{3}\left( x_{3i}-\overline{x}_{3}\right) +\ldots +\beta _{K}\left( x_{Ki}-%
\overline{x}_{K}\right) +\widetilde{\varepsilon }_{i}\ .
\end{equation*}%
Inoltre%
\begin{eqnarray*}
\underset{(1\times 1)}{\widehat{\beta }_{1}} &=&\left( X_{1}^{\prime
}X_{1}\right) ^{-1}X_{1}^{\prime }\left( y-X_{2}\underset{(K-1)\times 1}{%
\widehat{\beta }_{2}}\right) \\
&=&\frac{1}{n}S^{\prime }(y-X_{2}\widehat{\beta }_{2}) \\
&=&\overline{y}-\overline{x}^{\prime }\widehat{\beta }_{2}
\end{eqnarray*}%
dove $\overline{x}$ rappresenta il vettore $(K-1)\times 1$ delle medie delle
variabili esplicative.
\end{esempio}

\noindent Nota bene: la regressione scomposta \`{e} utilizzata in vari
ambiti dell'econometria e della statistica. Chi seguir\`{a} un corso di
analisi di serie storiche vedr\`{a} che la regressione scomposta \`{e} utile
per spiegare il concetto di \textquotedblright funzione di autocorrelazione
parziale\textquotedblright\ utilizzato per l'identificazione dell'ordine $%
(q,p)$ di un processo autoregressivo a media mobile ($ARMA$-process).

\part{La regressione generalizzata}

\chapter{Il modello generale}

\section{Introduzione}

\subsection{Breve ritorno sulla regressione classica}

Durante il corso \textquotedblleft Introduzione
all'Econometria\textquotedblright\ abbiamo introdotto e studiato le propriet%
\`{a} del modello di regressione lineare. Per un singolo individuo (o
istante temporale) $i$, la formulazione del modello lineare in forma
vettoriale \`{e} data da 
\begin{equation}
\underset{(1\times 1)}{y_{i}}=\underset{(1\times K)}{x_{i}^{\prime }}%
\underset{(K\times 1)}{\beta }+\underset{(1\times 1)}{\epsilon _{i}}
\end{equation}%
mentre per l'insieme degli $n$ individui (periodi) abbiamo 
\begin{equation}
\underset{(n\times 1)}{y}=\underset{(n\times K)}{X}\underset{(K\times 1)}{%
\beta }+\underset{(n\times 1)}{\epsilon }
\end{equation}%
Riprendiamo brevemente le ipotesi della regressione classica

\begin{enumerate}
\item sulle variabili esplicative: non stocastiche

\item sugli errori:

\begin{itemize}
\item di valore atteso nullo

\item omoschedastici

\item non correlati (o indipendenti)
\end{itemize}

\item relazione d'indipendenza tra variabili esplicative ed errori.
\end{enumerate}

\noindent Sotto queste ipotesi, abbiamo dimostrato che i $m.q.$ forniscono
degli stimatori efficienti (BLUE). Tuttavia, in molte applicazioni (sia in
economia che in finanza) le ipotesi classiche non possono essere mantenute.
In tal caso occorre porsi la domanda, se i minimi quadrati siano ancora
efficienti e, qualora non fosse il caso, adottare nuovi metodi di stima.
Questo \`{e} appunto il problema che affronteremo nella prima parte del
semestre, analizzando in particolare due situazioni:

\begin{enumerate}
\item In primo luogo abbandoneremo l'ipotesi di omoschedasticit\`{a} ed
indipendenza degli errori in favore di una struttura di varianza-covarianza
pi\`{u} generale $\Rightarrow $ regressione generalizzata (prima parte).

\item In secondo luogo, ammetteremo che le variabili esplicative siano
stocastiche, cio\`{e} esse stesse delle variabili aleatorie. Sappiamo gi\`{a}
che quando le variabili esplicative sono aleatorie ma indipendenti dagli
errori, i risultati della regressione classica restano validi. Invece,
quando queste sono correlate con gli errori tutto crolla ed \`{e} necessario
utilizzare metodi diversi $\Rightarrow $ regressione stocastica (seconda
parte).
\end{enumerate}

\subsection{Modelli econometrici che non soddisfano le ipotesi classiche}

Quando e perch\'{e} abbandonare l'ipotesi d'omoschedasticit\`{a} e
d'indipendenza degli errori?

\begin{enumerate}
\item con i dati trasversali ($n$ individui allo stesso momento) si
osservano frequentemente dei fenomeni di \emph{eteroschedasticit\`{a}}.

\item Con i dati temporali o pi\`{u} propriamente detto con le serie
storiche si osserva che l'errore a un dato istante (al momento $t$) dipende
dagli errori passati (in $t-1$, $t-2$, $\dots $): si parla di errori
autocorrelati.
\end{enumerate}

\section{Il modello e le ipotesi}

Ci interessiamo al modello di regressione 
\begin{equation}
y=X\beta + \epsilon
\end{equation}
con $K$ variabili esplicative ed $n$ osservazioni, munito delle seguenti
ipotesi:

\begin{enumerate}
\item[H1)] Sulle variabili esplicative

\begin{enumerate}
\item[H1a)] $X$ non stocastico.

\item[H1b)] $rg(X)=K$.
\end{enumerate}

\item[H2)] Sugli errori

\begin{enumerate}
\item[H2a)] $E(\epsilon)=0$.

\item[H2b)] $V(\epsilon )=\sigma ^{2}V$, dove $V$ \`{e} una matrice
qualsiasi definita positiva.
\end{enumerate}

\item[H3)] $X, \ \epsilon$ mutualmente indipendenti.
\end{enumerate}

Rispetto alla regressione classica la sola differenza \`e: 
\begin{equation}
\begin{array}{rcll}
V(\epsilon) & = & \sigma^2I & \text{ regressione classica} \\ 
V(\epsilon) & = & \sigma^2V & \text{ regressione generalizzata}%
\end{array}%
\end{equation}

\begin{esempio}
\label{esempio_mqg_puri} (Eteroschedasticit\`{a} pura)

Studiamo una funzione di risparmio del tipo: 
\begin{equation}
S_{i}=a+bR_{i}+\epsilon _{i},\ \ \ i=1,\dots ,n
\end{equation}%
dove

\begin{enumerate}
\item[-] $S_i$ \`e il risparmio della famiglia $i$.

\item[-] $R_i$ \`e il reddito della famiglia.
\end{enumerate}

\noindent Tipicamente il grafico degli $n$ punti $(S_i,R_i)$ \`e del tipo:

\begin{equation*}
\text{grafico}
\end{equation*}%
\vspace{5cm}

\noindent La varianza dell'errore non \`e costante: \`e una funzione
crescente del reddito. Si pu\`o quindi scrivere: 
\begin{equation*}
V(\epsilon_i)=f(R_i)>0 \text{ e } f^{\prime }(R_i) > 0.
\end{equation*}
Una specificazione naturale potrebbe quindi essere 
\begin{equation*}
V(\epsilon_i)=\sigma^2R_i^2,
\end{equation*}
dove $\sigma^2$ \`e un parametro di proporzionalit\`a sconosciuto. Con un
campione di $n$ famiglie ed ammettendo $Cov(\epsilon_i, \epsilon_j)=0$ per $%
i \neq j$ avremmo 
\begin{equation}
\underset{(n \times n)}{V(\epsilon)}=\sigma^2\underset{V}{\underbrace{ \left[
\begin{array}{cccc}
R_1^2 & 0 & \dots & 0 \\ 
0 & R_2^2 & \ddots & 0 \\ 
\vdots & \ddots & \ddots & 0 \\ 
0 & \dots & 0 & R_n^2%
\end{array}
\right] }}
\end{equation}
Da notare che in questo caso la matrice $V$ \`e nota in quanto le variabili $%
R_i$ sono osservate.
\end{esempio}

\section{Quando $V$ \`{e} conosciuta: i minimi quadrati generalizzati puri}

Quando la matrice $V$ \`e conosciuta (come nel precedente esempio) il metodo
di stima prende il nome di minimi quadrati generalizzati puri, o brevemente $%
m.q.g.$ puri. L'idea fondamentale \`e molto semplice. La presentiamo
prendendo spunto dall'Esempio \ref{esempio_mqg_puri}.

\begin{enumerate}
\item L'idea

Sappiamo risolvere questo problema quando la matrice $V=I$. Se fosse
possibile, tramite una trasformazione appropriata, ricondurci a tale caso,
il nostro problema sarebbe quindi risolto! Sul modello trasformato infatti,
potremmo applicare i $m.q.$ ordinari. Nell'esempio \ref{esempio_mqg_puri},
la varianza per l'individuo $i$ \`{e} uguale a $V(\epsilon _{i})=\sigma
^{2}R_{i}^{2}$ e lo $s.q.m.\ \sqrt{V(\epsilon _{i})}=\sigma R_{i}$.
Dividiamo ogni osservazione per il suo $s.q.m.$ (a meno di $\sigma $): 
\begin{eqnarray*}
S_{i} &=&a+bR_{i}+\epsilon _{i} \\
\underset{\text{(tasso di risparmio)}}{\frac{S_{i}}{R_{i}}} &=&a\frac{1}{%
R_{i}}+b+\frac{\epsilon _{i}}{R_{i}}
\end{eqnarray*}%
Questa \`{e} una regressione semplice dove:

\begin{itemize}
\item $S_{i}/R_{i}$ \`{e} la variabile dipendente, spiegata

\item $1/R_{i}$ \`{e} la variabile esplicativa

\item $b$ \`{e} la costante (che corrisponde alla propensione marginale al
risparmio)

\item $a$ \`{e} il coefficiente della variabile esplicativa
\end{itemize}

e gli errori hanno le propriet\`{a} seguenti:

\begin{itemize}
\item $\tilde{\epsilon}_{i}=\epsilon _{i}/R_{i}$

\item $E(\tilde{\epsilon}_{i})=1/R_{i}E(\epsilon _{i})=$

\item $V(\tilde{\epsilon}_{i})=1/R_{i}^{2}V(\epsilon _{i})=\hspace*{1cm}%
\forall \ i$ costante! $\Rightarrow $ omoschedasticit\`{a}

\item $Cov(\tilde{\epsilon}_{i},\tilde{\epsilon}_{j})=1/(R_{i}R_{j})Cov(%
\epsilon _{i},\epsilon _{j})=\hspace*{1cm}i\neq j$
\end{itemize}

Dunque il modello cos\`{\i} trasformato soddisfa appieno le ipotesi della
regressione classica $\Rightarrow $ applico su di esso lo stimatore dei $%
m.q.o.$!

\item L'esempio formalizzato

Vogliamo ora riscrivere quanto fatto nel punto precedente in forma
matriciale per poi coglierne la struttura. Chiamiamo dunque 
\begin{equation*}
\begin{array}{cccc}
y=\underset{(n\times 1)}{\left[ 
\begin{array}{c}
S_{1} \\ 
\vdots \\ 
S_{n}%
\end{array}%
\right] } & X=\underset{(n\times 2)}{\left[ 
\begin{array}{cc}
1 & R_{1} \\ 
\vdots & \vdots \\ 
1 & R_{n}%
\end{array}%
\right] } & \beta =\underset{(2\times 1)}{\left[ 
\begin{array}{c}
a \\ 
b%
\end{array}%
\right] } & \epsilon =\underset{(n\times 1)}{\left[ 
\begin{array}{c}
\epsilon _{1} \\ 
\vdots \\ 
\epsilon _{n}%
\end{array}%
\right] }%
\end{array}%
\end{equation*}

\begin{equation*}
\Rightarrow y=X\beta +\epsilon
\end{equation*}%
Inoltre sappiamo che $E(\epsilon )=0$ e $V(\epsilon )=\sigma ^{2}\left[ 
\begin{array}{ccc}
R_{1}^{2} &  & 0 \\ 
& \ddots &  \\ 
0 &  & R_{n}^{2}%
\end{array}%
\right] =\sigma ^{2}V$.

Che trasformazione abbiamo utilizzato in precedenza al fine di applicare il
metodo dei $m.q.$ ordinari? Ogni componente del vettore $y$ \`{e} stata
divisa per il reddito corrispondente (o moltiplicata per $\dfrac{1}{R_{i}}$%
). Come possiamo esprimere questa operazione utilizzando il calcolo
matriciale? Basta \textit{premoltiplicare} il vettore $y$ per la matrice
diagonale, notata $P$, avente $\dfrac{1}{R_{i}}$ in posizione $(i,i)$: 
\begin{equation*}
\underset{(n\times 1)}{\tilde{y}}\ ,\ \underset{(n\times n)}{P}=\left[ 
\begin{array}{lll}
\dfrac{1}{R_{1}} &  & 0 \\ 
& \ddots &  \\ 
0 &  & \dfrac{1}{R_{n}}%
\end{array}%
\right] \Rightarrow \tilde{y}=Py.
\end{equation*}%
Lo stesso per la parte destra dell'equazione. Il modello trasformato \`{e}
dunque: 
\begin{equation*}
\underset{(n\times n)}{P}\underset{(n\times 1)}{y}=\underset{(n\times n)}{P}%
\underset{(n\times 2)}{X\beta }+\underset{(n\times n)}{P}\underset{(n\times
1)}{\epsilon }
\end{equation*}%
La $i-$esima riga: 
\begin{equation*}
\dfrac{S_{i}}{R_{i}}=\left( \dfrac{1}{R_{i}},\dfrac{R_{i}}{R_{i}}\right)
\beta +\dfrac{\epsilon _{i}}{R_{i}}
\end{equation*}%
Il modello trasformato con la matrice non-singolare $P$ \`{e}: 
\begin{equation*}
\tilde{y}=\tilde{X}\beta +\tilde{\epsilon}
\end{equation*}%
dove dalle propriet\`{a} di $\epsilon $ ricaviamo che

\begin{itemize}
\item $E(\tilde{\epsilon})=E(P\epsilon )=0$,

\item $V(\tilde{\epsilon})=V(P\epsilon )=PV(\epsilon )P^{\prime }=$ 
\begin{eqnarray*}
&=&\sigma ^{2}\left[ 
\begin{array}{lll}
\dfrac{1}{R_{1}} &  & 0 \\ 
& \ddots &  \\ 
0 &  & \dfrac{1}{R_{n}}%
\end{array}%
\right] \left[ 
\begin{array}{lll}
R_{1}^{2} &  & 0 \\ 
& \ddots &  \\ 
0 &  & R_{n}^{2}%
\end{array}%
\right] \left[ 
\begin{array}{lll}
\dfrac{1}{R_{1}} &  & 0 \\ 
& \ddots &  \\ 
0 &  & \dfrac{1}{R_{n}}%
\end{array}%
\right] \\
&=&\sigma ^{2}I_{n},
\end{eqnarray*}
\end{itemize}

che sono le propriet\`{a} della regressione classica.
\end{enumerate}

\subsection{Gli stimatori e le loro proriet\`{a}}

\begin{enumerate}
\item Gli stimatori dei $m.q.g.$ come stimatori dei $m.q.o.$ sul modello
trasformato

Il modello iniziale \`{e} il seguente 
\begin{equation*}
\underset{(n\times 1)}{y}=\underset{(n\times K)}{X}\underset{(K\times 1)}{%
\beta }+\underset{(n\times 1)}{\epsilon },
\end{equation*}%
con le ipotesi $H1-H3$. In particolare abbiamo

\begin{itemize}
\item $E(\epsilon )=0$,

\item $V(\epsilon )=E(\epsilon \epsilon^{\prime })=\underset{\text{def. pos.}%
}{\sigma ^{2}V}$.
\end{itemize}

Sappiamo che per ogni matrice def. positiva $V$ esiste una matrice
non-singolare $P$ tale che 
\begin{equation*}
PVP^{\prime }=I.
\end{equation*}%
Trasformiamo dunque il modello \textit{premoltiplicando} per $P$ 
\begin{eqnarray}
\underset{\tilde{y}}{\underbrace{Py}} &=&\underset{\tilde{X}}{\underbrace{PX}%
}\ \beta +\underset{\tilde{\epsilon}}{\underbrace{P\epsilon }}  \notag \\
\tilde{y} &=&\tilde{X}\beta +\tilde{\epsilon}  \label{modello trasformato}
\end{eqnarray}
Il modello trasformato (\ref{modello trasformato}) ha un vettore d'errori $%
\tilde{\epsilon}$ tale per cui

\begin{itemize}
\item $E(\tilde{\epsilon})=0$

\item $V(\tilde{\epsilon})=\sigma ^{2}I$.
\end{itemize}

Questo modello trasformato soddisfa tutte le ipotesi della regressione
classica. Applichiamo dunque su di esso i $m.q.o.$! Sul modello trasformato
i risultati della stima saranno pertanto quelli della regressione classica e
cio\`{e}:

\begin{itemize}
\item lo stimatore di beta, chiamiamolo $\beta ^{\ast }$: 
\begin{equation*}
\beta ^{\ast }=(\tilde{X}^{\prime }\tilde{X})^{-1}\tilde{X}^{\prime }\tilde{y%
}
\end{equation*}

\item la sua vera varianza 
\begin{equation*}
V(\beta ^{\ast })=\sigma ^{2}(\tilde{X}^{\prime }\tilde{X})^{-1}
\end{equation*}

\item il vettore d'errori stimati sul modello trasformato: 
\begin{equation*}
\hat{\tilde{\epsilon}}=\tilde{y}-\tilde{X}\beta ^{\ast }
\end{equation*}

\item la somma dei quadrati degli errori (del modello trasformato!) 
\begin{equation*}
\tilde{SS}={\hat{\tilde{\epsilon}}}^{\prime }\hat{\tilde{\epsilon}}=\tilde{y}%
^{\prime }\tilde{y}-{\beta ^{\ast \prime }}\tilde{X}^{\prime }\tilde{y}=%
\tilde{y}^{\prime }\tilde{M}\tilde{y}
\end{equation*}%
dove 
\begin{equation*}
\tilde{M}=I-\tilde{X}(\tilde{X}^{\prime }\tilde{X})^{-1}\tilde{X}^{\prime }.
\end{equation*}

\item la stima corretta di $\sigma ^{2}$ 
\begin{equation*}
\hat{\sigma}^{2}=\dfrac{\tilde{SS}}{n-K}
\end{equation*}

\item la stima della matrice di varianza-covarianza di $\beta ^{\ast }$ 
\begin{equation*}
\hat{V}(\beta ^{\ast })=\hat{\sigma}^{2}(\tilde{X}^{\prime }\tilde{X})^{-1}
\end{equation*}
\end{itemize}

\item Lo stimatore dei $m.q.g.$

Lo stimatore $\beta ^{\ast }$ pu\`{o} essere riscritto in termini di $X,\ y$
e della matrice $P$ nel seguente modo: 
\begin{eqnarray*}
\beta ^{\ast } &=&(\tilde{X}^{\prime }\tilde{X})^{-1}\tilde{X}^{\prime }%
\tilde{y} \\
&=&(X^{\prime }P^{\prime }PX)^{-1}X^{\prime }P^{\prime }Py
\end{eqnarray*}%
Ora, sapendo che $PVP^{\prime }=I$ otteniamo successivamente 
\begin{eqnarray*}
VP^{\prime } &=&P^{-1} \\
P^{\prime } &=&V^{-1}P^{-1} \\
P^{\prime }P &=&V^{-1}
\end{eqnarray*}%
Dunque per lo stimatore $\beta ^{\ast }$ otteniamo, in termini delle
variabili di partenza:

\begin{itemize}
\item lo stimatore 
\begin{equation}
\beta ^{\ast }=(X^{\prime }V^{-1}X)^{-1}X^{\prime }V^{-1}y
\label{formula mqg}
\end{equation}

\item La vera varianza di $\beta ^{\ast }$%
\begin{equation*}
V(\beta ^{\ast })=\sigma ^{2}(X^{\prime }V^{-1}X)^{-1}
\end{equation*}

\item Il vettore d'errori stimati sul modello di partenza 
\begin{equation*}
\hat{\epsilon}=y-X\beta ^{\ast }
\end{equation*}

La relazione con gli errori stimati sul modello trasformato \`{e} la
seguente: 
\begin{equation*}
\hat{\tilde{\epsilon}}=\tilde{y}-\tilde{X}\beta ^{\ast }=Py-PX\beta ^{\ast
}=P(y-X\beta ^{\ast })=P\hat{\epsilon}
\end{equation*}

\item sul modello trasformato avevamo: 
\begin{equation*}
\tilde{SS}={\hat{\tilde{\epsilon}}}^{\prime }\hat{\tilde{\epsilon}}=\hat{%
\epsilon}^{\prime }P^{\prime }P\hat{\epsilon}=\hat{\epsilon}^{\prime }V^{-1}%
\hat{\epsilon}
\end{equation*}
\end{itemize}

Lo stimatore $\beta ^{\ast }$ definito dalla formula (\ref{formula mqg}) 
\`{e} chiamato stimatore dei minimi quadrati generalizzati ($m.q.g.$). Esso 
\`{e} cos\`{\i} chiamato perch\'{e} ottenuto minimizzando la somma dei
quadrati generalizzati 
\begin{equation}
\underset{\beta }{min}\ \underset{:=SSG}{\underbrace{(y-X\beta )^{\prime
}V^{-1}(y-X\beta )}}
\end{equation}%
la cui soluzione \`{e} appunto il nostro $\beta ^{\ast }$. Infatti%
\begin{eqnarray*}
SSG &=&(y-X\beta )^{\prime }V^{-1}(y-X\beta ) \\
&=&y^{\prime }V^{-1}y-y^{\prime }V^{-1}X\beta -\beta ^{\prime }X^{\prime
}V^{-1}y+\beta ^{\prime }X^{\prime }V^{-1}X\beta \\
&=&y^{\prime }V^{-1}y-\underset{\text{forma lineare}}{\underbrace{2y^{\prime
}V^{-1}X\beta }}+\underset{\text{forma quadratica}}{\underbrace{\beta
^{\prime }X^{\prime }V^{-1}X\beta }} \\
\frac{\partial SSG}{\partial \beta } &=&-2X^{\prime }V^{-1}y+2X^{\prime
}V^{-1}X\beta \overset{!}{=}0 \\
&&\text{da cui risolvendo rispetto a }\beta \\
\beta ^{\ast } &=&(X^{\prime }V^{-1}X)^{-1}X^{\prime }V^{-1}y
\end{eqnarray*}%
Noi lo abbiamo ricavato utilizzando un'argomentazione diversa. \`{E} utile
esprimere la quantit\`{a} $\tilde{SS}$ in termini delle variabili di
partenza $y$, $X$: 
\begin{eqnarray*}
\tilde{SS} &=&SSG=y^{\prime }V^{-1}y-{\beta ^{\ast \prime }}(X^{\prime
}V^{-1}y) \\
&& \\
\tilde{M} &=&I-\tilde{X}(\tilde{X}^{\prime }\tilde{X})^{-1}\tilde{X}^{\prime
} \\
&=&I-PX(X^{\prime }V^{-1}X)^{-1}X^{\prime }P^{\prime } \\
&& \\
\tilde{SS} &=&\tilde{y}^{\prime }\tilde{M}\tilde{y}=y^{\prime }P^{\prime }%
\tilde{M}Py \\
&=&y^{\prime }P^{\prime }Py-y^{\prime }P^{\prime }PX(X^{\prime
}V^{-1}X)^{-1}X^{\prime }P^{\prime }Py \\
&=&y^{\prime }V^{-1}y-y^{\prime }V^{-1}X(X^{\prime }V^{-1}X)^{-1}X^{\prime
}V^{-1}y
\end{eqnarray*}

\begin{table}[h]
\caption{Specchietto riassuntivo}
\label{tab:Spe}\vspace{0.5cm} 
\begin{tabular}{|c|c|}
\hline
&  \\ 
$m.q.o.$ su $\tilde{y}=\tilde{X}\beta +\tilde{\epsilon}$ & $m.q.g.$ su $%
y=X\beta +\varepsilon$ \\ \hline
&  \\ 
$%
\begin{array}{rcl}
\beta ^{\ast } & = & (\tilde{X}^{\prime }\tilde{X})^{-1}\tilde{X}^{\prime }%
\tilde{y} \\ 
&  &  \\ 
V(\beta ^{\ast }) & = & \sigma ^{2}(\tilde{X}^{\prime }\tilde{X})^{-1} \\ 
&  &  \\ 
\tilde{SS} & = & {\hat{\tilde{\epsilon}}}^{\prime }\hat{\tilde{\epsilon}}=%
\tilde{y}^{\prime }\widetilde{M}\tilde{y} \\ 
& = & \tilde{y}^{\prime }(I-\tilde{X}(\tilde{X}^{\prime }\tilde{X})^{-1}%
\tilde{X}^{\prime })\tilde{y} \\ 
&  &  \\ 
\hat{\sigma}^{2} & = & \tilde{SS}/(n-K) \\ 
&  &  \\ 
\widehat{V}(\beta ^{\ast }) & = & \widehat{\sigma }^{2}(\tilde{X}^{\prime }%
\tilde{X})^{-1}%
\end{array}%
$ & $%
\begin{array}{rcl}
\beta ^{\ast } & = & (X^{\prime }V^{-1}X)^{-1}X^{\prime }V^{-1}y \\ 
&  &  \\ 
V(\beta ^{\ast }) & = & \sigma ^{2}(X^{\prime }V^{-1}X)^{-1} \\ 
&  &  \\ 
SSG & = & \hat{\epsilon}^{\prime }V^{-1}\hat{\epsilon} \\ 
= & \multicolumn{2}{l}{y^{\prime }(V^{-1}-V^{-1}X(X^{\prime
}V^{-1}X)^{-1}X^{\prime }V^{-1})y} \\ 
&  &  \\ 
\hat{\sigma}^{2} & = & SSG/(n-K) \\ 
&  &  \\ 
\widehat{V}(\beta ^{\ast }) & = & \widehat{\sigma }^{2}(X^{\prime
}V^{-1}X)^{-1}%
\end{array}%
$ \\ \hline
\end{tabular}%
\end{table}

\item Propriet\`{a} di $\beta ^{\ast }$

\begin{itemize}
\item $\beta ^{\ast }$ \`{e} corretto! Infatti: 
\begin{equation*}
\beta ^{\ast }=\underset{A}{\underbrace{(X^{\prime }V^{-1}X)^{-1}X^{\prime
}V^{-1}}}\ y=Ay
\end{equation*}%
e poich\'{e} $E(y)=X\beta $, $V(y)=\sigma ^{2}V$, otteniamo $E(\beta ^{\ast
})=AE(y)=\underset{I}{\underbrace{AX}}\ \beta =\beta .$\newline
$V(\beta ^{\ast })=AV(y)A^{\prime }=\sigma ^{2}AVA^{\prime }=\sigma
^{2}(X^{\prime }V^{-1}X)^{-1}$.

\item \`{E} efficiente (BLUE)!

Dimostrazione: Condideriamo un qualsiasi altro stimatore lineare $\beta
^{\#}=A^{\#}y$ e mostriamo che per $\beta ^{\#}$ corretto 
\begin{equation*}
V(\beta ^{\#})>V(\beta ^{\ast })
\end{equation*}%
Ora: 
\begin{eqnarray*}
E(\beta ^{\#}) &=&A^{\#}E(y)=A^{\#}X\beta \\
V(\beta ^{\#}) &=&\sigma ^{2}A^{\#}V{A^{\#\prime }}
\end{eqnarray*}%
La condizione di correttezza di $\beta ^{\#}$ impone che $A^{\#}X=I$.
Poniamo inoltre $A^{\#}=A+C$. Otteniamo: 
\begin{eqnarray*}
V(\beta ^{\#}) &=&\sigma ^{2}(A+C)V(A^{\prime }+C^{\prime }) \\
&=&\sigma ^{2}\underset{V(\beta ^{\ast })}{\underbrace{AVA^{\prime }}}+%
\underset{\text{nulli quando }\beta ^{\#}\text{ \`{e} corretto}}{\underbrace{%
\sigma ^{2}AVC^{\prime }+\sigma ^{2}CVA^{\prime }}}+\underset{\text{def. non
neg.}}{\underbrace{\sigma ^{2}CVC^{\prime }}} \\
&&\text{poich\'{e}} \\
A^{\#}X &=&I\Rightarrow \underset{I}{\underbrace{AX}}+CX=I\Rightarrow CX=0 \\
AVC^{\prime } &=&(X^{\prime }V^{-1}X)^{-1}X^{\prime }V^{-1}VC^{\prime
}=(X^{\prime }V^{-1}X)^{-1}\underset{=0}{\underbrace{X^{\prime }C^{\prime }}}%
=0
\end{eqnarray*}
\end{itemize}

\item Stima corretta di $\sigma ^{2}$\newline
Definiamo $\hat{\sigma}^{2}$ come lo stimatore di $\sigma ^{2}$ del modello
trasformato, quindi 
\begin{equation*}
\hat{\sigma}^{2}=\dfrac{SSG}{n-K}\text{ dove }SSG=\tilde{SS}
\end{equation*}%
\begin{eqnarray*}
SSG &=&y^{\prime }\underset{Q}{\underbrace{\left[ V^{-1}-V^{-1}X(X^{\prime
}V^{-1}X)^{-1}X^{\prime }V^{-1}\right] }}y \\
&=&y^{\prime }Qy \\
E(SSG) &=&?
\end{eqnarray*}%
Sappiamo che $QX=0$, da cui segue $Qy=Q(X\beta +\epsilon )=Q\epsilon $.%
\newline
Quindi 
\begin{eqnarray*}
SSG &=&y^{\prime }Qy=\epsilon ^{\prime }Q\epsilon \text{ \`{e} una }FQ\text{
in }\epsilon \sim N(0,\sigma ^{2}V) \\
&& \\
E(SSG) &=&E(\epsilon ^{\prime }Q\epsilon )=trQV(\epsilon )=\sigma ^{2}trQV \\
&=&\sigma ^{2}tr(I_{n}-V^{-1}X(X^{\prime }V^{-1}X)^{-1}X^{\prime }) \\
&=&\sigma ^{2}tr(I_{n})-\sigma ^{2}tr(I_{K})=\sigma ^{2}(n-K)
\end{eqnarray*}%
Definendo $\hat{\sigma}^{2}=\dfrac{SSG}{n-K}$ otteniamo uno stimatore
corretto di $\sigma ^{2}$:%
\begin{equation*}
E(\hat{\sigma}^{2})=\sigma ^{2}.
\end{equation*}%
Di conseguenza:%
\begin{equation*}
\hat{V}(\beta ^{\ast })=\hat{\sigma}^{2}(\tilde{X}^{\prime }\tilde{X})^{-1}=%
\hat{\sigma}^{2}(X^{\prime }V^{-1}X)^{-1}
\end{equation*}%
\`{e} uno stimatore corretto della $V(\beta ^{\ast })$.
\end{enumerate}

\subsection{L'induzione statistica}

Sotto l'ipotesi di normalit\`{a} degli errori, tutte le procedure di test
sul modello trasformato sono rigorosamente identiche a quelle della
regressione classica. Le statistiche $t$ per un'ipotesi semplice o $F$ per
un'ipotesi congiunta date in termini delle variabili trasformate si possono
sempre esprimere in termini delle variabili di partenza. In particolare,
dato $\epsilon \sim N(0,\sigma ^{2}V)$ segue:

\begin{itemize}
\item $\beta ^{\ast }\sim N(\beta ,\sigma ^{2}(X^{\prime }V^{-1}X)^{-1})$.

\item $\dfrac{SSG}{\sigma ^{2}}\sim \chi _{n-K}^{2}$.

\item $\beta ^{\ast }$ e $SSG$ sono indipendenti.
\end{itemize}

\noindent Se chiamassimo $\tilde{x}^{ij}$ l'elemento in posizione $(i,j)$
della matrice inversa $(X^{\prime }V^{-1}X)^{-1}$ allora 
\begin{equation}
\dfrac{\beta _{k}^{\ast }-\beta _{k}}{\sqrt{\hat{\sigma}^{2}\tilde{x}^{kk}}}%
\sim t_{n-K}
\end{equation}

\subsection{La previsione}

Il problema:

\begin{itemize}
\item \`{E} dato un campione $y=X\beta +\epsilon $ di $n$ osservazioni.

\item Per un individuo $s$ al di fuori del campione vogliamo prevedere $%
y_{s} $ (il suo comportamento) date le sue caratteristiche contenute in $%
x_{s}^{\prime }$.
\end{itemize}

\noindent Il miglior previsore lineare di $y_{s}$ \`{e} tra tutti i
previsori lineari e corretti\footnote{%
Ricordiamo che un previsore $y_{s}^{p}$ di $y_{s}$ \`{e} corretto se $%
E(y_{s}^{p})=E(y_{s})$ o equivalentemente se l'errore di previsione $%
\varepsilon _{s}^{p}=y_{s}-y_{s}^{p}$ \`{e} nullo in valore atteso.} quello
la cui varianza dell'errore di previsione \`{e} minima. Si ottiene:

\begin{enumerate}
\item Nella regressione classica $\beta $ \`{e} stimato per mezzo dei $%
m.q.o. $ e%
\begin{equation*}
y_{s}^{p}=x_{s}^{\prime }\hat{\beta}
\end{equation*}

\item Nella regressione generalizzata, $\beta $ \`{e} stimato per mezzo dei $%
m.q.g.$ Quando l'errore dell'individuo $s$ non \`{e} correlato con gli
errori del campione,%
\begin{equation*}
y_{s}^{p}=x_{s}^{\prime }\beta ^{\ast }
\end{equation*}

\item Salvo nel caso di eteroschedasticit\`{a} pura, nella regressione
generalizzata l'errore dell'individuo $s$ sar\`{a} correlato con gli errori
del campione. Chiamiamo $\underset{(1\times n)}{\sigma ^{2}R^{\prime }}%
=Cov(\epsilon _{s},\epsilon )$. Il miglior previsore lineare \`{e} dato da%
\begin{equation*}
y_{s}^{p}=x_{s}^{\prime }\beta ^{\ast }+Cov(\epsilon _{s},\epsilon
)V(\epsilon )^{-1}\hat{\epsilon}
\end{equation*}
\end{enumerate}

\section{Le conseguenze dell'applicazione dei $m.q.o.$}

Vogliamo analizzare le conseguenze dell'applicazione dei minimi quadrati
ordinari ad un modello con errori correlati. Poich\'{e} per ipotesi $rg(X)=K$
i $m.q.o.$ sono calcolabili:

\begin{equation*}
\hat{\beta}=(X^{\prime }X)^{-1}X^{\prime }y.
\end{equation*}

Che propriet\`{a} hanno?

\subsection{Correttezza e perdita d'efficienza}

\begin{itemize}
\item $\hat{\beta}$ \`{e} corretto.

\item $\hat{\beta}$ non \`{e} efficiente: $V(\hat{\beta})\geq V(\hat{\beta}%
^{\ast })$
\end{itemize}

\subsection{Stima convergente della varianza dello stimatore dei $m.q.o.$}

La vera matrice delle varianze covarianze di $\hat{\beta}$ non \`{e} quella
data dalla regressione classica $\sigma ^{2}(X^{\prime }X)^{-1}$ bens\`{\i} 
\begin{equation}
V(\hat{\beta})=V((X^{\prime }X)^{-1}X^{\prime }y)=\sigma ^{2}(X^{\prime
}X)^{-1}X^{\prime }VX(X^{\prime }X)^{-1}
\end{equation}%
Stimando la varianza di $\hat{\beta}$ con $V(\hat{\beta})=\hat{\sigma}%
^{2}(X^{\prime }X)^{-1}$ non si ottiene una stima corretta per 2 motivi:

\begin{enumerate}
\item $\hat{\sigma}^{2}=\dfrac{SS}{n-K}$ non \`{e} uno stimatore corretto di 
$\sigma ^{2}$.

\item $(X^{\prime }X)^{-1}$ dovrebbe essere rimpiazzato con $(X^{\prime
}X)^{-1}X^{\prime }VX(X^{\prime }X)^{-1}$.
\end{enumerate}

\noindent $\Rightarrow $ Conseguenza: Utilizzare $\hat{\sigma}^{2}(X^{\prime
}X)^{-1}$ quale stimatore della varianza di $\hat{\beta}$ quando $V(\epsilon
)\neq \sigma ^{2}I$ \`{e} sbagliato: i test sono falsi!

\begin{remark}
Nel caso della \emph{regressione eteroschedastica}, cio\`{e} nella
situazione in cui 
\begin{equation*}
V(\epsilon _{i})=\sigma _{i}^{2}\text{ e }Cov(\epsilon _{i},\epsilon
_{j})=0,\ i\neq j
\end{equation*}%
da cui 
\begin{equation*}
V(\epsilon )=\left[ 
\begin{array}{ccc}
\sigma _{1}^{2} &  & 0 \\ 
& \ddots &  \\ 
0 &  & \sigma _{n}^{2}%
\end{array}%
\right] =V\ \text{matrice diagonale,}
\end{equation*}%
\`{e} tuttavia possibile ottenere dei test approssimativamente esatti
utilizzando quale stima della matrice delle varianze covarianze 
\begin{eqnarray*}
\hat{V}(\hat{\beta}) &=&(X^{\prime }X)^{-1}X^{\prime }\hat{V}X(X^{\prime
}X)^{-1}, \\
\text{ con }\hat{V} &=&\left[ 
\begin{array}{ccc}
\hat{\epsilon}_{1}^{2} &  & 0 \\ 
& \ddots &  \\ 
0 &  & \hat{\epsilon}_{n}^{2}%
\end{array}%
\right] .
\end{eqnarray*}%
Quest'ultima formula permette (nel caso della regressione eteroschedastica)
di eseguire test asintoticamente corretti.
\end{remark}

\subsection{Equivalenza fra $m.q.o.$ e $m.q.g.$}

Esistono casi in cui $\hat{\beta}=\beta ^{\ast }$ (per qualsiasi vettore $y$)

\begin{itemize}
\item Caso triviale $V=I$.

\item Caso generale: Si pu\`{o} dimostrare che $\hat{\beta}=\beta ^{\ast }$
se e solo se esiste una matrice non singolare $A$ tale che 
\begin{equation}
\underset{(n\times n)}{V}\underset{(n\times K)}{X}=\underset{(n\times K)}{X}%
\underset{(K\times K)}{A}
\end{equation}
\end{itemize}

\section{Quando $V$ dipende da parametri sconosciuti}

Quando $V$ \`e completamente arbitraria, il problema non ha soluzioni.
Infatti $V$, matrice d'ordine $n$, possiede $n(n+1)/2$ elementi.
L'economista \`e interessato a strutture di varianze covarianze che
dipendono da un numero fisso di parametri sconosciuti. Chiamiamo con $m$ il
numero di parametri sconosciuti contenuti nel vettore $\underset{(m \times 1)%
}{\theta}$. Ipotizziamo quindi che conoscendo $\theta$ si possa costruire $V$
per ogni $n$. Designamo quindi la matrice delle varianze covarianze di $%
\epsilon$ con $V(\epsilon)=\sigma^2 V(\theta)$.

\begin{esempio}
Struttura di equicorrelazione

\begin{itemize}
\item $E(\epsilon_i) = 0 \ \forall i$.

\item $V(\epsilon_i) = \sigma^2 \ \forall i$.

\item $Cov(\epsilon_i,\epsilon_j)=\theta \sigma^2, \ i \ne j, \ 0 \le \theta
< 1$

\item Con un campione di $n$ osservazioni, abbiamo dunque 
\begin{equation}
\underset{(n\times n)}{V(\epsilon )}=\sigma ^{2}\underset{V(\theta )}{%
\underbrace{\left[ 
\begin{array}{cccc}
1 & \theta & \dots & \theta \\ 
\theta & 1 & \ddots & \vdots \\ 
\vdots & \ddots & \ddots & \theta \\ 
\theta & \dots & \theta & 1%
\end{array}%
\right] }}
\end{equation}
\end{itemize}
\end{esempio}

Strategia per la stima:

\begin{enumerate}
\item Prima possibilit\`{a}

Siccome sappiamo risolvere il problema di stima quando $\theta $ \`{e}
conosciuto, l'idea naturale \`{e} quella di stimare $\theta $ in una prima
tappa per poi applicare lo stimatore dei $m.q.g.$. Questo procedimento \`{e}
chiamato \textit{minimi quadrati generalizzati a due tappe} (oppure $m.q.g.$
realizzabili).

\item Seconda possibiilt\`{a}

Quale alternativa ai $m.q.g.$ a due tappe si potrebbe immaginare di stimare 
\textit{simultaneamente} tutti i parametri sconosciuti del modello che sono 
\begin{equation*}
\underset{(K\times 1)}{\beta },\ \underset{(1\times 1)}{\sigma ^{2}},\ 
\underset{(m\times 1)}{\theta }
\end{equation*}%
utilizzando il metodo di massima verosimiglianza ($MLE$ = Maximum Likelihood
Estimation).
\end{enumerate}

\subsection{I $m.q.g.$ a due tappe}

\emph{Prima tappa}: Si cerca uno stimatore convergente di $\theta $. Non c'%
\`{e} una ricetta universale: in numerosi casi la stima del modello con i
minimi quadrati ordinari permette di stimare $\theta $ come funzione degli
errori stimati. \newline
\newline
\noindent \emph{Seconda tappa}: Si utilizzano i minimi quadrati
generalizzati con $V(\hat{\theta})$ al posto di $V(\theta )$. \newline
\newline
\noindent Propriet\`{a}: Asintoticamente (con $n\rightarrow \infty $) lo
stimatore a due tappe possiede la medesima efficienza dello stimatore dei $%
m.q.g.$ puri (cio\'{e} come se conoscessimo il vero $\theta $).

\subsection{Massimo di Verosimiglianza}

Assumendo la normalit\`{a} degli errori $\epsilon \sim N(0,\sigma
^{2}V(\theta ))$ otteniamo grazie alla propriet\`{a} della legge normale che 
$y\sim N(X\beta ,\sigma ^{2}V(\theta ))$. La verosimiglianza, per un
campione di $n$ osservazioni dipende dai parametri sconosciuti $\beta
,\sigma ^{2},\theta $ e si scrive 
\begin{equation*}
L(y\mid \beta ,\sigma ^{2},\theta )=(2\pi \sigma ^{2})^{-n/2}det\left[
V(\theta )\right] ^{-1/2}\exp {(-\frac{1}{2}(y-X\beta )}^{\prime }{\frac{1}{%
\sigma ^{2}}V(\theta )^{-1}(y-X\beta ))}
\end{equation*}%
Gli stimatori di massima verosimiglianza sono soluzioni del programma 
\begin{equation*}
\underset{\{\beta ,\sigma ^{2},\theta \}}{\max }L(y\mid \beta ,\sigma
^{2},\theta )
\end{equation*}%
dal quale si ricavano le condizioni di primo ordine 
\begin{equation*}
\left\{ 
\begin{array}{ccc}
\dfrac{\partial L}{\partial \sigma ^{2}} & = & 0 \\ 
\dfrac{\partial L}{\partial \beta } & = & 0 \\ 
\dfrac{\partial L}{\partial \theta } & = & 0%
\end{array}%
\right.
\end{equation*}%
Questo sistema non possiede una soluzione analitica e quindi \`{e}
necessario utilizzare un algoritmo numerico di risoluzione. Un'importante
propriet\`{a} dello stimatore di massima verosimiglianza consiste nella sua
efficienza asintotica.

\chapter{L'autocorrelazione\label{capitolo autocorrelazione}}

Questo capitolo \`{e} presentato nell'ambito delle serie storiche (o
cronologiche). L'autocorrelazione \`{e} un fenomeno proprio delle serie
storiche. Essa esprime l'idea che l'errore a un dato istante dipende in
parte dagli errori passati. Ci sono svariati modelli per descrivere errori
autocorrelati. In questo capitolo tratteremo un modello molto diffuso in
econometria, il modello autoregressivo di primo ordine: l'errore al tempo $t$
dipende dall'errore al tempo $t-1$.

\section{Il modello markoviano di primo ordine}

Il modello markoviano di primo ordine \`{e} anche detto modello \emph{%
autoregressivo} di primo ordine, abbreviato $AR(1)$. Vediamo ora le ipotesi
di questo modello.

\subsection{Ipotesi e propriet\`{a}}

La regressione abituale 
\begin{equation*}
\left\{ 
\begin{array}{c}
y_{t}=x_{t}^{\prime }\beta +\epsilon _{t}\ \ \ \ t=1,\dots ,n \\ 
\epsilon _{t}=\underset{\text{parte regressiva}}{\underbrace{\rho \epsilon
_{t-1}}}+\underset{\text{parte residua}}{\underbrace{u_{t}}} \\ 
| \rho | <1\text{ e }u_{t}\sim iid(0,\sigma ^{2})%
\end{array}%
\right.
\end{equation*}%
dove $u_{t}$ rappresenta un residuo aleatorio anch'esso non osservabile, $%
iid(0,\sigma ^{2})$. Vale dunque%
\begin{eqnarray*}
E(u_{t}) &=&0\ \forall t \\
E(u_{t}^{2}) &=&\sigma ^{2}\ \forall t \\
Cov(u_{t},u_{s}) &=&0\ \forall t\neq s
\end{eqnarray*}

\noindent \emph{Interpretazione ed osservazioni}

\begin{enumerate}
\item Il termine "modello markoviano" non \`{e} riferito all'equazione%
\begin{equation*}
y_{t}=x_{t}^{\prime }\beta +\epsilon _{t}
\end{equation*}%
ma bens\`{\i} alla struttura autoregressiva del suo termine d'errore 
\begin{equation*}
\epsilon _{t}=\rho \epsilon _{t-1}+u_{t}\text{.}
\end{equation*}%
Nella seconda parte del corso, quando tratteremo i modelli dinamici,
studieremo le propriet\`{a} dello stimatore dei minimi quadrati sul modello
a variabile endogena ritardata 
\begin{equation*}
y_{t}=\alpha _{1}y_{t-1}+\varepsilon _{t}
\end{equation*}%
con $\varepsilon _{t}\sim iid(0,\sigma ^{2})$. Come potete vedere, questo
non \`{e} altro che il modello markoviano di ordine 1 appena definito
applicato alla variabile endogena $y$.

\item Il processo $\epsilon _{t}=\rho \epsilon _{t-1}+u_{t}$ \`{e} senza
memoria, nel senso che tutta l'informazione disponibile in $t-1$ al fine di
spiegare $\epsilon _{t}$ \`{e} riassunta in $\epsilon _{t-1}$. In altre
parole, tutta la storia antecendente a $\epsilon _{t-1}$ non serve a nulla
per spiegare $\epsilon _{t}$: quello che capita nel periodo $t$ dipende
unicamente\footnote{%
Salvo il termine $u_{t}$ che per altro \`{e} indipendente sia da $\epsilon
_{t-1}$ che da $\epsilon _{t-2}$, $\epsilon _{t-2}$, \ldots\ .} dallo stato
attuale di $\epsilon _{t-1}$ non dalla storia antecedente $\epsilon _{t-2}$, 
$\epsilon _{t-3}$, \ldots .

\begin{equation*}
\text{grafico}
\end{equation*}%
\vspace{5cm}

\item Per $\rho $ positivo (caso tipico in economia) il processo $\left\{
\epsilon _{t}\right\} _{t\in \mathbb{Z}}$ presenta una persistenza di valori
dello stesso segno: $P(\epsilon _{t}>0\mid \epsilon _{t-1}>0)>0.5$.

\item Sostituzione per ricorrenza

Sostituendo recursivamente otteniamo la rappresenzatione di $\epsilon _{t}$
in funzione degli errori $u_{t}$: 
\begin{eqnarray}
\epsilon _{t} &=&u_{t}+\underset{\underset{\epsilon _{t-1}=u_{t-1}+\rho
\epsilon _{t-2}}{\Uparrow }}{\rho \epsilon _{t-1}}  \notag \\
&=&u_{t}+\rho u_{t-1}+\underset{\underset{\epsilon _{t-2}=u_{t-2}+\rho
\epsilon _{t-3}}{\Uparrow }}{\rho ^{2}\epsilon _{t-2}}  \notag \\
&\vdots &  \notag \\
&=&\sum_{j=0}^{\infty }\rho ^{j}u_{t-j}\ .  \label{rappresentazione ma}
\end{eqnarray}%
$\epsilon _{t}$ \`{e} una somma ponderata di variabili non correlate di
media 0 e varianza $\sigma ^{2}$. Dalla rappresentazione (\ref%
{rappresentazione ma}) ricaviamo che

\begin{itemize}
\item Propriet\`{a} 1: $E(\epsilon _{t})=0$

\item Propriet\`{a} 2: $V(\epsilon _{t})=\sigma ^{2}\sum_{j=0}^{\infty
}(\rho ^{2})^{j}=\dfrac{1}{1-\rho ^{2}}\sigma ^{2}$.

\item Propriet\`{a} 3: La covarianza fra $\epsilon _{t}$ e $\epsilon _{t-1}$ 
\begin{eqnarray*}
Cov(\epsilon _{t},\epsilon _{t-1}) &=&E(\epsilon _{t}\epsilon _{t-1}) \\
&=&E([u_{t}+\rho u_{t-1}+\rho u_{t-2}+\dots ][u_{t-1}+\rho u_{t-2}+\dots ])
\\
&=&\rho E(u_{t-1}^{2})+\rho ^{3}E(u_{t-2}^{2})+\rho ^{5}E(u_{t-3}^{2})+\dots
\\
&=&\sigma ^{2}\rho \ (1+\rho ^{2}+\rho ^{4}+\dots ) \\
&=&\dfrac{\rho }{1-\rho ^{2}}\sigma ^{2}.
\end{eqnarray*}
\end{itemize}

Facilmente si ottiene che $Cov(\epsilon _{t},\epsilon _{t-k})=\dfrac{\rho
^{k}}{1-\rho ^{2}}\sigma ^{2}$. Notiamo che la covarianza non dipende da $t$
ma solo dalla distanza $k$ tra i due errori.
\end{enumerate}

\begin{remark}
Un processo che soddisfa le propriet\`{a} 1-3 \`{e} chiamato debolmente
stazionario.
\end{remark}

\begin{esempio}
Per un campione di $n$ osservazioni costruiamo la matrice delle varianze
covarianze di $\epsilon $. Per $n=3$ 
\begin{equation*}
\underset{(3\times 3)}{V(\epsilon )}=\sigma ^{2}\dfrac{1}{1-\rho ^{2}}\left[ 
\begin{array}{ccc}
1 & \rho & \rho ^{2} \\ 
\rho & 1 & \rho \\ 
\rho ^{2} & \rho & 1%
\end{array}%
\right]
\end{equation*}%
In generale 
\begin{equation*}
V(\epsilon )=\sigma ^{2}\underset{\text{\`{e} la matrice }V(\theta )\text{
che qui \`{e} }V(\rho )\text{ e che dipende da un solo parametro }\rho .}{%
\underbrace{\dfrac{1}{1-\rho ^{2}}\left[ 
\begin{array}{ccccc}
1 & \rho & \rho ^{2} & \dots & \rho ^{n-1} \\ 
\rho & 1 & \rho & \ddots & \vdots \\ 
\rho ^{2} & \rho & 1 & \ddots & \rho ^{2} \\ 
\vdots & \ddots & \ddots & \ddots & \rho \\ 
\rho ^{n-1} & \dots & \rho ^{2} & \rho & 1%
\end{array}%
\right] }}
\end{equation*}%
Sappiamo tutto sulla matrice $V(\rho )$:%
\begin{eqnarray*}
\mid V(\rho )\mid &=&\dfrac{1}{1-\rho ^{2}} \\
V(\rho )^{-1} &=&\left[ 
\begin{array}{ccccc}
1 & -\rho & 0 & \dots & 0 \\ 
-\rho & 1+\rho ^{2} & -\rho & \ddots & \vdots \\ 
0 & -\rho & \ddots & \ddots & 0 \\ 
\vdots & \ddots & \ddots & 1+\rho ^{2} & -\rho \\ 
0 & \dots & 0 & -\rho & 1%
\end{array}%
\right]
\end{eqnarray*}
matrice tridiagonale.
\end{esempio}

Cerchiamo ora la matrice $P(\rho )$ che trasforma il modello di regressione
generalizzata in un modello di regressione classica e cio\`{e} 
\begin{equation*}
P(\rho )V(\rho )P(\rho )^{\prime }=I
\end{equation*}%
Se io considerassi la trasformazione $\tilde{\epsilon}_{t}=\epsilon
_{t}-\rho \epsilon _{t-1}$ otterrei $u_{t}$. Il modello di partenza \`{e}
come sappiamo 
\begin{eqnarray*}
y_{t} &=&\beta _{0}+\beta _{1}x_{t}+\epsilon _{t} \\
\underset{\tilde{y}_{t}}{\underbrace{y_{t}-\rho y_{t-1}}} &=&\underset{\text{%
costante}}{\underbrace{\beta _{0}(1-\rho )}}+\beta _{1}\underset{\tilde{x}%
_{t}}{\underbrace{(x_{t}-\rho x_{t-1})}}+u_{t} \\
\tilde{y}_{t} &=&\tilde{\beta}_{0}+\beta _{1}\tilde{x}_{t}+u_{t}
\end{eqnarray*}%
Quest'ultima equazione rappresenta il modello trasformato i cui parametri
possono essere stimati per mezzo dei $m.q.o$.

Problema: per l'osservazione $t=1$ non abbiamo l'osservazione precedente. Si
perde cos\`{\i} un'osservazione (grado di libert\`{a}). La soluzione esatta
consiste nel trasformare la prima osservazione nel seguente modo: 
\begin{equation*}
\sqrt{1-\rho ^{2}}y_{1}=\beta _{0}\sqrt{1-\rho ^{2}}+\beta _{1}\sqrt{1-\rho
^{2}}x_{1}+\text{errore}
\end{equation*}%
Come sar\`{a} quindi la matrice $P(\rho )$ con un campione di $n=5$
osservazioni? Riempite lo spazio a disposizione

\begin{equation*}
P(\rho )=\left[ 
\begin{array}{ccccc}
&  &  &  &  \\ 
&  &  &  &  \\ 
&  &  &  &  \\ 
&  &  &  &  \\ 
&  &  &  & 
\end{array}%
\right]
\end{equation*}

\subsection{Test di autocorrelazione}

In un problema di regressione con osservazioni temporali come possiamo
verificare la presenza di errori autocorrelati? Sappiamo che quando $\rho =0$
non c'\`{e} autocorrelazione. Questo parametro $\rho $ \`{e} una
caratteristica degli errori $\epsilon _{t}$ che tuttavia non sono
osservabili. La prima tappa dunque \`{e} quella di stimare questi errori
applicando i $m.q.o.$ sul modello, avendo cura di sempre includere la
costante. Dopodich\'{e} deriviamo una stima degli $\epsilon _{i}$: 
\begin{equation*}
\hat{\epsilon}_{t}=y_{t}-x_{t}^{\prime }\hat{\beta}
\end{equation*}%
Il test d'ipotesi $H_{0}:\rho =0$ si basa sull'esame di questi residui.

\begin{enumerate}
\item Test informale

Si fa il grafico degli errori. Se, come nel seguente esempio \newpage

\begin{equation*}
\text{Grafico}
\end{equation*}%
\vspace{5cm}

si osserva una persistenza di errori dello stesso segno, allora c'\`{e}
evidenza di autocorrelazione.

\item Test formale: Durbin-Watson

\begin{enumerate}
\item Si calcola la quantit\`{a}%
\begin{equation*}
d=\frac{\sum\limits_{t=2}^{n}\left( \widehat{\epsilon }_{t}-\widehat{%
\epsilon }_{t-1}\right) ^{2}}{\sum\limits_{t=1}^{n}\widehat{\epsilon }%
_{t}^{2}}.
\end{equation*}%
Per $n$ sufficientemente grande $d\underset{\text{circa}}{\thicksim }%
2(1-\rho )$.

Interpretazione di $d$:%
\begin{equation*}
d=\frac{\sum\limits_{t=2}^{n}\widehat{\epsilon }_{t}^{2}+\sum%
\limits_{t=2}^{n}\widehat{\epsilon }_{t-1}^{2}-2\sum\limits_{t=2}^{n}%
\widehat{\epsilon }_{t}\widehat{\epsilon }_{t-1}}{\sum\limits_{t=1}^{n}%
\widehat{\epsilon }_{t}^{2}}\simeq 2-2\frac{\frac{1}{n}\sum\limits_{t=2}^{n}%
\widehat{\epsilon }_{t}\widehat{\epsilon }_{t-1}}{\frac{1}{n}%
\sum\limits_{t=1}^{n}\widehat{\epsilon }_{t}^{2}}
\end{equation*}%
Il \emph{denumeratore} \`{e} la varianza empirica di $\epsilon _{t}$ e tende
verso la vera varianza che, come calcolato precedentemente, \`{e} uguale a $%
\sigma ^{2}/\left( 1-\rho ^{2}\right) $.

Il \emph{numeratore} \`{e} una somma di prodotti incrociati, \`{e} la
covarianza empirica fra l'errore $\epsilon _{t}$ e l'errore ritardato di un
periodo, $\epsilon _{t-1}$. Questa covarianza empirica converge verso la
vera covarianza $Cov(\epsilon _{t},\epsilon _{t-1})=\dfrac{\rho }{1-\rho ^{2}%
}\sigma ^{2}$.

Ecco spiegato perch\'{e} al crescere di $n$ la statistica $d$ converge verso 
$2(1-\rho )$. Avremo quindi per $n$ sufficientemente grande 
\begin{equation}
d\underset{\text{circa}}{\simeq }2(1-\rho ).  \label{d_circa}
\end{equation}

Abbiamo allora la seguente situazione:

\begin{itemize}
\item Esiste una forte autocorrelazione positiva quando $\rho \simeq 1$ che
per la (\ref{d_circa}) \`{e} equivalente a $d\simeq 0$.

\item Se gli $\epsilon _{t}$ sono indipendenti segue $\rho =0$. Per la (\ref%
{d_circa}) l'indipendenza implica $d\simeq 2.$

\item Esiste una forte autocorrelazione negativa quando $\rho \simeq -1$ che
per la (\ref{d_circa}) \`{e} equivalente a $d\simeq 4$.
\end{itemize}
\end{enumerate}

Per testare l'autocorrelazione positiva (che \`{e} il caso tipicamente
incontrato in economia) eseguir\`{o} il test dell'ipotesi nulla $H_{0}:\rho
=0$ $\Leftrightarrow d=2$ contro l'ipotesi alternativa $H_{0}:\rho
>0\Leftrightarrow d<2$. Si tratta quindi di un test unilaterale sinistro.

\begin{itemize}
\item Quindi, quando $d\thicksim 2$ $\Rightarrow $ assenza di
autocorrelazione.
\end{itemize}

\begin{equation*}
\text{grafico}
\end{equation*}%
\vspace*{5cm}

Idealmente vorremmo trovare la soglia $d^{\ast }$ tale che se $d<d^{\ast
}\Rightarrow $ autocorrelazione, mentre se $d>d^{\ast }\Rightarrow $ no.
Sfortunatamente la vera distribuzione della statistica $d$ \`{e}
sconosciuta. Essa dipende dai valori presenti nella matrice $X$ o, in altre
parole, dal campione osservato. Gli autori Durbin-Watson hanno potuto
trovare due limiti che delimitano il livello critico $d^{\ast }$, notati
rispettivamente $d_{L}$ e $d_{U}$. Vale la seguente regola di decisione:

$\ \ \ \ \ \ \ \ \ \ \ \ \ \ \ 
\begin{array}{l}
\text{se }d<d_{L}\text{ sar\`{a} sicuramente }d<d^{\ast }\Rightarrow \text{%
autocorrelazione} \\ 
\text{se }d>d_{U}\text{ sar\`{a} sicuramente }d>d^{\ast }\Rightarrow \text{%
assenza di autocorr.} \\ 
\text{se }d_{L}<d<d_{U}\Rightarrow \text{non possiamo concludere nulla}%
\end{array}%
$

Questi limiti $d_{L}$ e $d_{U}$ dipendono da:

\begin{itemize}
\item Livello di significativit\`{a}

\item $n$: il numero d'osservazioni

\item $K^{\prime }$: il numero di variabili esplicative senza la costante
(quindi $K^{\prime }=K-1$).
\end{itemize}
\end{enumerate}

\begin{esempio}
Abbiamo stimato una regressione semplice con $n=30,\ K^{\prime }=1$. Se $d$
fosse uguale a $1.30$, all'$1$\% di significativit\`{a} non potrei rifiutare
l'ipotesi nulla $H_{0}:\rho =0$. Se invece avessimo $K^{\prime }=2$, con un $%
d=1.30$ non si potrebbe concludere.
\end{esempio}

\begin{remark}
Attenzione:\ il test non \`{e} applicabile quando nel modello appare una
variabile endogena ritardata. Ad esempio:%
\begin{equation*}
c_{t}=a+bR_{t}+cC_{t-1}+\varepsilon _{t}
\end{equation*}
\end{remark}

Se desiderate testare l'autocorrelazione negativa si prende $4-d$ e si
procede come per l'autocorrelazione positiva.

\subsection{\label{autocorr i metodi di stima}I metodi di stima}

Abbiamo stimato il modello con i $m.q.o.$ ed il test $DW$ ci indica
autocorrelazione. Cosa fare? Il $DW$ \`{e} un allarme che indica la presenza
di una sistematicit\`{a} nel fenomeno studiato che il modello non ha saputo
incorporare. Non \`{e} necessariamente frutto dell'autocorrelazione ma
potrebbe essere causato da una cattiva specificazione del modello, ad esempio

\begin{enumerate}
\item 
\begin{itemize}
\item sono state trascurate delle variabili esplicative

\item la forma funzionale che lega le variabili esplicative alla variabile
dipendente (quadratica anzich\'{e} lineare)

\item forma dinamica invece che statica:

vero modello (DGP)\ 
\begin{equation*}
C_{t}=a+bR_{t}+cC_{t-1}+\varepsilon _{t},
\end{equation*}%
modello stimato%
\begin{equation*}
C_{t}=a+bR_{t}+\varepsilon _{t}\Leftarrow \text{ autocorrelazione}
\end{equation*}
\end{itemize}

\item Dopo aver considerato tutte le osservazioni [1], se l'autocorrelazione
persiste allora utilizziamo i $m.q.g.$

\begin{enumerate}
\item[2.1] $m.q.g.$ a due tappe

\begin{enumerate}
\item Prima tappa: stimiamo utilizzando i $m.q.o.$ e ricaviamo $\widehat{%
\varepsilon }_{t}$. In seguito costruiamo la regressione ausiliaria%
\begin{equation}
\widehat{\varepsilon }_{t}=\rho \widehat{\varepsilon }_{t-1}+u_{t}
\label{regressione_ausiliaria}
\end{equation}%
ed utilizzando nuovamente i $m.q.o.$ sul modello (\ref%
{regressione_ausiliaria}) otteniamo%
\begin{equation*}
\widehat{\rho }=\frac{\sum \widehat{\varepsilon }_{t}\widehat{\varepsilon }%
_{t-1}}{\sum \widehat{\varepsilon }_{t}^{2}}\underset{n\rightarrow \infty }{%
\longrightarrow }\rho .
\end{equation*}

\item Seconda tappa: $m.q.g.$ calcolabili

\begin{itemize}
\item Sia con la formula 
\begin{equation*}
\beta ^{\ast }=(X^{\prime }V(\widehat{\rho })^{-1}X)^{-1}X^{\prime }V(%
\widehat{\rho })^{-1}y
\end{equation*}

\item oppure con i $m.q.o.$ sul modello trasformato con $P(\widehat{\rho })$.
\end{itemize}
\end{enumerate}

\item[2.2] Quale alternativa al 2.1 eseguire una stima di massima
verosimiglianza.
\end{enumerate}
\end{enumerate}

\subsection{La previsione}

Abbiamo delle osservazioni $t=1,\ldots ,n$ e desideriamo prevedere $y$ per
il periodo successivo conoscendo $x_{n+1}$. Il vero $y_{n+1}$ \`{e} dato dal
medesimo modello:%
\begin{equation*}
y_{n+1}=x_{n+1}^{\prime }\beta +\varepsilon _{n+1}
\end{equation*}%
con 
\begin{equation*}
\varepsilon _{n+1}=\rho \varepsilon _{n}+u_{n+1}.
\end{equation*}%
Sappiamo che il previsore lineare ottimale \`{e} dato da:%
\begin{equation*}
y_{n+1}^{p}=x_{n+1}^{\prime }\beta ^{\ast }+R^{\prime }V^{-1}\widehat{%
\varepsilon }
\end{equation*}%
dove 
\begin{equation*}
\widehat{\varepsilon }=y-X\beta ^{\ast }.
\end{equation*}%
Nel nostro caso:%
\begin{equation*}
V^{-1}=\left[ 
\begin{array}{ccccc}
1 & -\rho & 0 & \dots & 0 \\ 
-\rho & 1+\rho ^{2} & -\rho & \ddots & \vdots \\ 
0 & -\rho & \ddots & \ddots & 0 \\ 
\vdots & \ddots & \ddots & 1+\rho ^{2} & -\rho \\ 
0 & \dots & 0 & -\rho & 1%
\end{array}%
\right]
\end{equation*}%
e dobbiamo calcolare 
\begin{eqnarray*}
cov(\varepsilon _{n+1},\varepsilon ) &=&E(\varepsilon _{n+1}\varepsilon
^{\prime }) \\
&=&E(\varepsilon _{n+1}\varepsilon _{1},\dots ,\varepsilon _{n+1}\varepsilon
_{n}) \\
&=&\frac{\sigma ^{2}}{1-\rho ^{2}}\left[ \rho ^{n},\rho ^{n-1},\dots ,\rho %
\right]
\end{eqnarray*}%
Un semplice calcolo mostra che 
\begin{equation*}
\underset{(1\times n)}{R^{\prime }}\underset{(n\times n)}{V^{-1}}=\underset{%
(1\times n)}{[0,0,\dots ,\rho ]}
\end{equation*}%
e quindi $R^{\prime }V^{-1}\widehat{\varepsilon }=\rho \widehat{\varepsilon }%
_{n}$ da cui segue la formula%
\begin{equation*}
y_{n+1}^{p}=x_{n+1}^{\prime }\beta ^{\ast }+\rho \widehat{\varepsilon }_{n}
\end{equation*}%
Poich\'{e} $\rho $ \`{e} sconosciuto lo sostituiamo con la stima $\widehat{%
\rho }$ ottenendo in questo modo la formula finale%
\begin{equation*}
y_{n+1}^{p}=x_{n+1}^{\prime }\beta ^{\ast }+\widehat{\rho }\widehat{%
\varepsilon }_{n}
\end{equation*}

\section{Altre forme di autocorrelazione}

Il modello statistico considerato in questo capitolo si chiama
autoregressivo di primo ordine:\ $AR(1)$%
\begin{equation*}
\epsilon _{t}=\rho \epsilon _{t-1}+u_{t},\ \ \ \ u_{t}\sim iid(0,\sigma ^{2})
\end{equation*}%
\`{E} possibile generalizzare i modelli autoregressivi ad ordini superiori.
In tal caso si parla di $AR(p)$, dove il numero intero $p$ indica appunto
l'ordine\ del modello autoregressivo%
\begin{equation*}
\epsilon _{t}=\rho _{1}\epsilon _{t-1}+\rho _{2}\epsilon _{t-2}+\ldots +\rho
_{p}\epsilon _{t-p}+u_{t}
\end{equation*}%
I modelli a media mobile (Moving Average) definiscono un'altra classe di
modelli molto utilizzati nell'analisi delle serie storiche. Un modello a
media mobile di ordine 1 $MA(1)$ \`{e} definito da%
\begin{equation*}
\epsilon _{t}=u_{t}+cu_{t-1}\ .
\end{equation*}%
Anche in questo caso \`{e} immediata la sua generalizzazione a ordini
superiori: 
\begin{equation*}
MA(q):\epsilon _{t}=u_{t}+c_{1}u_{t-1}+c_{2}u_{t-2}+\ldots +c_{q}u_{t-q}\ .
\end{equation*}%
Infine un processo stocastico generale frequentemente incontrato in economia
e statistica \`{e} l'$ARMA(p,q):$%
\begin{equation*}
\epsilon _{t}=\underset{\text{parte autoregressiva}}{\underbrace{\rho
_{1}\epsilon _{t-1}+\rho _{2}\epsilon _{t-2}+\ldots +\rho _{p}\epsilon _{t-p}%
}}+\underset{\text{parte media mobile}}{\underbrace{%
u_{t}+c_{1}u_{t-1}+c_{2}u_{t-2}+\ldots +c_{q}u_{t-q}}}
\end{equation*}

\part{La regressione stocastica}

\chapter{Elementi di analisi asintotica}

Nel contesto di una stima del modello econometrico, con l'analisi asintotica
si intende lo studio delle propriet\`{a} del campione quando la sua numerosit%
\`{a} cresce infinitamente. Perch\'{e} un tale studio?

\begin{enumerate}
\item Un motivo teorico.

Quando la numerosit\`{a} del campione aumenta, l'informazione disponibile
diventa sempre pi\`{u} importante. Col crescere dell'informazione la
precisione dello stimatore dovrebbe aumentare. Al limite per $n=\infty $
l'informazione \`{e} totale e la precisione dello stimatore dovrebbe essere
assoluta. Il metodo di stima proposto dovrebbe quindi condurre
asintoticamente al vero valore del parametro sconosciuto con assoluta
certezza (con probabilit\`{a} 1). Questo \`{e} il problema della convergenza
dello stimatore verso il vero valore del parametro.

\item Un motivo pratico

Fin qui ci siamo occupati quasi esclusivamente di stimatori lineari. Per
questi \`{e} possibile calcolare il valore atteso che ci permette di
verificare la loro correttezza. \`{E} pure possibile calcolare la varianza
dello stimatore lineare. Essa ci indica la precisione con la quale stimiamo
il parametro e ci permette di condurre dei test.

Al di l\`{a} della regressione classica saremo confrontati con stimatori non
lineari. Per questi \`{e} praticamente impossibile calcolare il valore
atteso (e quindi parlare di correttezza) e a fortiori calcolare la varianza
ed eseguire i test.

L'analisi asintotica d\`{a} una risposta a questo tipo di problemi. I
risultati saranno da interpretare come un'approssimazione molto buona per
grandi campioni.
\end{enumerate}

\section{La convergenza in probabilit\`{a}}

Prima di dare la definizione formale di convergenza in probabilit\`{a}
consideriamo il seguente esempio.

\begin{esempio}
\label{esempio media}Estraiamo un campione aleatorio da una popolazione di
variabili aleatorie ($\mu $, $\sigma ^{2}$), $\mu $ sconosciuto. Per stimare 
$\mu $ prendiamo la media del campione. Quando nel campione c'\`{e} una sola
osservazione%
\begin{equation*}
\overline{X}_{1}=X_{1},
\end{equation*}%
due osservazioni 
\begin{equation*}
\overline{X}_{2}=\frac{X_{1}+X_{2}}{2}
\end{equation*}%
e cos\`{\i} via $\cdots $ 
\begin{equation*}
\overline{X}_{n}=\frac{X_{1}+X_{2}+\cdots +X_{n}}{n}\ .
\end{equation*}%
Le quantit\`{a} $\overline{X}_{1},\overline{X}_{2},\cdots $ formano una
successione che noteremo ${\overline{X}_{n}}$. Qual \`{e} il limite di
questa sucessione quando $n\rightarrow \infty $? La legge dei grandi numeri
ci assicura che questo limite \`{e} proprio $\mu $. Possiamo dunque
concludere che la successione ${\overline{X}_{n}}$ converge verso $\mu $. 
\`{E} questo un primo esempio di convergenza in probabilit\`{a}.
\end{esempio}

\subsection{{Definizione e propriet\`{a}}}

Uno stimatore \`{e} detto convergente se la probabilit\`{a} che ci si
discosti anche solo minimamente dal suo vero valore tende a $0$ con $%
n\rightarrow \infty $. Sia $\theta $ il vero parametro, $\hat{\theta}_{n}$
uno stimatore di $\theta $ basato sul campione di numerosit\`{a} $n$.
Formalmente:

\begin{definizione}
$\hat{\theta}_{n}$ \`{e} uno stimatore convergente di $\theta $ se, $\forall
\varepsilon >0$ 
\begin{equation*}
\underset{n\rightarrow \infty }{\lim }Prob\left\{ \left\vert \hat{\theta}%
_{n}-\theta \right\vert >\varepsilon \right\} \rightarrow 0
\end{equation*}
\end{definizione}

\begin{center}
Grafico\vspace{5cm}
\end{center}

Al limite dobbiamo arrivare ad una situazione per cui tutta la massa di
probabilit\`{a} \`{e} concentrata in un solo punto, il vero valore $\theta $%
. Per indicare che il vero valore converge verso $\theta $, scriveremo $%
p\lim \ \hat{\theta}_{n}=\theta $. In certi casi \`{e} possibile valutare
esplicitamente questo limite in probabilit\`{a}, come ad esempio nel caso
della media del campione se la popolazione \`{e} normale. In altri casi
invece \`{e} necessario ricorrere ad alcune propriet\`{a} che il limite in
probabilit\`{a} $p\lim $ possiede e che enunciamo adesso.

\noindent Consideriamo le due successioni $\widehat{\theta }_{n}$ e $%
\widehat{\beta }_{n}$ con 
\begin{equation*}
p\lim \widehat{\theta }_{n}=\theta <\infty \hspace{0.5cm}\text{ e }\hspace{%
0.5cm}p\lim \widehat{\beta }_{n}=\beta <\infty .
\end{equation*}%
Il limite in probabilit\`{a} possiede\ le seguenti 4 propriet\`{a}:

\begin{enumerate}
\item[$P1$] $p\lim \left( \widehat{\theta }_{n}\pm \widehat{\beta }%
_{n}\right) =p\lim \widehat{\theta }_{n}\pm p\lim \widehat{\beta }%
_{n}=\theta \pm \beta $

\item[$P2$] $p\lim \left( \widehat{\theta }_{n}\ast \widehat{\beta }%
_{n}\right) =p\lim \widehat{\theta }_{n}\ast p\lim \widehat{\beta }%
_{n}=\theta \ast \beta $

\item[$P3$] $\beta \neq 0,$%
\begin{equation*}
p\lim \left( \frac{\widehat{\theta }_{n}}{\widehat{\beta }_{n}}\right) =%
\frac{p\lim \widehat{\theta }_{n}}{p\lim \widehat{\beta }_{n}}=\frac{\theta 
}{\beta }
\end{equation*}

\item[$P4$] Invarianza. Se $g\left( \cdot \right) $ \`{e} una funzione
continua:%
\begin{equation*}
p\lim g\left( \widehat{\theta }_{n}\right) =g\left( p\lim \widehat{\theta }%
_{n}\right) =g\left( \theta \right)
\end{equation*}
\end{enumerate}

\subsection{La convergenza in media quadratica}

Sia $\widehat{\theta }_{n}$ uno stimatore di $\theta $. Se per ogni $n$ \`{e}
possibile

\begin{itemize}
\item Calcolare il valore atteso $E(\widehat{\theta }_{n})$ e quindi anche
il termine di distorsione o \emph{Bias}\ definito come%
\begin{equation*}
B(\widehat{\theta }_{n}):=E(\widehat{\theta }_{n})-\theta =E(\widehat{\theta 
}_{n}-\theta )
\end{equation*}

\item Calcolare $V(\widehat{\theta }_{n})=E\left( \widehat{\theta }_{n}-E(%
\widehat{\theta }_{n})\right) ^{2}$
\end{itemize}

\noindent \`{e} chiaro allora che per la convergenza in probabilit\`{a} \`{e}
sufficiente che

\begin{enumerate}
\item Il Bias tenda a zero con $n\rightarrow \infty :$%
\begin{equation}
\lim_{n\rightarrow \infty }B(\widehat{\theta }_{n})=0  \label{bias_n}
\end{equation}

\item La varianza di $\widehat{\theta }_{n}$ tenda anch'essa a 0:%
\begin{equation}
\lim_{n\rightarrow \infty }V(\widehat{\theta }_{n})=0  \label{var_n}
\end{equation}
\end{enumerate}

\noindent Questo criterio \`{e} chiamato \emph{convergenza in media
quadratica}. Come andremo a dimostrare le condizioni (\ref{bias_n}) e (\ref%
{var_n}) sono equivalenti a%
\begin{equation}
\lim_{n\rightarrow \infty }E(\widehat{\theta }_{n}-\theta )^{2}=0
\end{equation}%
Dimostriamo che $\lim_{n\rightarrow \infty }E(\widehat{\theta }_{n}-\theta
)^{2}=0$ $\Leftrightarrow $ $B(\widehat{\theta }_{n})\rightarrow 0$ e $V(%
\widehat{\theta }_{n})\rightarrow 0$. Abbiamo: 
\begin{eqnarray*}
E\left( \widehat{\theta }_{n}-\theta \right) ^{2} &=&E\left[ \left( \widehat{%
\theta }_{n}-E(\widehat{\theta }_{n})\right) +\left( E(\widehat{\theta }%
_{n})-\theta \right) \right] ^{2}= \\
&=&E\left( \widehat{\theta }_{n}-E(\widehat{\theta }_{n})\right) ^{2}+B(%
\widehat{\theta }_{n})^{2}+2B(\widehat{\theta }_{n})E\left( \widehat{\theta }%
_{n}-E(\widehat{\theta }_{n})\right) = \\
&=&E\left( \widehat{\theta }_{n}-E(\widehat{\theta }_{n})\right) ^{2}+B(%
\widehat{\theta }_{n})^{2}= \\
&=&V\left( \widehat{\theta }_{n}\right) +B(\widehat{\theta }_{n})^{2}
\end{eqnarray*}%
Nell'Esempio \ref{esempio media} anche senza la normalit\`{a} di $X$ 
\begin{equation*}
\left\{ 
\begin{array}{l}
E(\overline{X}_{n})=\mu \Rightarrow B(\overline{X}_{n})=0 \\ 
V(\overline{X}_{n})=\frac{\sigma ^{2}}{n}\rightarrow 0\text{ quando }%
n\rightarrow \infty%
\end{array}%
\right.
\end{equation*}

\subsection{Convergenza dei momenti del campione}

Consideriamo una popolazione caratterizzata dalla variabile aleatoria $X$.

\begin{itemize}
\item Il primo momento della popolazione \`{e}: $E(X)=\mu $

\item Il secondo momento \`{e}: $E(X^{2})=\mu _{2}$

\item Il secondo momento centrato: $E(X-\mu )^{2}=V(X)=\sigma ^{2}$
\end{itemize}

\noindent Prendiamo un campione aleatorio di numerosit\`{a} $n$: $%
X_{1},\ldots ,X_{n}$. I momenti del campione sono:

\begin{itemize}
\item Primo momento: $\overline{X}=\frac{1}{n}\sum x_{i}$.

\item Secondo momento: $\frac{1}{n}\sum x_{i}^{2}$.

\item Secondo momento centrato: $\frac{1}{n}\sum (x_{i}-\overline{X})^{2}$
\end{itemize}

\noindent \emph{Propriet\`{a} fondamentale}: I momenti del campione
convergono in pobabilit\`{a} verso i momenti corrispondenti della
popolazione. In particolare%
\begin{eqnarray*}
p\lim \frac{1}{n}\sum X_{i} &=&E\left( X\right) =\mu \\
p\lim \frac{1}{n}\sum X_{i}^{2} &=&E\left( X^{2}\right) =\mu _{2} \\
p\lim \frac{1}{n}\sum \left( X_{i}-\overline{X}\right) ^{2} &=&V\left(
X\right) =\sigma ^{2}
\end{eqnarray*}%
Se la popolazione \`{e} caratterizzata da due variabili aleatorie $X$ e $Z$,
oltre ai momenti individuali ci sono i momenti incrociati.

\noindent Popolazione:

\begin{itemize}
\item Il seconodo momento incrociato: $E(XZ)$

\item Il secondo momento centrato incrociato: 
\begin{equation*}
E(X-\mu _{x})(Z-\mu _{Z})=Cov(X,Z)
\end{equation*}%
dove $\mu _{X}=E(X)$ e $\mu _{Z}=E(Z)$
\end{itemize}

\noindent Campione: $X_{1},\ldots ,X_{n}$ e $Z_{1},\ldots ,Z_{n}$

\begin{itemize}
\item Secondo momento incrociato: $\frac{1}{n}\sum X_{i}Z_{i}$

\item Centrato: $\frac{1}{n}\sum (X_{i}-\overline{X})(Z_{i}-\overline{Z})$
\end{itemize}

\noindent Anche in questo caso vale la propriet\`{a} fondamentale,%
\begin{eqnarray*}
p\lim \frac{1}{n}\sum X_{i}Z_{i} &=&E\left( XZ\right) \\
p\lim \frac{1}{n}\sum \left( X_{i}-\overline{X}\right) \left( Z_{i}-%
\overline{Z}\right) &=&Cov\left( X,Z\right)
\end{eqnarray*}

\begin{esempio}
La popolazione \`{e} composta da $X$, $Z$ indipendenti. Calcolare: 
\begin{equation*}
p\lim \underset{\text{Covarianza empirica}}{\underbrace{\frac{1}{n}\sum
\left( X_{i}-\overline{X}\right) \left( Z_{i}-\overline{Z}\right) }}=0=%
\underset{\text{vera covarianza}}{\underbrace{Cov\left( X,Z\right) }}
\end{equation*}
\end{esempio}

\section{La convergenza in distribuzione}

\subsection{Un esempio}

Ritornando alle medie del campione abbiamo "dimostrato" che, 
\begin{eqnarray*}
\overline{X}_{n} &\rightarrow &\mu \\
\left( \overline{X}_{n}-\mu \right) &\rightarrow &0
\end{eqnarray*}%
Ci\`{o} significa che la distribuzione limite di $\overline{X}_{n}-\mu $ 
\`{e} degenerata (tutta la massa di probabilit\`{a} \`{e} centrata su 0). Se
ammettiamo la normalit\`{a} della popolazione 
\begin{eqnarray*}
\left( \overline{X}_{n}-\mu \right) &\sim &N(0,\frac{\sigma ^{2}}{n}) \\
\sqrt{n}\left( \overline{X}_{n}-\mu \right) &\sim &N\left( 0,\sigma
^{2}\right)
\end{eqnarray*}%
Notiamo che la distribuzione di $\sqrt{n}\left( \overline{X}_{n}-\mu \right) 
$ \`{e} sempre la stessa al variare di $n$. Anche al limite sar\`{a} sempre
una distribuzione normale, $N\left( 0,\sigma ^{2}\right) $. In questo caso
la distribuzione di $\left( \overline{X}_{n}-\mu \right) $ \`{e} degenerata,
ma quella di $\left( \overline{X}_{n}-\mu \right) $ dilatata per $\sqrt{n}$ 
\`{e} una normale di varianza $\sigma ^{2}$ e valore atteso $0$.

\noindent Se la popolazione non \`{e} normale, il teorema del limite
centrale ci assicura che la distribuzione limite di $\sqrt{n}\left( 
\overline{X}_{n}-\mu \right) $ \`{e} normale $\left( 0,\sigma ^{2}\right) $.
Scriveremo: 
\begin{equation*}
\sqrt{n}\left( \overline{X}_{n}-\mu \right) \overset{D}{\sim }N\left(
0,\sigma ^{2}\right)
\end{equation*}

\subsection{Teoremi del limite centrale}

Prendendo $X_{i}-\mu $ dall'esempio precedente, chiamiamo $\varepsilon
_{i}=X_{i}-\mu $, con $E(\varepsilon _{i})=0$ e $V(\varepsilon _{i})=\sigma
^{2}$. Dato il carattere aleatorio del campione si ha che: 
\begin{equation*}
\varepsilon _{i}\sim i.i.d.(0,\sigma ^{2})
\end{equation*}%
D'altra parte 
\begin{eqnarray*}
\sqrt{n}\left( \overline{X}_{n}-\mu \right) &=&\sqrt{n}\left( \frac{1}{n}%
\sum X_{i}-\mu \right) =\sqrt{n}\left( \frac{1}{n}\sum \left( X_{i}-\mu
\right) \right) \\
&=&\sqrt{n}\left( \frac{1}{n}\sum \varepsilon _{i}\right) =\frac{1}{\sqrt{n}}%
\sum \varepsilon _{i}
\end{eqnarray*}

\begin{theorem}
Limite centrale semplice. Condizione:%
\begin{equation*}
\varepsilon _{i}\sim i.i.d.(0,\sigma ^{2}).
\end{equation*}

Ne segue che 
\begin{equation*}
\frac{1}{\sqrt{n}}\sum \varepsilon _{i}\overset{D}{\sim }N\left( 0,\sigma
^{2}\right) .
\end{equation*}
\end{theorem}

\noindent Consideriamo ora i vettori 
\begin{equation*}
\underset{(K\times 1)}{v_{i}}\underset{(1\times 1)}{\varepsilon _{i}}
\end{equation*}%
dove i $v_{i}$ sono composti da numeri (non aleatori). Abbiamo quindi: 
\begin{eqnarray*}
E[v_{i}\varepsilon _{i}] &=&v_{i}E[\varepsilon _{i}]=v_{i}\ 0=0 \\
V[v_{i}\varepsilon _{i}] &=&v_{i}V[\varepsilon _{i}]v_{i}^{\prime }=\sigma
^{2}\underset{(K\times K)}{v_{i}v_{i}^{\prime }}
\end{eqnarray*}

\begin{theorem}
Limite centrale generalizzato. Condizioni:
\end{theorem}

\begin{enumerate}
\item $\varepsilon _{i}\sim i.i.d.(0,\sigma ^{2})$

\item $v_{i}$ sono dei vettori non stocastici $(K\times 1)$ tali che
\end{enumerate}

\begin{equation*}
\lim_{n\rightarrow \infty }\frac{1}{n}\sum\limits_{i=1}^{n}V(v_{i}%
\varepsilon _{i})=\lim_{n\rightarrow \infty }\frac{1}{n}\sum%
\limits_{i=1}^{n}\sigma ^{2}v_{i}v_{i}^{\prime }=Q,\text{matrice d.p.}
\end{equation*}

Ne segue che%
\begin{equation*}
\Longrightarrow \frac{1}{\sqrt{n}}\sum\limits_{i=1}^{n}v_{i}\varepsilon _{i}%
\overset{D}{\sim }N(0,Q)
\end{equation*}

\begin{theorem}
Prodotto. Dati:
\end{theorem}

\begin{enumerate}
\item Una successione di vettori aleatori $\left\{ Y_{n}\right\} $ d'ordine $%
K$ che tende in distribuzione verso la variabile $Y\sim N(0,Q)$, dove $Q$ 
\`{e} una matrice definita positiva, $Y_{n}\overset{D}{\sim }Y$ ed $Y\sim
N(0,Q)$.

\item Una successione di matrici $\left\{ A_{n}\right\} $ d'ordine $(m\times
K)$ di elementi aleatori o meno, tale che $plimA_{n}=A$, $A$ matrice non
stocastica di rango $m$.
\end{enumerate}

Ne segue che la successione dei prodotti $\left\{ A_{n}Y_{n}\right\} $ tende
in distribuzione verso

la variabile aleatoria $AY\sim N(0,AQA^{\prime })$, e cio\`{e} 
\begin{equation*}
A_{n}Y_{n}\overset{D}{\sim }N(0,AQA^{\prime })
\end{equation*}

\subsection{Consiglio pratico}

In pratica, per uno stimatore $\hat{\theta}_{n}$ del parametro $\theta $, si
deve verificare dapprima la convergenza: 
\begin{equation*}
p\lim \hat{\theta}_{n}=\theta
\end{equation*}%
o equivalentemente 
\begin{equation*}
p\lim (\hat{\theta}_{n}-\theta )=0
\end{equation*}%
Poi si eseguono le seguenti tappe:

\begin{itemize}
\item[{[1]}] La distribuzione asintotica di $\hat{\theta}_{n}$ \`{e} la
distribuzione limite della successione $\left\{ \sqrt{n}(\hat{\theta}%
_{n}-\theta )\right\} $ che in moltissimi casi sar\`{a} normale. Per
trovarla si applicano i teoremi 1, 2 e 3.

\item[{[2]}] La varianza asintotica dello stimatore $\hat{\theta}_{n}$, che
noteremo $VarAs(\hat{\theta}_{n})$ \`{e} la varianza della distribuzione
asintotica divisa per $n$.

\begin{remark}
Lo stimatore $\hat{\theta}_{n}$ \`{e} asintoticamente efficiente se \`{e}
convergente e asintoticamente normale a varianza asintotica minima nella
classe di tutti gli stimatori convergenti e asintoticamente normali (BAN).
Lo stimatore di massima verosimiglianza \`{e} asintoticamente efficiente.
\end{remark}
\end{itemize}

\section{Analisi asintotica della regressione classica}

\begin{enumerate}
\item {Il modello} 
\begin{eqnarray*}
y_{i} &=&x_{i}^{\prime }\beta +\varepsilon _{i} \\
y &=&X\beta +\varepsilon
\end{eqnarray*}

\item {Le ipotesi}

H1: Sulle variabili esplicative

\begin{enumerate}
\item Non stocastiche

\item $rg(X)=K$

\item Nuovo: comportamento asintotico 
\begin{equation*}
\lim_{n\rightarrow \infty }\frac{1}{n}\underset{(K\times K)}{X^{\prime }X}%
=Q,\ \ d.p.
\end{equation*}
\end{enumerate}

H2: Sugli errori

\begin{enumerate}
\item $E[\varepsilon ]=0$

\item $V[\varepsilon ]=\sigma ^{2}I$

\item Nuovo: $\varepsilon _{i}\sim i.i.d.(0,\sigma ^{2})$
\end{enumerate}

H3: $X,\varepsilon $ mutuamente indipendenti.

\item {Lo stimatore dei }${m.q.o.}$%
\begin{eqnarray*}
\hat{\beta} &=&(X^{\prime }X^{\prime })^{-1}X^{\prime }Y \\
E[\hat{\beta}] &=&\beta \\
V[\hat{\beta}] &=&\sigma ^{2}(X^{\prime }X)^{-1}
\end{eqnarray*}

\item {La convergenza quadratica}

\begin{itemize}
\item La distorsione: $B(\hat{\beta})=0$

\item La varianza: $V(\hat{\beta})=\sigma ^{2}(X^{\prime }X)^{-1}=\frac{%
\sigma ^{2}}{n}(\frac{X^{\prime }X}{n})^{-1}$

Segue che :%
\begin{equation*}
\lim \frac{\sigma ^{2}}{n}\ \lim (\frac{X^{\prime }X}{n})^{-1}=\lim \frac{%
\sigma ^{2}}{n}Q^{-1}=0
\end{equation*}
\end{itemize}

\item {Prova diretta della convergenza di $\hat{\beta}$} 
\begin{eqnarray*}
\widehat{\beta } &=&\left( X^{\prime }X\right) ^{-1}X^{\prime }Y \\
&=&\beta +\left( X^{\prime }X\right) ^{-1}X^{\prime }\varepsilon \\
p\lim \widehat{\beta } &=&p\lim \beta +p\lim \left( X^{\prime }X\right)
^{-1}X^{\prime }\varepsilon \\
&=&\beta +p\lim \left( \frac{X^{\prime }X}{n}\right) ^{-1}p\lim \left( \frac{%
X^{\prime }\varepsilon }{n}\right) \\
&=&\beta +Q^{-1}p\lim \left( \frac{X^{\prime }\varepsilon }{n}\right)
\end{eqnarray*}%
Ora:%
\begin{equation*}
\frac{1}{n}\underset{(K\times 1)}{X^{\prime }\varepsilon }=\left[ 
\begin{array}{c}
\frac{1}{n}\sum X_{i1}\varepsilon _{i} \\ 
\frac{1}{n}\sum X_{i2}\varepsilon _{i} \\ 
\vdots \\ 
\frac{1}{n}\sum X_{iK}\varepsilon _{i}%
\end{array}%
\right]
\end{equation*}%
ogni termine \`{e} una somma di prodotti incrociati divisi per $n$. Siccome
un termine del prodotto \`{e} nullo in valore atteso, questa somma
rappresenta la covarianza empirica nel campione tra una variabile
esplicativa ($k-esima$) e gli errori. Questa tende verso la vera covarianza
che \`{e} $0$ poich\'{e} le variabili esplicative sono indipendenti dagli
errori.

Segue che: 
\begin{eqnarray*}
p\lim \frac{1}{n}X^{\prime }\varepsilon &=&0 \\
p\lim \hat{\beta} &=&\beta ,\text{ convergente}
\end{eqnarray*}

\item {Normalit\`{a} asintotica}

Si cerca la distribuzione limite di $\sqrt{n}(\hat{\beta}-\beta )$. Ora: 
\begin{equation*}
\hat{\beta}-\beta =(X^{\prime }X)^{-1}X^{\prime }\varepsilon \\
=(\frac{1}{n}X^{\prime }X)^{-1}\frac{1}{n}X^{\prime }\varepsilon
\end{equation*}%
Ne consegue che, 
\begin{equation*}
\sqrt{n}(\hat{\beta}-\beta )=(\frac{1}{n}X^{\prime }X)^{-1}\frac{1}{\sqrt{n}}%
X^{\prime }\varepsilon .
\end{equation*}%
Si tratta di un prodotto di due termini.

\begin{itemize}
\item Il primo termine, $(\frac{1}{n}X^{\prime }X)^{-1}$, tende verso la
matrice di costanti, $Q^{-1}$.

\item Il secondo termine pu\`{o} scriversi:%
\begin{equation*}
\frac{1}{\sqrt{n}}X^{\prime }\varepsilon =\frac{1}{\sqrt{n}}\left[ 
\begin{array}{cccc}
x_{1} & x_{2} & \cdots & x_{n}%
\end{array}%
\right] \left[ 
\begin{array}{c}
\varepsilon _{1} \\ 
\varepsilon _{2} \\ 
\vdots \\ 
\varepsilon _{n}%
\end{array}%
\right] =\frac{1}{\sqrt{n}}\sum\limits_{i=1}^{n}x_{i}\varepsilon _{i}
\end{equation*}
\end{itemize}

Quest'espressione \`{e} proprio quella del teorema 2: verifichiamo le due
condizioni.

\begin{enumerate}
\item $\varepsilon _{i}\sim i.i.d.(0,\sigma ^{2})$

\item $x_{i}$ non stocastici, tali che 
\begin{eqnarray*}
\lim \frac{1}{n}\sum V(x_{i}\varepsilon _{i}) &=&\lim \frac{1}{n}\sum
x_{i}\sigma ^{2}x_{i}^{\prime }= \\
&=&\sigma ^{2}\lim \frac{1}{n}\sum x_{i}x_{i}^{\prime }= \\
&=&\sigma ^{2}\lim \frac{1}{n}X^{\prime }X=\sigma ^{2}Q,d.p.
\end{eqnarray*}
\end{enumerate}

Per il teorema 2: $\frac{1}{\sqrt{n}}X^{\prime }\varepsilon \overset{D}{\sim 
}N(0,\sigma ^{2}Q)$, e per il teorema del prodotto: 
\begin{equation*}
\sqrt{n}(\hat{\beta}-\beta )\overset{D}{\sim }N(0,\sigma ^{2}Q^{-1}).
\end{equation*}%
Perch\'{e} questa varianza? Siccome $\sqrt{n}(\hat{\beta}-\beta )\overset{D}{%
\sim }Q^{-1}\cdot N(0,\sigma ^{2}Q)$, $Var=Q^{-1}(\sigma
^{2}Q)(Q^{-1})^{\prime }=\sigma ^{2}Q^{-1}$.

La \emph{varianza asintotica} come definita in precedenza \`{e} la varianza
della distribuzione asintotica divisa per $n$:%
\begin{equation*}
VarAs(\hat{\beta})=\frac{\sigma ^{2}}{n}Q^{-1}
\end{equation*}
\end{enumerate}

\chapter{Il modello di regressione stocastica}

\section{Il modello e le ipotesi}

\subsection{Il modello}

Sin qui abbiamo ammesso che le variabili esplicative sono non stocastiche,
dei numeri. Quest'ipotesi \`{e} valida nelle scienze sperimentali dove colui
che conduce l'analisi pu\`{o} scegliere debitamente le variabili
esplicative. In economia questa ipotesi non \`{e} sempre soddisfatta. Quando
le variabili esplicative sono (almeno in parte) aleatorie i metodi di stima
e le loro propriet\`{a} dipendono non soltanto dalla distribuzione degli
errori, ma dalla \emph{distribuzione congiunta} delle variabili esplicative
e degli errori. In particolare, tali le propriet\`{a} dipenderanno
dall'indipendenza o meno tra le variabili esplicative e l'errore.

\noindent In certi casi l'indipendenza tra variabili esplicative ed errori 
\`{e} una conseguenza logica del modello studiato.

\begin{esempio}
Il modello keynesiano semplice

Funzione di consumo: $C_{t}=a+bR_{t}+\varepsilon _{t}$

La definizione del reddito: $R_{t}=C_{t}+I_{t}$

$R_{t}$ \`{e} la variabile esplicativa. Tuttavia $R_{t}$ e $\varepsilon _{t}$
non possono essere indipendenti in quanto, dalla definizione di reddito, $%
R_{t}$ contiene $C_{t}$ il quale contiene $\varepsilon _{t}$.
\end{esempio}

\subsection{Le ipotesi}

Partiamo come al solito dal modello di regressione%
\begin{eqnarray*}
y_{i} &=&x_{i}^{\prime }\beta +\varepsilon _{i}\text{ \ \ }i=1,...,n \\
&&\text{oppure} \\
y &=&X\beta +\varepsilon
\end{eqnarray*}

\bigskip \noindent H1: Sulle variabili esplicative

\begin{enumerate}
\item Stocastiche (almeno in parte)

\item $rg(X)=K$, con probabilit\`{a} 1

\item $p\lim \frac{1}{n}X^{\prime }X=Q_{X}$, definita positiva
\end{enumerate}

\bigskip \noindent H2: Sugli errori

\begin{enumerate}
\item[ ] $\varepsilon \sim i.i.d.(0,\sigma ^{2})$
\end{enumerate}

\bigskip \noindent H3: Sulla relazione tra $X$ e $\varepsilon $. Due casi:

\begin{enumerate}
\item Indipendenza: $p\lim \frac{1}{n}X^{\prime }\varepsilon =0$

\item Dipendenza: $p\lim \frac{1}{n}X^{\prime }\varepsilon =C\neq 0$
\end{enumerate}

\section{Quando i regressori e gli errori sono indipendenti}

Tutto funziona come nella regressione classica.

\section{Quando sono correlati}

\subsection{{La non convergenza dei }${m.q.o.}$}

\`{E} immediato verificare che i ${m.q.o.}$ non sono convergenti quando le
variabili esplicative sono correlate con gli errori:%
\begin{eqnarray*}
\widehat{\beta } &=&\beta +(X^{\prime }X)^{-1}X^{\prime }\varepsilon \\
&=&\beta +(\frac{1}{n}X^{\prime }X)^{-1}\frac{1}{n}X^{\prime }\varepsilon \\
p\lim \widehat{\beta } &=&\beta +Q^{-1}C\neq 0.
\end{eqnarray*}%
In tal caso come bisogna procedere?\ Prima di proporre un nuovo stimatore
consideriamo {una reinterpretazione dello stimatore dei minimi quadrati. }%
Premoltiplichiamo il modello 
\begin{equation*}
y=X\beta +\varepsilon
\end{equation*}%
per la matrice $X^{\prime }$ e dividiamo per $n$%
\begin{equation}
\frac{1}{n}X^{\prime }y=\frac{1}{n}X^{\prime }X\beta +\frac{1}{n}X^{\prime
}\varepsilon  \label{esempio_iv_mqo}
\end{equation}%
Quando $X$ e $\varepsilon $ sono indipendenti $\frac{1}{n}X^{\prime
}\varepsilon \rightarrow 0$ e lo si pu\`{o} tralasciare. Risolvendo (\ref%
{esempio_iv_mqo}) rispetto a $\beta $ otteniamo cos\`{\i} lo stimatore dei
minimi quadrati ordinari: 
\begin{equation*}
\hat{\beta}=(X^{\prime }X)^{-1}X^{\prime }y.
\end{equation*}%
Quando, invece, $X$ ed $\varepsilon $ sono correlati, il termine $\frac{1}{n}%
X^{\prime }\varepsilon $ non tende a zero, ed \`{e} questa la ragione della
non convergenza al vero valore. Viene per\`{o} l'idea di premoltiplicare il
modello non per $X^{\prime }$, ma per una matrice $Z^{\prime }$ (dello
stesso ordine $K\times n$ che $X^{\prime }$) non correlata con $\varepsilon $%
, 
\begin{equation*}
\frac{1}{n}Z^{\prime }y=\frac{1}{n}Z^{\prime }X\beta +\frac{1}{n}Z^{\prime
}\varepsilon
\end{equation*}%
Se allora la matrice $Z^{\prime }X$ \`{e} non singolare (ci\`{o} che si deve
ammettere), risolvendo per $\beta $ si ottiene, 
\begin{equation}
\hat{\beta}_{vs}=(Z^{\prime }X)^{-1}Z^{\prime }y  \label{stimatore vs kxk}
\end{equation}%
Per concludere, la matrice $Z$ \`{e} chiamata matrice di strumenti e lo
stimatore definito dalla (\ref{stimatore vs kxk}) \`{e} chiamato stimatore
delle variabili strumentali. \`{E} questo il caso in cui il numero di
strumenti (il numero di colonne della matrice $Z$) \`{e} uguale al numero di
variabili esplicative e cio\'{e} $K$.

\begin{esempio}
Ritornando all'esempio per la funzione di consumo%
\begin{equation*}
y=\left[ 
\begin{array}{c}
C_{1} \\ 
\vdots \\ 
C_{n}%
\end{array}%
\right] ;\hspace{0.5cm}X=\left[ 
\begin{array}{cc}
1 & R_{1} \\ 
\vdots & \vdots \\ 
1 & R_{n}%
\end{array}%
\right] ,\hspace{0.5cm}p\lim \frac{1}{n}\sum R_{i}\varepsilon _{i}\neq 0
\end{equation*}%
I $m.q.o.$ non sono convergenti! Devo trovare due strumenti:
\end{esempio}

\begin{itemize}
\item Il termine costante non \`{e} stocastico: \`{e} il suo proprio
strumento.

\item Dobbiamo trovare invece uno strumento per $R_{t}$. Ammettiamo che
l'investimento sia esogeno, dunque $I_{t}$ indipendente da $\varepsilon _{t}$%
. $R_{t}$ \`{e} sicuramente correlato con $I_{t}$, dato che $%
R_{t}=C_{t}+I_{t}$. Dunque la matrice $Z$ dei due strumenti \`{e} data da: 
\begin{equation*}
Z=\left[ 
\begin{array}{cc}
1 & I_{1} \\ 
\vdots & \vdots \\ 
1 & I_{n}%
\end{array}%
\right]
\end{equation*}
\end{itemize}

\noindent Sciogliamo l'espressione 
\begin{equation*}
\hat{\beta}_{vs}=(Z^{\prime }X)^{-1}Z^{\prime }y\Longleftrightarrow
(Z^{\prime }X){\beta }=Z^{\prime }y
\end{equation*}%
Introduciamo i valori nell'espressione: 
\begin{eqnarray*}
\left[ 
\begin{array}{ccc}
1 & \cdots & 1 \\ 
I_{1} & \cdots & I_{n}%
\end{array}%
\right] \left[ 
\begin{array}{cc}
1 & R_{1} \\ 
\vdots & \vdots \\ 
1 & R_{n}%
\end{array}%
\right] \left[ 
\begin{array}{c}
a \\ 
b%
\end{array}%
\right] &=&\left[ 
\begin{array}{ccc}
1 & \cdots & 1 \\ 
I_{1} & \cdots & I_{n}%
\end{array}%
\right] \left[ 
\begin{array}{c}
C_{1} \\ 
\vdots \\ 
C_{n}%
\end{array}%
\right] \\
\left[ 
\begin{array}{cc}
n & \sum R_{t} \\ 
\sum I_{t} & \sum R_{t}I_{t}%
\end{array}%
\right] \left[ 
\begin{array}{c}
a \\ 
b%
\end{array}%
\right] &=&%
\begin{array}{c}
\sum C_{t} \\ 
\sum I_{t}C_{t}%
\end{array}%
\end{eqnarray*}%
\begin{equation*}
\left\{ 
\begin{array}{c}
na+b\sum R_{t}=\sum C_{t} \\ 
a\sum I_{t}+b\sum R_{t}I_{t}=\sum I_{t}C_{t}%
\end{array}%
\right.
\end{equation*}%
Risolviamo il sistema e troviamo lo stimatore di $a$. 
\begin{eqnarray*}
\widehat{a}_{vs} &=&\frac{1}{n}\sum C_{t}-\widehat{b}_{vs}\frac{1}{n}\sum
R_{t} \\
\widehat{a}_{vs} &=&\overline{C}-\widehat{b}_{vs}\overline{R}
\end{eqnarray*}%
Troviamo ora lo stimatore per $b$.%
\begin{eqnarray*}
\sum I_{t}(\overline{C}-\widehat{b}_{vs}\overline{R})+\sum I_{t}R_{t}%
\widehat{b}_{vs} &=&\sum C_{t}I_{t} \\
\underset{m_{IR}}{\underbrace{\left[ \sum I_{t}R_{t}-\overline{R}\sum I_{t}%
\right] }}\widehat{b}_{vs} &=&\underset{m_{IC}}{\underbrace{\sum C_{t}I_{t}-%
\overline{C}\sum I_{t}}}
\end{eqnarray*}%
\begin{equation*}
\widehat{b}_{vs}=\frac{m_{IC}}{m_{IR}}\neq \widehat{b}=\frac{m_{CR}}{m_{RR}}
\end{equation*}

\subsection{Il metodo delle variabili strumentali}

Abbiamo discusso precedentemente il caso particolare dove il numero di
strumenti utilizzati (il numero di colonne della matrice $Z$)\ era uguale a $%
K$, ovvero uguale al numero di variabili esplicative. {Consideriamo ora la
procedura generale nella quale il numero di strumenti pu\`{o} essere
superiore a }$K$.

\subsubsection{Gli strumenti}

Si ammette l'esistenza di una matrice $Z$ d'ordine $n\times m$ di strumenti
tali che

\begin{enumerate}
\item Gli strumenti non sono correlati con gli errori: 
\begin{equation*}
p\lim \frac{1}{n}\underset{(m\times 1)}{{Z^{\prime }\varepsilon }}=\underset{%
(m\times 1)}{0}
\end{equation*}

\item Gli strumenti sono correlati con le variabili esplicative: 
\begin{equation*}
p\lim \frac{1}{n}\underset{(m\times K)}{{Z^{\prime }X}}=\underset{(m\times K)%
}{Q_{ZX}}
\end{equation*}%
dove $Q_{ZX}$ \`{e} di rango $K$. Ovviamente deve essere $m\geq K$. Questa
condizione essenziale \`{e} detta condizione d'identificazione.

\item Gli strumenti posseggono dei momenti finiti di ordine 2. 
\begin{equation*}
p\lim \frac{1}{n}\sum {Z^{\prime }Z}=Q_{Z},\hspace{0.5cm}d.p.
\end{equation*}
\end{enumerate}

\noindent Nell'esempio:

\begin{enumerate}
\item La costante \`{e} indipendente da $\varepsilon _{t}$, e $I_{t}$ \`{e}
indipendente da $\varepsilon _{t}$.

\item 
\begin{equation*}
\frac{1}{n}\sum {Z^{\prime }X}=\left[ 
\begin{array}{cc}
1 & \overline{R} \\ 
\overline{I} & \frac{1}{n}\sum R_{t}I_{t}%
\end{array}%
\right] ,\text{ di rango 2}\Longleftrightarrow m_{IR}\neq 0
\end{equation*}

\item 
\begin{equation*}
\frac{1}{n}Z^{\prime }Z=\left[ 
\begin{array}{cc}
1 & \overline{I} \\ 
\overline{I} & \frac{1}{n}\sum I_{t}^{2}%
\end{array}%
\right] ,\text{ di rango 2 }(d.p.)\Longleftrightarrow m_{ZZ}\neq 0
\end{equation*}
\end{enumerate}

\subsubsection{La procedura}

Si premoltiplica il modello $y=X\beta +\varepsilon $ per $\underset{(m\times
n)}{Z^{\prime }}$: 
\begin{eqnarray*}
Z^{\prime }y &=&Z^{\prime }X\beta +Z^{\prime }\varepsilon \\
\underset{(m\times 1)}{\tilde{y}} &=&\underset{(m\times K)}{\tilde{X}}\beta +%
\underset{(m\times 1)}{\tilde{\varepsilon}}
\end{eqnarray*}%
Questo ha tutti gli aspetti di un modello di regressione con $m$
osservazioni e $K$ variabili esplicative $m\geq K$. In questa regressione
gli errori $\tilde{\varepsilon}$ non hanno le propriet\`{a} dei $m.q.o.$.
Infatti 
\begin{eqnarray*}
\tilde{\varepsilon} &=&Z^{\prime }\varepsilon \\
E\left[ \widetilde{\varepsilon }\mid Z\right] &=&0 \\
V\left[ \widetilde{\varepsilon }\mid Z\right] &=&Z^{\prime }V(\varepsilon
)Z=\sigma ^{2}Z^{\prime }Z
\end{eqnarray*}%
Questa varianza \`{e} conosciuta. Siamo dunque in una situazione in cui i $%
m.q.g.$ puri sono applicabili. L'applicazione dei $m.q.g.$ puri ci d\`{a}: 
\begin{eqnarray*}
\widehat{\beta }_{vs} &=&\left[ \widetilde{X}^{\prime }V^{-1}\widetilde{X}%
\right] ^{-1}\widetilde{X}^{\prime }V^{-1}\widetilde{y} \\
&=&\left( X^{\prime }Z(Z^{\prime }Z)^{-1}Z^{\prime }X\right) ^{-1}X^{\prime
}Z(Z^{\prime }Z)^{-1}Z^{\prime }y \\
&=&(X^{\prime }P_{Z}X)^{-1}X^{\prime }P_{Z}y.
\end{eqnarray*}%
Quando $m=K$, $X^{\prime }Z$ \`{e} quadrata e invertibile per ipotesi:%
\begin{eqnarray*}
\widehat{\beta }_{vs} &=&(Z^{\prime }X)^{-1}Z^{\prime }Z(X^{\prime
}Z)^{-1}X^{\prime }Z(Z^{\prime }Z)^{-1}Z^{\prime }y \\
&=&(Z^{\prime }X)^{-1}Z^{\prime }y
\end{eqnarray*}

\subsection{La scelta degli strumenti}

Dal punto di vista teorico: gli strumenti devono essere

\begin{itemize}
\item non correlati con gli errori e

\item il pi\`{u} correlati possibile con le variabili esplicative.\vspace{%
0.5cm}
\end{itemize}

\noindent Dal punto di vista pratico:

\begin{itemize}
\item Le variabili esplicative non stocastiche (come la costante, il trend,
le dummy) come pure quelle non correlate con gli errori, sono i loro propri
strumenti.

\item Nei modelli macroeconomici, le variabili \emph{esogene del sistema}
(nel nostro caso l'investimento) costituiscono gli strumenti naturali.

\item In certi casi in economia si pu\`{o} ammettere che il valore ritardato
delle variabili esplicative non sia correlato con l'errore presente. In tal
caso il valore ritardato \`{e} uno strumento valido.

\item L'ordine con cui gli strumenti sono posizionati nelle colonne di $Z$
non ha alcuna conseguenza.
\end{itemize}

\subsubsection{{Le propriet\`{a} di $\hat{\protect\beta}_{vs}$}}

Scriviamo, sostituendo $y$ con $X\beta +\varepsilon $, il nostro stimatore 
\begin{eqnarray*}
\widehat{\beta }_{vs} &=&\left( X^{\prime }Z(Z^{\prime }Z)^{-1}Z^{\prime
}X\right) ^{-1}X^{\prime }Z(Z^{\prime }Z)^{-1}Z^{\prime }(X\beta
+\varepsilon ) \\
&=&\beta +\left( X^{\prime }Z(Z^{\prime }Z)^{-1}Z^{\prime }X\right)
^{-1}X^{\prime }Z(Z^{\prime }Z)^{-1}Z^{\prime }\varepsilon
\end{eqnarray*}%
Non essendo lineare in $\varepsilon $ e $X$ non si pu\`{o} calcolare n\'{e}
il valore atteso n\'{e} la varianza.

\begin{itemize}
\item La convergenza in probabilit\`{a} 
\begin{eqnarray*}
\widehat{\beta }_{vs} &=&\beta +\left[ \frac{1}{n}X^{\prime }Z(\frac{1}{n}%
Z^{\prime }Z)^{-1}\frac{1}{n}Z^{\prime }X\right] ^{-1}\frac{1}{n}X^{\prime
}Z(\frac{1}{n}Z^{\prime }Z)^{-1}\frac{1}{n}Z^{\prime }\varepsilon \\
&=&\beta +A_{n}\frac{1}{n}Z^{\prime }\varepsilon
\end{eqnarray*}

Ora:

\begin{itemize}
\item $p\lim \frac{1}{n}Z^{\prime }\varepsilon =p\lim \frac{1}{n}%
\sum_{i=1}^{n}z_{i}\varepsilon _{i}=\ldots \ldots \ldots =0$

\item $p\lim A_{n}=\left[ Q_{XZ}Q_{Z}^{-1}Q_{ZX}\right]
^{-1}Q_{XZ}Q_{Z}^{-1}=A$, finito
\end{itemize}

Ne segue che: 
\begin{equation*}
p\lim \hat{\beta}=\beta
\end{equation*}

\item La normalit\`{a} asintotica 
\begin{equation*}
\sqrt{n}(\widehat{\beta }-\beta )=\underset{p\lim A_{n}=A}{\underbrace{A_{n}}%
}\underset{N(0,\sigma ^{2}Q_{Z})}{\underbrace{\frac{1}{\sqrt{n}}Z^{\prime
}\varepsilon }}
\end{equation*}%
Per il teorema 3: 
\begin{equation*}
\sqrt{n}(\widehat{\beta }-\beta )\overset{D}{\sim }N(0,\sigma
^{2}AQ_{Z}A^{\prime })
\end{equation*}%
Ora%
\begin{eqnarray*}
AQ_{Z}A^{\prime } &=&\left[ Q_{XZ}Q_{Z}^{-1}Q_{ZX}\right]
^{-1}Q_{XZ}Q_{Z}^{-1}(Q_{Z})Q_{Z}^{-1}\ldots \\
&=&\left[ Q_{XZ}Q_{Z}^{-1}Q_{ZX}\right] ^{-1}
\end{eqnarray*}

\item {La varianza asintotica} 
\begin{equation*}
VarAs(\hat{\beta}_{vs})=\frac{\sigma ^{2}}{n}\left[ Q_{XZ}Q_{Z}^{-1}Q_{ZX}%
\right] ^{-1}
\end{equation*}%
Sar\`{a} stimata in maniera convergente con: 
\begin{equation*}
\widehat{VarAs}(\hat{\beta}_{vs})=\hat{\sigma}^{2}\left[
Q_{XZ}Q_{Z}^{-1}Q_{ZX}\right] ^{-1}
\end{equation*}%
dove: 
\begin{eqnarray*}
\hat{\sigma}^{2} &=&\frac{1}{n}\hat{\varepsilon}^{\prime }\hat{\varepsilon}
\\
\hat{\varepsilon} &=&y-X\hat{\beta}_{vs}
\end{eqnarray*}
\end{itemize}

\part{Modelli dinamici}

\chapter{Variabili endogene ritardate}

I modelli di regressione considerati fino a questo momento sono di \emph{%
natura statica}, nel senso che sia la variabile dipendente che le variabili
esplicative (di cui essa ne \`{e} funzione) sono osservate al medesimo
istante temporale. Per contro, un modello \emph{dinamico} implica una
relazione non contemporanea fra la variabile dipendente ed una o pi\`{u}
variabili esplicative. Un semplice esempio potrebbe essere il seguente:%
\begin{equation}
y_{t}=bx_{t-3}+\varepsilon _{t}.  \label{esempio modello dinamico semplice}
\end{equation}%
In questo caso l'effetto di una variazione in $x$ \`{e} trasmesso ad $y$
solo dopo tre periodi. Un modello economico \`{e} detto dinamico se \`{e}
capace di evolvere nel tempo di sua propria iniziativa. Una semplice
definizione di modello dinamico \`{e} la seguente:

\begin{definizione}
Un modello economico \`{e} detto dinamico se nell'equazione che lo definisce
intervengono variabili datate a momenti diversi nel tempo.
\end{definizione}

Sebbene il modello (\ref{esempio modello dinamico semplice}) sia per
definizione dinamico, esso non ne cattura pienamente lo spirito in quanto
l'effetto che $x$ esercita su $y$ \`{e} realizzato in un'unica tappa (e per
questo motivo non \`{e} di grande interesse). Nel Capitolo \ref{Capitolo
ritardi distribuiti} vedremo modelli pi\`{u} ricchi dal punto di vista
dinamico in cui la variabile dipendente \`{e} influenzata da due o pi\`{u}
ritardi della stessa variabile esplicativa. Ad esempio il modello precedente
potrebbe essere esteso nel seguente modo%
\begin{equation*}
y_{t}=b_{1}x_{t-1}+b_{2}x_{t-2}+b_{3}x_{t-3}+\varepsilon _{t}.
\end{equation*}

Un caso particolare e molto interessante di modello dinamico \`{e} quello in
cui nell'insieme delle variabili esplicative sono presenti uno o pi\`{u}
ritardi della variabile dipendente $y$. Si parla in questo caso di modelli
dinamici a variabili endogene ritardate. In economia capita sovente di
imbattersi in modelli a variabili endogene ritardate ed \`{e} quindi giusto
analizzare quali sono le problematiche che questa classe di modelli pone dal
punto di vista della stima e dei test.

\section{Esempi economici}

\subsection{La formazioni d'abitudini}

Prendiamo una funzione di consumo. Per Keynes il consumo \`{e} una funzione
del reddito: $C=f(R)$%
\begin{equation}
C_{t}=a+bR_{t}+\varepsilon _{t}  \label{consumo reddito lungo termine}
\end{equation}%
con $\varepsilon _{t}\sim iid(0,\sigma ^{2})$ ed il coefficiente $b$ a
rappresentare la propensione marginale al consumo: $0<b<1$. \`{E} questa una
relazione statitica fra consumo e reddito. Il consumo dipende dal reddito in
modo istantaneo. L'equazione (\ref{consumo reddito lungo termine})
rappresenta piuttosto una relazione d'equilibrio di lungo termine. Questa
relazione statica non \`{e} sufficiente per rappresentare il comportamento
dei consumatori nel corto periodo. Supponiamo infatti, che il reddito
diminuisca. Il consumo tender\`{a} anch'esso a diminuire, ma non
immediatamente a causa della formazione di abitudini: si \`{e} abituati ad
un certo tenore di vita e anche se il reddito cambia le abitudini al consumo
non reagiscono altrettanto velocemente. Le formazioni di abitudini sono
rappresentate dal consumo passato.%
\begin{equation*}
C_{t}=a+cR_{t}+dC_{t-1}+\varepsilon _{t}\ .
\end{equation*}%
$C_{t-1}$ rappresenta la variabile endogena ritardata. Essa \`{e} aleatoria
e quindi siamo in presenza di un modello di regressione stocastica!\ Come
interpretiamo i parametri? Per rispondere a questa domanda poniamoci in $t=0$
e supponiamo che da lungo tempo $(t<0)$ il sistema si trovi in equilibrio
con reddito e consumo costanti. Ci\`{o} significa che prima di $t=0,$ $%
\Delta R_{t}=0$ e $\Delta C_{t}=0$. Scriviamo il nostro modello in termini
di variazioni, trascurando momentaneamente il termine d'errore:%
\begin{equation*}
\Delta C_{t}=c\Delta R_{t}+d\Delta C_{t-1}\ .
\end{equation*}%
Supponiamo che al tempo $t=0$ il reddito aumenti di una unit\`{a} (una volta
per sempre), cio\`{e}%
\begin{eqnarray*}
\Delta R_{t} &=&0\ \text{per}\ t<0 \\
\Delta R_{t} &=&1\ \text{per}\ t=0 \\
\Delta R_{t} &=&0\ \text{per}\ t>0
\end{eqnarray*}%
Studiamo la variazione indotta sul consumo al tempo $t=0$ e nei momenti
successivi.%
\begin{equation*}
t=0:\ \ \ \Delta C_{0}=c\underset{1}{\underbrace{\Delta R_{0}}}+d\underset{0}%
{\underbrace{\Delta C_{-1}}}=c
\end{equation*}%
$c$ \`{e} la propensione marginale al consumo di corto periodo, l'effetto
immediato. 
\begin{eqnarray*}
t &=&1:\ \ \ \Delta C_{1}=c\underset{0}{\underbrace{\Delta R_{1}}}+d\underset%
{c}{\underbrace{\Delta C_{0}}}=dc \\
t &=&2:\ \ \ \Delta C_{2}=c\underset{0}{\underbrace{\Delta R_{2}}}+d\underset%
{dc}{\underbrace{\Delta C_{1}}}=d^{2}c \\
&&\vdots \\
t &=&n:\ \ \ \Delta C_{n}=c\underset{0}{\underbrace{\Delta R_{n}}}+d\underset%
{d^{n-1}c}{\underbrace{\Delta C_{n-1}}}=d^{n}c
\end{eqnarray*}%
L'effetto cumulato dopo $n$ periodi:%
\begin{eqnarray*}
\Delta C_{TOT(n)} &=&\Delta C_{0}+\Delta C_{1}+\ldots +\Delta C_{n} \\
&=&c\left( 1+d+d^{2}+\ldots +d^{n}\right)
\end{eqnarray*}%
Quando $n\rightarrow \infty $, l'effetto totale $\Delta C_{TOT}$ \`{e}
finito quando $\mid d\mid <1$. In tal caso si ha: $\Delta C_{TOT}=\frac{c}{%
1-d}=b$ che chiameremo la propensione marginale di lungo periodo. Possiamo
dedurla anche direttamente dall'equazione di partenza. Infatti supponiamo
che tutte le variabili del modello siano in valore uguale al loro livello
d'equilibrio di lungo termine:%
\begin{eqnarray*}
R_{t} &=&R^{\ast }\ \forall t \\
C_{t} &=&C^{\ast }\ \forall t
\end{eqnarray*}%
da cui%
\begin{equation*}
C^{\ast }=a+cR^{\ast }+dC^{\ast }\Rightarrow (1-d)C^{\ast }=a+cR^{\ast }
\end{equation*}%
e per finire%
\begin{eqnarray*}
C^{\ast } &=&\frac{a}{1-d}+\frac{c}{1-d}R^{\ast } \\
&=&a^{\ast }+c^{\ast }R^{\ast }
\end{eqnarray*}%
Nell'equazione $C_{t}=a+cR_{t}+dC_{t-1}+\varepsilon _{t}$

\begin{itemize}
\item $C_{t}$ \`{e} la variabile spiegata, detta anche endogena,

\item $R_{t}$ \`{e} una variabile esplicativa esogena,

\item $C_{t-1}$ \`{e} pure una variabile esplicativa: si tratta di una
variabile endogena ritardata.
\end{itemize}

\noindent In contrasto, la formulazione 
\begin{equation*}
C_{t}=a+b_{1}R_{t}+b_{2}R_{t-1}+\varepsilon _{t}
\end{equation*}%
\`{e} pure un modello dinamico ma senza variabile endogena ritardata. La
variabile ritardata \`{e} la variabile esogena%
\begin{equation*}
\Delta C_{t}=b_{1}\Delta R_{t}+b_{2}\Delta R_{t-1}
\end{equation*}%
\begin{eqnarray*}
\Delta R_{0}=1\Rightarrow \Delta C_{0} &=&b_{1}\underset{1}{\underbrace{%
\Delta R_{0}}}+b_{2}\underset{0}{\underbrace{\Delta R_{-1}}}=b_{1} \\
\Delta C_{1} &=&b_{1}\underset{0}{\underbrace{\Delta R_{1}}}+b_{2}\underset{1%
}{\underbrace{\Delta R_{0}}}=b_{2} \\
\Delta C_{2} &=&b_{1}\underset{0}{\underbrace{\Delta R_{2}}}+b_{2}\underset{0%
}{\underbrace{\Delta R_{1}}}=0
\end{eqnarray*}%
da cui deduciamo%
\begin{equation*}
\Delta C_{TOT}=\Delta C_{0}+\Delta C_{1}=b_{1}+b_{2}
\end{equation*}%
\pagebreak

\subsection{Il modello d'aggiustamento parziale}

In una data industria il livello desiderato degli stock, notato $S_{t}^{\ast
}$, \`{e} dato da 
\begin{equation}
S_{t}^{\ast }=a+bV_{t}+\varepsilon _{t}  \label{livello stock}
\end{equation}%
Gli imprenditori hanno un comportamento d'aggiustamento parziale: colmano
soltanto una parte della differenza tra il livello attuale e quello
desiderato:%
\begin{equation}
\underset{%
\begin{array}{c}
\text{{\tiny variazione}} \\ 
\text{{\tiny (o investimento)}} \\ 
\text{{\tiny negli stock}}%
\end{array}%
}{\underbrace{S_{t}-S_{t-1}}}=\lambda \underset{%
\begin{array}{c}
\text{{\tiny distanza che ci}} \\ 
\text{{\tiny separa dall'obiettivo}}%
\end{array}%
}{\underbrace{\left( S_{t}^{\ast }-S_{t-1}\right) }}\text{ \ dove }0<\lambda
<1  \label{aggiustamento livello stock}
\end{equation}%
$S^{\ast }$ non \`{e} osservabile. Le sole variabili osservabili sono $S_{t}$
e $V_{t}$ con i loro valori passati. In questo tipo di problemi bisogna fare
3 cose:

\begin{enumerate}
\item Derivare un'equazione stimabile, cio\`{e} un'equazione nella quale
intervengono esclusivamente variabili effettivamente osservate.

\item Studiare l'identificazione dei parametri del modello di partenza
(detto modello strutturale) a partire dai coefficienti stimati
nell'equazione stimabile.

\item Studiare le propriet\`{a} statistiche dell'equazione stimabile e
proporre un metodo di stima appropriato.
\end{enumerate}

\noindent Consideriamo questi tre punti in dettaglio.

\begin{enumerate}
\item L'equazione stimata

Dalla (\ref{aggiustamento livello stock}):%
\begin{eqnarray*}
\lambda S_{t}^{\ast } &=&S_{t}-(1-\lambda )S_{t-1} \\
\lambda a+\lambda bV_{t}+\lambda \varepsilon _{t} &=&S_{t}-(1-\lambda
)S_{t-1} \\
S_{t} &=&\lambda a+\lambda bV_{t}+(1-\lambda )S_{t-1}+\lambda \varepsilon
_{t} \\
S_{t} &=&\beta _{1}+\beta _{2}V_{t}+\beta _{3}S_{t-1}+u_{t}
\end{eqnarray*}

\item L'identificazione dei parametri strutturali

Conoscendo $\beta _{1},\ \beta _{2}$ e $\beta _{3}$ (dopo averli stimati) 
\`{e} possibile ottenere in maniera \emph{unica} i parametri strutturali:\ $%
a $, $b$ e $\lambda $.%
\begin{equation*}
\begin{array}{ccccccc}
\beta _{3} & = & (1-\lambda ) & \Rightarrow & \lambda & = & 1-\beta _{3} \\ 
\beta _{2} & = & \lambda b & \Rightarrow & b & = & \frac{\beta _{2}}{1-\beta
_{3}} \\ 
\beta _{1} & = & \lambda a & \Rightarrow & a & = & \frac{\beta _{1}}{1-\beta
_{3}}%
\end{array}%
\end{equation*}

\item Le propriet\`{a} statistiche dell'equazione stimabile

\begin{enumerate}
\item L'equazione stimabile \`{e} un'equazione con \emph{variabile endogena
ritardata}.

\item Propriet\`{a} degli errori:\ $u_{t}=\lambda \varepsilon _{t}$.

Se $\varepsilon _{t}\sim iid(0,\sigma ^{2})\Rightarrow u_{t}\sim
iid(0,\lambda ^{2}\sigma ^{2})$.
\end{enumerate}
\end{enumerate}

\noindent Si tratta quindi di un modello a variabile endogena ritardata ed
errori indipendenti: i $m.q.o.$ sono convergenti (e asintoticamente
efficienti).

\subsection{Modello ad aspettative adattive (adaptive expectation)}

In una certa industria il volume delle scorte, $S_{t}$, \`{e} legato al
volume previsto delle vendite, $V_{t+1}^{\ast }$, secondo la relazione%
\begin{eqnarray}
S_{t} &=&a+bV_{t+1}^{\ast }+\varepsilon _{t}
\label{scorte aspettative adattive} \\
\varepsilon _{t} &\sim &iid(0,\sigma ^{2})  \notag
\end{eqnarray}%
Le imprese aggiornano le loro previsioni in funzione degli errori di
previsione passati%
\begin{equation}
\underset{\text{aggiornamento previsione}}{\underbrace{V_{t+1}^{\ast
}-V_{t}^{\ast }}}=\gamma \underset{\text{errore passato}}{\underbrace{%
(V_{t}-V_{t}^{\ast })}}\ \ \ \ \ \mid \gamma \mid <1
\label{aggiornamento previsione}
\end{equation}

\begin{enumerate}
\item L'equazione stimata

Dalla (\ref{aggiornamento previsione})%
\begin{equation*}
V_{t+1}^{\ast }-(1-\gamma )V_{t}^{\ast }=\gamma V_{t}
\end{equation*}

Dalla (\ref{scorte aspettative adattive})%
\begin{equation*}
\left\{ 
\begin{array}{cc}
S_{t}=a+bV_{t+1}^{\ast }+\varepsilon _{t} & \ \ \ (a) \\ 
S_{t-1}=a+bV_{t}^{\ast }+\varepsilon _{t-1} & \ \ \ (b)%
\end{array}%
\right.
\end{equation*}

$(a)-(1-\gamma )(b)$:%
\begin{eqnarray*}
S_{t}-(1-\gamma )S_{t-1} &=&\gamma a+b(V_{t+1}^{\ast }-(1-\gamma
)V_{t}^{\ast })+\varepsilon _{t}-(1-\gamma )\ \varepsilon _{t-1} \\
&=&\gamma a+b\gamma V_{t}+\varepsilon _{t}-(1-\gamma )\ \varepsilon _{t-1} \\
S_{t} &=&\gamma a+b\gamma V_{t}+(1-\gamma )S_{t-1}+\varepsilon
_{t}-(1-\gamma )\ \varepsilon _{t-1} \\
S_{t} &=&\beta _{1}+\beta _{2}S_{t-1}+\beta _{3}V_{t}+u_{t}
\end{eqnarray*}%
dove $u_{t}=\varepsilon _{t}-(1-\gamma )\ \varepsilon _{t-1}$.

\item L'identificazione dei parametri strutturali

\begin{equation*}
\begin{array}{ccccccc}
\beta _{3} & = & 1-\gamma & \Rightarrow & \gamma & = & 1-\beta _{3} \\ 
\beta _{2} & = & b\gamma & \Rightarrow & b & = & \frac{\beta _{2}}{1-\beta
_{3}} \\ 
\beta _{1} & = & \gamma a & \Rightarrow & a & = & \frac{\beta _{1}}{1-\beta
_{3}}%
\end{array}%
\end{equation*}

\item Le propriet\`{a} statistiche dell'equazione stimabile

L'equazione stimabile \`{e} un modello a variabile endogena ritardata con
errori che non sono indipendenti. Infatti $u_{t}=\varepsilon _{t}-(1-\gamma
)\ \varepsilon _{t-1}$: poich\'{e} $\varepsilon _{t}\sim iid(0,\sigma ^{2})$
otteniamo%
\begin{eqnarray*}
E(u_{t}) &=& \\
V(u_{t}) &=& \\
Cov(u_{t},u_{t-1}) &=& \\
&=& \\
&& \\
Cov(u_{t},u_{t-s}) &=&0\text{ \ \ \ \ \ \ }\forall s>1
\end{eqnarray*}
\end{enumerate}

\noindent Mostreremo che in un modello a variabile endogena ritardata ed
errori autocorrelati i $m.q.o.$ non sono convergenti. Sar\`{a} quindi
necessario definire una nuova classe di stimatori.

\section{La stima dei modelli a variabile endogena ritardata}

La formulazione generale di questa classe di modelli \`{e} la seguente:%
\begin{equation*}
y_{t}=\underset{p\text{ ritardi della variabile endogena}}{\underbrace{%
\alpha _{1}y_{t-1}+\alpha _{2}y_{t-2}+\ldots +\alpha _{p}y_{t-p}}}+\underset{%
K\text{ variabili esogene}}{\underbrace{\beta _{1}x_{1t}+\ldots +\beta
_{K}x_{Kt}}}+\varepsilon _{t}
\end{equation*}%
$\varepsilon _{t}$ \`{e} il solito errore non osservabile, $\varepsilon
_{t}\sim (0,\sigma ^{2})$.

\noindent I problemi di stima di un tale modello si possono esaminare
studiando il modello semplice 
\begin{equation*}
y_{t}=\alpha _{1}y_{t-1}+\varepsilon _{t}
\end{equation*}%
che come abbiamo visto nella prima parte del corso \`{e} detto modello $%
AR(1) $ (modello autoregressivo di ordine 1).\vspace{0.5cm}

\noindent \emph{Prima costatazione}: la variabile endogena ritardata
(variabile stocastica) \`{e} necessariamente correlata con gli \emph{errori
passati}. Infatti%
\begin{equation}
y_{t}=\alpha _{1}y_{t-1}+\varepsilon _{t}  \label{ar1}
\end{equation}%
per il periodo precedente%
\begin{eqnarray*}
y_{t-1} &=&\alpha _{1}y_{t-2}+\varepsilon _{t-1} \\
y_{t-2} &=&\alpha _{1}y_{t-3}+\varepsilon _{t-2} \\
&&\vdots
\end{eqnarray*}%
Quindi la variabile $y_{t-1}$ della (\ref{ar1}) dipende da $y_{t-2}$ e da $%
\varepsilon _{t-1}$. $y_{t-2}$ a sua volta dipende da $\varepsilon _{t-2}$ e
cos\`{\i} via: $y_{t-1}$ \`{e} dunque correlato con $\varepsilon
_{t-1},\varepsilon _{t-2},\ldots $.\vspace{0.5cm}

\noindent \emph{Seconda costatazione}: siccome $y_{t-1}$ dipende da $%
\varepsilon _{t-1},\varepsilon _{t-2},\ldots $ e quindi non direttamente da $%
\varepsilon _{t}$, $y_{t-1}$ non \`{e} correlata con $\varepsilon _{t}$
(nella (\ref{ar1})) quando gli errori $\varepsilon _{t}$ sono indipendenti
(non correlati).

\noindent Invece, quando gli errori $\varepsilon _{t}$ sono correlati, la
situazone \`{e} la seguente%
\begin{equation*}
\left. 
\begin{array}{ccc}
\varepsilon _{t} & \text{dipende da} & \varepsilon _{t-1},\varepsilon
_{t-2},\ldots \\ 
y_{t-1} & \text{dipende da} & \varepsilon _{t-1},\varepsilon _{t-2},\ldots%
\end{array}%
\right\} \Rightarrow y_{t-1}\text{ e }\varepsilon _{t}\text{ nella (\ref{ar1}%
) sono correlati!}
\end{equation*}%
\vspace{0.5cm}

\noindent In conclusione: il modello a variabile endogena ritardata \`{e} un
modello stocastico non correlato con gli errori quando quest'ultimi sono
indipendenti (in tal caso i $m.q.o.$ sono ok) ed invece si tratta di un
modello di regressione stocastica dipendente quando quest'ultimi sono
correlati (i $m.q.o.$ non sono convergenti).

\subsection{\label{ver ed errori indipendenti}Quando gli errori sono
indipendenti}

\begin{equation}
y_{t}=\alpha y_{t-1}+\varepsilon _{t}\ \ \ \ \ \varepsilon _{t}\sim
iid(0,\sigma ^{2}).  \label{ar1 bis}
\end{equation}%
$y_{t-1}$ e $\varepsilon _{t}$ sono non correlati: $Cov(y_{t-1},\varepsilon
_{t})=E(y_{t-1}\varepsilon _{t})=0$\vspace{0.5cm}

\noindent Lo stimatore dei $m.q.o.$ applicato al modello (\ref{ar1 bis}) 
\`{e} semplicemente%
\begin{equation*}
\widehat{\alpha }=\frac{\sum y_{t}y_{t-1}}{\sum y_{t-1}^{2}}\text{.}
\end{equation*}%
$\widehat{\alpha }$ non \`{e} uno stimatore lineare. Non \`{e} possibile
calcolare il suo valore atteso e la sua varianza. Le sole propriet\`{a}
verificabili sono le propriet\`{a} asintotiche.

\begin{enumerate}
\item Convergenza

\begin{equation*}
\widehat{\alpha }=\frac{\sum y_{t}y_{t-1}}{\sum y_{t-1}^{2}}=\alpha +\frac{%
\frac{1}{n}\sum \varepsilon _{t}y_{t-1}}{\frac{1}{n}\sum y_{t-1}^{2}}
\end{equation*}%
Il $p\lim $ del numeratore $\frac{1}{n}\sum \varepsilon _{t}y_{t-1}$
corrisponde alla covarianza empirica nel campione tra $\varepsilon _{t}$ e $%
y_{t-1}$: esso converge alla covarianza della popolazione $cov(\varepsilon
_{t},y_{t-1})$.\vspace{0.5cm}

Il denominatore $\frac{1}{n}\sum y_{t-1}^{2}$ corrisponde alla varianza
empirica del campione. Il suo limite in probabilit\`{a} \`{e} $E(y_{t}^{2})=%
\frac{\sigma ^{2}}{1-\alpha ^{2}}$ che corrisponde anche alla varianza di $%
y_{t}$ (perch\'{e}?).

\item Distribuzione asintotica

\begin{equation*}
\sqrt{n}\left( \widehat{\alpha }-\alpha \right) =\underset{p\lim \ldots =%
\frac{\sigma ^{2}}{1-\alpha ^{2}}}{\underbrace{\left( \frac{1}{n}\sum
y_{t-1}^{2}\right) ^{-1}}}\underset{\overset{D}{\longrightarrow }N\left(
0,\sigma ^{4}/(1-\alpha ^{2})\right) }{\underbrace{\frac{1}{\sqrt{n}}\sum
\varepsilon _{t}y_{t-1}}}\overset{D}{\longrightarrow }N\left( 0,1-\alpha
^{2}\right)
\end{equation*}%
La varianza asintotica \`{e} dunque%
\begin{equation*}
VarAs\left( \widehat{\alpha }\right) =\frac{1-\alpha ^{2}}{n}
\end{equation*}%
che potrebbe essere stimato con 
\begin{equation*}
\widehat{VarAs}\left( \widehat{\alpha }\right) =\frac{1-\widehat{\alpha }^{2}%
}{n}
\end{equation*}

Se a torto utilizzassimo la formula per la varianza di $\widehat{\alpha }$
data dai $m.q.o.$:

\begin{eqnarray*}
Var(\widehat{\alpha }) &=&\frac{\widehat{\sigma }^{2}}{\tsum y_{t-1}^{2}},\
\ \widehat{\sigma }^{2}=\frac{1}{n-1}\tsum \widehat{\varepsilon }_{t}^{2}=%
\frac{1}{n-1}\tsum \left( y_{t}-\widehat{\alpha }y_{t-1}\right) ^{2} \\
Var(\widehat{\alpha }) &\simeq &\frac{\frac{1}{n}\tsum \left( y_{t}^{2}-2%
\widehat{\alpha }y_{t}y_{t-1}+\widehat{\alpha }^{2}y_{t-1}^{2}\right) }{%
\tsum y_{t-1}^{2}} \\
&=&\frac{1}{n}\frac{\tsum y_{t}^{2}-\widehat{\alpha }\tsum y_{t}y_{t-1}}{%
\tsum y_{t-1}^{2}}\simeq \frac{1}{n}\left( 1-\widehat{\alpha }^{2}\right)
\end{eqnarray*}
\end{enumerate}

In pratica dunque, quando gli errori sono indipendenti, si applicano tutte
le formule dei $m.q.o.$, anche per le varianze. Il metodo dei $m.q.$ \`{e}
la migliore cosa che si possa fare. Sotto l'ipotesi di normalit\`{a} lo
stimatore dei minimi quadrati \`{e} essenzialmente uguale allo stimatore $ML$
(Massima Verosimiglianza) e quindi asintoticamente efficiente.

\subsection{Quando gli errori sono autocorrelati}

Studiamo il comportamento dello stimatore dei $m.q.o.$ nel caso in cui gli
errori siano autocorrelati, ed in particolare nel caso in cui l'errore $%
\varepsilon _{t}$ segua un processo di tipo $AR(1)$.

\begin{eqnarray}
y_{t} &=&\alpha y_{t-1}+\varepsilon _{t}  \label{ar1_errore1} \\
\varepsilon _{t} &=&\rho \varepsilon _{t-1}+u_{t}  \label{ar1_errore2}
\end{eqnarray}%
dove $\mid \rho \mid <1,\ u_{t}\sim iid(0,\sigma ^{2})$.

\noindent Combinando la (\ref{ar1_errore1}) e (\ref{ar1_errore2}) 
\begin{equation*}
\left\{ 
\begin{array}{rclr}
y_{t} & = & \alpha y_{t-1}+\varepsilon _{t} & (a) \\ 
y_{t-1} & = & \alpha y_{t-2}+\varepsilon _{t-1} & (b)%
\end{array}%
\right.
\end{equation*}%
$(a)-\rho (b)$:%
\begin{eqnarray}
y_{t}-\rho y_{t-1} &=&\alpha y_{t-1}-\alpha \rho y_{t-2}+\varepsilon
_{t}-\rho \varepsilon _{t-1}  \notag \\
y_{t} &=&\underset{\beta _{1}}{\underbrace{(\alpha +\rho )}}y_{t-1}\underset{%
\beta _{2}}{+\underbrace{-\alpha \rho }}y_{t-2}+u_{t}
\label{ar2_con_errore_indipendenti}
\end{eqnarray}%
Questo modello (\ref{ar2_con_errore_indipendenti}) \`{e} un modello con
variabile endogena ritardata ed errori indipendenti. I $m.q.o.$
sull'equazione (\ref{ar2_con_errore_indipendenti}) produrrebbero delle stime
convegenti di $\beta _{1}$ e $\beta _{2}$. Sfortunatamente conoscendo i
coefficienti $\beta _{1}$ e $\beta _{2}$ non si ottiene una soluzione unica
di $\alpha $ e $\rho $: $\alpha $ e $\rho $ non sono identificabili! Ad
esempio%
\begin{equation*}
\left. 
\begin{array}{c}
\beta _{1}=\alpha +\rho =\frac{3}{4} \\ 
-\beta _{2}=\alpha \rho =\frac{1}{8}%
\end{array}%
\right\} \Rightarrow \text{due soluzioni }%
\begin{array}{c}
\alpha =0.5,\text{ }\rho =0.25 \\ 
\alpha =0.25,\text{ }\rho =0.5%
\end{array}%
\end{equation*}%
Se lo scopo \`{e} la previsione, il modello (\ref%
{ar2_con_errore_indipendenti}) \`{e} la miglior cosa che possiate fare. Per
la stima dei parametri $\alpha $ e $\rho $ il modello (\ref%
{ar2_con_errore_indipendenti}) non \`{e} di alcun aiuto.

\subsubsection{La stima di $\protect\alpha $ con i $m.q.o.$ nella (\protect
\ref{ar1_errore1})}

Partiamo dallo stimatore dei $m.q.o.$ di $\alpha $ 
\begin{equation*}
\widehat{\alpha }=\frac{\sum y_{t}y_{t-1}}{\sum y_{t-1}^{2}}
\end{equation*}%
e sostituiamo $y_{t}$ con l'espressione data dalla (\ref%
{ar2_con_errore_indipendenti}), ottenendo in tal modo 
\begin{eqnarray*}
\widehat{\alpha } &=&\frac{\sum \left( (\alpha +\rho )y_{t-1}-\alpha \rho
y_{t-2}+u_{t}\right) y_{t-1}}{\sum y_{t-1}^{2}} \\
&=&\alpha +\rho -\alpha \rho \underset{\approx \widehat{\alpha }}{%
\underbrace{\frac{\sum y_{t-2}y_{t-1}}{\sum y_{t-1}^{2}}}}+\frac{\frac{1}{n}%
\sum u_{t}y_{t-1}}{\frac{1}{n}\sum y_{t-1}^{2}}
\end{eqnarray*}%
Poich\'{e} $E(y_{t})=0,$ $V(y_{t})=E(y_{t}^{2})$. Inoltre, sappiamo che%
\begin{eqnarray*}
p\lim \frac{1}{n}\sum u_{t}y_{t-1} &=&0 \\
p\lim \frac{1}{n}\sum y_{t-1}^{2} &=&V(y_{t})
\end{eqnarray*}%
Quindi 
\begin{eqnarray*}
p\lim \ \widehat{\alpha } &=&\alpha +\rho -\alpha \rho \ p\lim \widehat{%
\alpha }+\frac{p\lim \frac{1}{n}\sum u_{t}y_{t-1}}{p\lim \frac{1}{n}\sum
y_{t-1}^{2}} \\
&=&\alpha +\rho -\alpha \rho \ p\lim \widehat{\alpha }
\end{eqnarray*}%
da cui concludiamo che%
\begin{equation*}
p\lim \ \widehat{\alpha }=\frac{\alpha +\rho }{1+\alpha \rho }=\alpha \ \ +%
\underset{\text{distorsione asintotica}}{\underbrace{\frac{\rho (1-\alpha
^{2})}{1+\alpha \rho }}}\neq \alpha ,\text{ \ \ quando }\rho \neq 0.
\end{equation*}%
Osserviamo quindi che

\begin{itemize}
\item nel caso in cui $\rho \neq 0$, lo stimatore dei minimi quadrati sul
modello (\ref{ar2_con_errore_indipendenti}) \`{e} distorto: la distorsione
asintotica non scompare all'aumentare dell'informazione all'infinito.

\item $\widehat{\alpha }$ \`{e} convergente solo quando $\rho =0$, cio\`{e}
quando gli errori $\varepsilon _{t}$ sono indipendenti.

\item Dato che $\rho $ e $\alpha $ sono positivi e compresi tra 0 e 1, la
distorsione \`{e} positiva.
\end{itemize}

\noindent \emph{Conclusione}: Nel caso di variabili endogene ritardate ed
errori autocorrelati i $m.q.o.$ non sono convergenti!

\subsubsection{\label{come stimare alpha}Come stimare $\protect\alpha $}

Notiamo che se il termine $\rho $ fosse conosciuto la (\ref{ar1_errore1})
potrebbe scriversi 
\begin{eqnarray}
y_{t}-\rho y_{t-1} &=&\alpha (y_{t-1}-\rho y_{t-2})+u_{t}
\label{ar1 modello trasformato} \\
\widetilde{y}_{t} &=&\alpha \widetilde{y}_{t-1}+u_{t}
\label{ar1 modello trasformato1}
\end{eqnarray}%
Il nuovo modello cos\`{\i} trasformato sarebbe un modello a variabile
endogena ritardata con errori indipendenti e, come appena dimostrato nel
paragrafo (\ref{ver ed errori indipendenti}), $\alpha $ sarebbe stimabile
tramite i $m.q.o.$. Come possiamo stimare $\rho $? Ci troviamo in una
situazione simile a quella incontrata nel paragrafo (\ref{autocorr i metodi
di stima}) quando abbiamo utilizzato i $m.q.g.$ a due tappe stimando
dapprima il parametro $\rho $ tramite la regressione ausiliaria (\ref%
{regressione_ausiliaria}). Si potrebbe quindi pensare di fare la stessa cosa
e cio\`{e}:

\begin{enumerate}
\item[{[1]}] Stimare con i $m.q.o.$ l'equazione (\ref{ar1_errore1})%
\begin{equation*}
y_{t}=\alpha y_{t-1}+\varepsilon _{t}.
\end{equation*}

\item[{[2]}] Prendere i residui stimati $\widehat{\varepsilon }$
dell'equazione (\ref{ar1_errore1}) e stimare $\rho $ applicando i $m.q.o.$
all'equazione 
\begin{equation*}
\widehat{\varepsilon }_{t}=\rho \widehat{\varepsilon }_{t-1}+u_{t}.
\end{equation*}

\item[{[3]}] Stimare $\alpha $ applicando i $m.q.o.$ al modello trasformato (%
\ref{ar1 modello trasformato1}).
\end{enumerate}

\noindent Tuttavia abbiamo dimostrato che i $m.q.o.$ sul modello (\ref%
{ar1_errore1}) (passo [1]) non sono convergenti. Questo implica una
distorsione nella stima degli $\widehat{\varepsilon }_{t}$ e a sua volta
anche nella stima di $\rho $ nel passo [2]:%
\begin{equation*}
\widehat{\rho }=\frac{\sum \widehat{\varepsilon }_{t}\widehat{\varepsilon }%
_{t-1}}{\sum \widehat{\varepsilon }_{t}^{2}}.
\end{equation*}%
Come vedremo in seguito la soluzione sar\`{a} quella di modificare il passo
[1], rimpiazzando i $m.q.o.$ con uno stimatore convergente quale ad esempio
lo stimatore delle variabili strumentali.

\subsection{Il test di autocorrelazione}

Consideriamo il modello generale%
\begin{eqnarray}
y_{t} &=&\underset{K\text{ variabili esogene}}{x_{t}^{\prime }\beta }+%
\underset{p\text{ ritardi nella variabile endogena}}{\alpha
_{1}y_{t-1}+\ldots +\alpha _{p}y_{t-p}}+\varepsilon _{t}
\label{k variabili esogene e p} \\
\varepsilon _{t} &=&\rho \varepsilon _{t-1}+u_{t}\ \ \ u_{t}\sim
iid(0,\sigma ^{2})  \notag
\end{eqnarray}%
\`{E} dunque un modello a variabile endogena ritardata. Desideriamo testare
l'ipotesi nulla di assenza di autocorrelazione negli errori $\varepsilon
_{t} $. Formalmente $H_{0}:\rho =0$.

\begin{remark}
In un modello a variabile endogena ritardata il test $DW$ non \`{e}
appropriato poich\'{e} necessariamente distorto in favore dell'indipendenza.
Non va assolutamente applicato.
\end{remark}

Al suo posto si utilizza un test proposto da Durbin e chiamato test "$h$".
La procedura del test \`{e} la seguente.

\begin{itemize}
\item Si stima il modello sotto l'ipotesi nulla $\rho =0$, vale a dire con i 
$m.q.o.,$ ottenendo $\widehat{\beta },\widehat{\alpha }_{1},...,\widehat{%
\alpha }_{p}$, $\widehat{V}(\widehat{\alpha }_{1})$ e costruendo 
\begin{equation*}
\widehat{\varepsilon }_{t}=y-x_{t}^{\prime }\widehat{\beta }-\widehat{\alpha 
}_{1}y_{t-1}-\ldots -\widehat{\alpha }_{p}y_{t-p}.
\end{equation*}

\item Si calcola la quantit\`{a}%
\begin{equation*}
h=\widehat{\rho }\sqrt{\frac{n}{1-n\widehat{V}(\widehat{\alpha }_{1})}}
\end{equation*}%
dove $\widehat{\rho }$ \`{e} uno stimatore convergente di $\rho $ sotto $%
H_{0}$ quale ad esempio

\begin{itemize}
\item 
\begin{equation*}
\widehat{\rho }=\frac{\sum \widehat{\varepsilon }_{t}\widehat{\varepsilon }%
_{t-1}}{\sum \widehat{\varepsilon }_{t-1}^{2}}
\end{equation*}%
ottenuto dalla regressione $\widehat{\varepsilon }_{t}=\rho \widehat{%
\varepsilon }_{t-1}+errore,$

\item oppure $\widehat{\rho }=1-\frac{1}{2}d$, dove $d$ \`{e} la quantit\`{a}
del test $DW$.
\end{itemize}
\end{itemize}

Si dimostra che asintoticamente $h\sim N(0,1)$ e quindi, al $5\%$ di
significativit\`{a}:%
\begin{eqnarray*}
|h| &>&1.96\Rightarrow \text{ rifiuto }H_{0} \\
|h| &<&1.96\Rightarrow \text{ accetto }H_{0}
\end{eqnarray*}

\subsection{Stima convergente ed asintoticamente efficiente}

Riprendendo il modello (\ref{k variabili esogene e p}) consideriamo un
modello con un solo ritardo e con $K$ variabili esplicative esogene%
\begin{eqnarray}
y_{t} &=&x_{t}^{\prime }\beta +\alpha y_{t-1}+\varepsilon _{t}
\label{modello stima convergente1} \\
\varepsilon _{t} &=&\rho \varepsilon _{t-1}+u_{t}\ \ \ u_{t}\sim
iid(0,\sigma ^{2})  \label{modello stima convergente2}
\end{eqnarray}%
Discutiamo una stima convergente dei parametri dell'equazione (\ref{modello
stima convergente1}) ottenuta con il metodo delle variabili strumentali.

\begin{itemize}
\item Poich\'{e} il numero di variabili esplicative \`{e} $K+1$ dobbiamo
trovare almeno $K+1$ strumenti.

\item Le $K$ variabili $x_{t}^{\prime }$ sono esogene e conseguentemente
indipendenti dagli errori e quindi i loro stessi strumenti.

\item Il compito si riconduce a trovare strumenti per $y_{t-1}$!
\end{itemize}

\noindent Ammettiamo che il modello contenga la costante ed almeno una
variabile esogena. Quali strumenti per $y_{t-1}$ si possono prendere i
valori ritardati in $t-1$ delle variabili esogene (escludendo la costante!),
che indichiamo con $\widetilde{x}_{t}^{\prime }$. Abbiamo in totale $K+(K-1)$
strumenti raccolti nella variabile $z_{t}^{\prime }:=[x_{t}^{\prime }\ 
\widetilde{x}_{t}^{\prime }]$. Con questi strumenti otteniamo una stima
convergente di $\beta ,\alpha $ ed inoltre anche di $\rho $ tramite la
regressione ausiliaria%
\begin{equation}
\widehat{\rho }=\frac{\sum \widehat{\varepsilon }_{t}\widehat{\varepsilon }%
_{t-1}}{\sum \widehat{\varepsilon }_{t-1}^{2}}  \label{stima di rho}
\end{equation}%
dell'equazione%
\begin{equation*}
\widehat{\varepsilon }_{t}=\rho \widehat{\varepsilon }_{t-1}+errore
\end{equation*}%
dove $\widehat{\varepsilon }_{t}=y_{t}-x_{t}^{\prime }\widehat{\beta }_{vs}-%
\widehat{\alpha }_{vs}y_{t-1}$.

\subsubsection{Stima efficiente}

La tappa precedente non fornisce degli stimatori efficienti in quanto
tralascia la struttura di varianza/covarianza degli errori.

\noindent \emph{Idea}: avendo una stima convergente di $\rho $ si pu\`{o}
pensare ad una stima coi $m.q.g.$, vale a dire ad una stima coi $m.q.o.$ del
modello trasformato (confronta \ref{come stimare alpha})%
\begin{equation}
y_{t}-\rho y_{t-1}=(x_{t}^{\prime }-\rho x_{t-1}^{\prime })\beta +\alpha
(y_{t-1}-\rho y_{t-2})+u_{t}
\label{modello trasformato per efficienza asintotica}
\end{equation}%
rimpiazzando $\rho $ con $\widehat{\rho }$ dalla (\ref{stima di rho}).

\noindent Nel suo insieme questa procedura \`{e} simile ai $m.q.g.$
realizzabili (o a due tappe) descritta in precedenza ma con un'importante
differenza: anzich\'{e} utilizzare i $m.q.o.$ nel primo passo abbiamo
utilizzato il metodo delle variabili strumentali.

\begin{remark}
Quando nel modello ci sono delle variabili endogene ritardate i $m.q.g.$
realizzabili \emph{non assicurano} l'efficienza asintotica. L'efficienza
asintotica \`{e} ottenuta stimando sempre con i $m.q.o.$ il modello
trasformato (\ref{modello trasformato per efficienza asintotica}) ma con
l'aggiunta di una variabile esplicativa: $\widehat{\varepsilon }_{t-1}$. In
altre parole il modello trasformato da utilizzare \`{e}%
\begin{equation}
y_{t}-\rho y_{t-1}=(x_{t}^{\prime }-\rho x_{t-1}^{\prime })\beta +\alpha
(y_{t-1}-\rho y_{t-2})+d\widehat{\varepsilon }_{t-1}+u_{t}.
\label{modello trasformato per efficienza asintotica1}
\end{equation}%
Il passaggio da (\ref{modello trasformato per efficienza asintotica}) a (\ref%
{modello trasformato per efficienza asintotica1}) si interpreta nel modo
seguente. Definiamo 
\begin{equation*}
\rho =\widehat{\rho }+\underset{d}{\underbrace{(\rho -\widehat{\rho })}}=%
\widehat{\rho }+d.
\end{equation*}%
e sostituiamo nella (\ref{modello trasformato per efficienza asintotica})%
\begin{eqnarray*}
y_{t}-\left( \widehat{\rho }+d\right) y_{t-1} &=&\left[ x_{t}^{\prime
}-\left( \widehat{\rho }+d\right) x_{t-1}^{\prime }\right] \beta +\left[
y_{t-1}-\left( \widehat{\rho }+d\right) y_{t-2}\right] \alpha +u_{t} \\
y_{t}-\widehat{\rho }y_{t-1} &=&(x_{t}^{\prime }-\widehat{\rho }%
x_{t-1}^{\prime })\beta +\alpha (y_{t-1}-\widehat{\rho }y_{t-2})+d\underset{%
\varepsilon _{t-1}}{\underbrace{\left( y_{t-1}-x_{t-1}^{\prime }\beta
-\alpha y_{t-2}\right) }}+u_{t}.
\end{eqnarray*}%
dove $\varepsilon _{t-1}$ \`{e} sostituito da $\widehat{\varepsilon }%
_{t-1}=y_{t-1}-x_{t-1}^{\prime }\widehat{\beta }_{vs}-\widehat{\alpha }%
_{vs}y_{t-2}$.
\end{remark}

\noindent La stima con i $m.q.o.$ di questo modello garantisce una stima
efficiente di $\beta $, $\alpha $ e $\rho :\widehat{\rho }_{effic.}=\widehat{%
\rho }+\widehat{d}$ e di tutte le varianze covarianze asintotiche\footnote{%
La varianza asintotica di $\widehat{\rho }_{effic.}$ \`{e} data solo dalla
varianza di $\widehat{d}$. $\widehat{\rho }$ \`{e} da trascurare nel calcolo
della varianza di $\widehat{\rho }_{effic.}$.}.

\chapter{\label{Capitolo ritardi distribuiti}Modelli con ritardi distribuiti}

\section{Specificazione e propriet\`{a}. Moltiplicatori d'impatto,
moltiplicatori dinamici, ritardo medio}

\subsection{Specificazione}

\subsubsection{Idea di fondo}

In un modello statico l'influenza esercitata da una variabile \`{e}
istantanea e immediata. Si esaurisce completamente nel momento in cui
appare: $x_{t}\longrightarrow y_{t}$. In economia questa maniera
d'interpretare una relazione non \`{e} molto attrattiva. Molto spesso
infatti l'influenza di una variabile sull'altra non si esaurisce in un solo
periodo, ma si realizza progressivamente nel tempo. In un modello dinamico%
\begin{equation*}
\text{Grafico}
\end{equation*}

\vspace{5cm}\noindent Ci possono essere svariati motivi che giustificano una
simile dinamica, fra cui possiamo citare

\begin{itemize}
\item Abitudini: i comportamenti abitudinari sono un altro motivo per cui 
\`{e} necessario introdurre modelli dinamici in economia. I consumatori
hanno certi standard e si adeguano progressivamente.

\item Costi inerenti ad ogni cambiamento.

\item Ritardi nell'esecuzione di svariati compiti.

\item Rigidit\`{a} istituzionali.

\item Presenza di scorte.

\item Incertezza: spesso prima di reagire si attende per vedere se l'evento
che ci sorprende \`{e} temporaneo o permanente.

\item Tempo necessario per pianificare e realizzare un certo investimento:
ogni decisione in economia richiede un certo tempo di reazione e di
realizzazione. La reazione non \`{e} quasi mai immediata.
\end{itemize}

\subsubsection{Formulazione generale}

\begin{equation*}
y_{t}=a+b_{0}x_{t}+b_{1}x_{t-1}+b_{2}x_{t-2}+\ldots +\varepsilon _{t}
\end{equation*}%
Le due ipotesi della formulazione generale sono:

\begin{enumerate}
\item i $b_{i}$ sono tutti dello stesso segno

\item $\sum_{i=1}^{\infty }b_{i}=b$ finito. \`{E} questa una \emph{%
condizione di stabilit\`{a}}: c'\`{e} un equilibrio di lungo termine.
\end{enumerate}

\paragraph{Idea}

\begin{itemize}
\item Partire da una situazione iniziale di equilibrio (variabili costanti)

\item darsi nel periodo successivo un incremento unitario di $x$ e studiarne
gli effetti su $y$ (presenti e futuri).
\end{itemize}

\noindent All'inizio, in $t=0$, ci troviamo in una situazione di equilibrio.
Tutti gli incrementi di $y$ e di $x$ sono nulli per tutti i periodi
anteriori a $t=0$ incluso:%
\begin{equation*}
\Delta y_{t}=\Delta x_{t}=0\text{ \ \ }t=0,-1,...
\end{equation*}%
Nel periodo $t=1$ consideriamo un aumento unitario di $x$, vale a dire%
\begin{equation*}
\Delta x_{1}=1.
\end{equation*}%
In termine di incrementi, l'equazione del modello si scrive%
\begin{equation*}
\Delta y_{t}=b_{0}\Delta x_{t}+b_{1}\Delta x_{t-1}+b_{2}\Delta x_{t-2}+\ldots
\end{equation*}%
da cui segue che%
\begin{eqnarray*}
\Delta y_{0} &=&0 \\
\Delta y_{1} &=&b_{0}\underset{=1}{\underbrace{\Delta x_{1}}}+b_{1}\underset{%
=0}{\underbrace{\Delta x_{0}}}+b_{2}\underset{=0}{\underbrace{\Delta x_{-1}}}%
+\ldots =b_{0} \\
\Delta y_{2} &=&b_{0}\underset{=0}{\underbrace{\Delta x_{2}}}+b_{1}\underset{%
=1}{\underbrace{\Delta x_{1}}}+b_{2}\underset{=0}{\underbrace{\Delta x_{0}}}%
+...=b_{1} \\
\Delta y_{3} &=&b_{0}\underset{=0}{\underbrace{\Delta x_{3}}}+b_{1}\underset{%
=0}{\underbrace{\Delta x_{2}}}+b_{2}\underset{=1}{\underbrace{\Delta x_{1}}}%
+...=b_{2}
\end{eqnarray*}%
$b_{0}$ \`{e} l'effetto immediato.

\noindent $b_{1}$ \`{e} l'effetto che si manifesta nel primo periodo
successivo.

\noindent $b_{2}$ \`{e} l'effetto che si manifesta nel secondo periodo
successivo.\bigskip

\noindent Dopo $s$ periodi, l'effetto cumulato \`{e} $\Delta y_{1}+\Delta
y_{2}+\Delta y_{3}+...\Delta y_{s}=\sum_{i=0}^{s-1}b_{i}$ e al limite,
l'effetto totale di lungo termine \`{e}%
\begin{equation*}
\Delta _{y_{tot}}=\sum_{i=0}^{\infty }b_{i}=b
\end{equation*}%
$b$ \`{e} quindi l'effetto di lungo termine o \emph{moltiplicatore dinamico}.

\subsubsection{La relazione d'equilibrio di lungo termine}

Per esprimere la relazione d'equilibrio di lungo termine, basta utilizzare
nella relazione i valori d'equilibrio delle variabili:%
\begin{eqnarray*}
y_{t} &=&y^{\ast }\ \ \ \forall t\text{ in equilibrio di lungo termine} \\
x_{t} &=&x^{\ast }\ \ \ \forall t\text{ in equilibrio di lungo termine}
\end{eqnarray*}

da cui ricaviamo%
\begin{equation*}
y^{\ast }=a+b_{0}x^{\ast }+b_{1}x^{\ast }+b_{2}x^{\ast }+\ldots =a+\underset{%
b}{\underbrace{(b_{0}+b_{1}+...)}}x^{\ast }=a+bx^{\ast }
\end{equation*}%
e quindi vale pure%
\begin{equation*}
\frac{dy^{\ast }}{dx^{\ast }}=b.
\end{equation*}%
Definiamo per ogni $i$ la proporzione 
\begin{equation*}
p_{i}:=\frac{b_{i}}{b}
\end{equation*}%
Risulta che:

\begin{enumerate}
\item $p_{i}\geq 0$

\item $\sum_{i=1}^{\infty }p_{i}=1$ \ \ I pesi $p_{i}$ sono le probabilit%
\`{a} di una distribuzione discreta
\end{enumerate}

Questo ci permette di definire il profilo dei ritardi, cio\'{e} il grafico
dei coefficienti "standardizzati".

\begin{equation*}
\text{Grafico}
\end{equation*}%
\vspace{5cm}

\noindent \`{E} possibile definire il ritardo medio%
\begin{equation*}
\overline{R}=\sum_{i=0}^{\infty }ip_{i}=\sum_{i=0}^{\infty }i\frac{b_{i}}{b}=%
\frac{\sum_{i=0}^{\infty }ib_{i}}{\sum_{i=0}^{\infty }b_{i}}
\end{equation*}

\subsubsection{Problemi di stima}

La formulazione generale del modello non \`{e} stimabile in quanto comporta
un'infinit\`{a} di parametri $b_{i}$. \`{E} quindi necessario \emph{%
approssimare} la dinamica fondamentale con un modello che faccia intervenire
un numero fisso e piccolo di parametri da stimare. Due sono le
approssimazioni proposte nella letteratura.

\begin{enumerate}
\item Limitare il numero di ritardi, ammettendo che al di l\`{a} di un certo
ritardo i coefficienti siano nulli. Si parla in questo caso di \emph{%
distribuzione finita dei ritardi}.

\item Ammettere un'infinit\`{a} di ritardi i cui coefficienti sono generati
da un meccanismo che dipende da un piccolo numero fisso di parametri
(distribuzione infinita).
\end{enumerate}

\section{Distribuzione dei ritardi finita}

\subsection{Il caso generale}

Come detto ipotizziamo che a partire da un certo ritardo tutti i
coefficienti siano uguali a zero. Formalmente ci\`{o} equivale a dire che%
\begin{equation*}
b_{i}=0\ \ \ \ \ \ \ \ i>r
\end{equation*}%
dove $r$ \`{e} un intero positivo che denota il ritardo massimo diverso da 0%
\begin{equation*}
y_{t}=a+b_{0}x_{t}+\underset{r\text{ - ritardi}}{\underbrace{%
b_{1}x_{t-1}+b_{2}x_{t-2}+\ldots +b_{r}x_{t-r}}}+\varepsilon _{t}
\end{equation*}%
Ammettiamo che il campione comporti $n$ osservazioni (ad esempio
osservazioni annuali dal 1961 al 2005 per un totale di $2005-1961+1=n=45$
osservazioni. Il numero effettivo di osservazioni, chiamiamolo $n^{\ast }$,
disponibile per stimare il modello con $r$ ritardi \`{e} $n-r$! Il numero di
parametri da stimare \`{e}:%
\begin{equation*}
k=1+1+r=r+2
\end{equation*}%
ed i gradi di libert\`{a} saranno quindi pari a $g.l.=n-2r-2$. Il modello 
\`{e} una regressione di tipo abituale che verr\`{a} stimata con i $m.q.o.$
(oppure con i $m.q.g.$ se necessario). L'implementazione di questo modello
comporta alcuni problemi e specificatamente

\begin{itemize}
\item La scelta arbitraria di $r$.

\item La perdita di gradi di libert\`{a} (due per ogni ritardo
supplementare) $\rightarrow $ ci\`{o} comporta una stima meno precisa.

\item Il rischio di collinearit\`{a} che aumenta all'aumentare del numero di
ritardi: $rg(X)$ non \`{e} pieno $\Rightarrow $ poca precisione nella stima
individuale dei parametri.
\end{itemize}

Per risolvere questi problemi:

\begin{itemize}
\item Per il primo: darsi un criterio di selezione del numero di ritardi.

\item Per gli altri due Almon propone un metodo che prende il suo nome (vedi
punto \ref{ritardi almon}).
\end{itemize}

Un criterio utilizzato frequentemente per la selezione di $r$ \`{e} il
criterio di informazione di Akaike%
\begin{equation*}
AIC:=n^{\ast }\ln \widehat{\sigma }^{2}+2K
\end{equation*}%
Si stima il modello per diversi valori di $r$ e dunque di $K$. Per esempio $%
r=1,2,3,\ldots $ e si sceglie quello con l'$AIC$ minimo.

\subsection{\label{ritardi almon}Distribuzione di Almon (una donna!)}

Come in precedenza si ammette un numero $r$ fisso di ritardi. Un numero
appropriato dipender\`{a} dai dati (annuali, trimestrali, \ldots ) dal
numero di osservazioni e da quanto suggerito dalla teoria economica.
Supponiamo che l'effetto sia positivo $(b>0)$. In certe applicazioni
economiche si pu\`{o} ammettere che i coefficienti $b_{i}$ decrescano
linearmente, seguendo il profilo

\begin{equation*}
\text{Grafico}
\end{equation*}%
\vspace{5cm}

\begin{eqnarray*}
b_{0} &=&h_{0} \\
b_{1} &=&h_{0}+h_{1} \\
b_{2} &=&h_{0}+2h_{1} \\
&&\vdots \\
b_{r} &=&h_{0}+rh_{1}
\end{eqnarray*}%
Conoscendo i due valori di $h_{0}$ e $h_{1}$ possiamo generare tutti i
coefficienti $b_{i}$!\ Introduciamo quindi questi $b_{i}$ nel modello di
partenza:%
\begin{eqnarray*}
y_{t} &=&a+b_{0}x_{t}+b_{1}x_{t-1}+\ldots +b_{r}x_{t-r}+\varepsilon _{t} \\
&=&a+h_{0}x_{t}+\left( h_{0}+h_{1}\right) x_{t-1}+\ldots +\left(
h_{0}+rh_{1}\right) x_{t-r}+\varepsilon _{t} \\
&=&a+h_{0}\underset{Z_{1t}}{\underbrace{\left( x_{t}+x_{t-1}+\ldots
+x_{t-r}\right) }}+h_{1}\underset{Z_{2t}}{\underbrace{\left(
x_{t-1}+2x_{t-2}+\ldots +rx_{t-r}\right) }}+\varepsilon _{t}
\end{eqnarray*}%
La regressione da stimare \`{e} dunque%
\begin{equation*}
y_{t}=a+h_{0}Z_{1t}+h_{1}Z_{2t}+\varepsilon _{t}
\end{equation*}%
Se imponessimo ad Ilaria la restrizione (sicuramente non riuscirebbe a
calcolarla ...)%
\begin{eqnarray*}
b_{r+1} &=&0 \\
&\Updownarrow & \\
h_{0}+(r+1)h_{1} &=&0 \\
h_{1} &=&\frac{-h_{0}}{r+1}
\end{eqnarray*}%
otterremmo%
\begin{eqnarray*}
y_{t} &=&a+h_{0}Z_{1t}-\frac{h_{0}}{r+1}Z_{2t}+\varepsilon _{t} \\
&=&a+h_{0}\underset{w_{t}}{\underbrace{\left( Z_{1t}-\frac{1}{r+1}%
Z_{2t}\right) }}+\varepsilon _{t} \\
&=&a+h_{0}w_{t}+\varepsilon _{t}
\end{eqnarray*}

\noindent Nel caso specifico abbiamo generato i coefficienti $b_{i}$ con un
polinomio di primo grado:\ $b_{i}=h_{0}+h_{1}i$. Questo profilo \`{e} troppo
restrittivo. Un profilo che all'inizio aumenta e poi diminuisce%
\begin{equation*}
\text{Grafico}
\end{equation*}%
\vspace{5cm}

\noindent pu\`{o} essere approssimato con un polinomio di secondo grado:%
\begin{equation*}
b_{i}=h_{0}+h_{1}i+h_{2}i^{2}
\end{equation*}%
mentre un profilo 
\begin{equation*}
\text{Grafico}
\end{equation*}%
\vspace{5cm}

\noindent con un polinomio di terzo grado 
\begin{equation*}
b_{i}=h_{0}+h_{1}i+h_{2}i^{2}+h_{3}i^{3}
\end{equation*}

\noindent In genere Almon propone di utilizzare un polinomio di grado $s<r$:%
\begin{equation*}
b_{i}=h_{0}+h_{1}i+\ldots +h_{s}i^{s}
\end{equation*}

\section{Distribuzione dei ritardi infinita}

\subsection{Il modello}

\begin{equation*}
y_{t}=a+b_{0}x_{t}+b_{1}x_{t-1}+b_{2}x_{t-2}+\ldots +\varepsilon _{t}
\end{equation*}%
del quale analizziamo l'aspetto dinamico e non di stima (tralasciamo
l'errore $\varepsilon _{t}$).

\begin{esempio}
\label{progressione geo}La progressione geometrica decrescente

\begin{equation*}
\text{Grafico}
\end{equation*}%
\vspace{5cm}
\end{esempio}

\begin{eqnarray*}
b_{0} &=&b_{0} \\
b_{1} &=&b_{0}c\ \ \ \ \ \ 0<c<1 \\
b_{2} &=&b_{0}c^{2} \\
&&\vdots \\
b_{s} &=&b_{0}c^{s}
\end{eqnarray*}

\begin{itemize}
\item Effetto immediato: $b_{0}$.

\item Effetto di lungo periodo: $b=\sum b_{i}=b_{0}\left( 1+c+c^{2}+\ldots
\right) =b_{0}\frac{1}{1-c}$.

\item Ritardo medio: $\overline{R}=\frac{c}{1-c}$
\end{itemize}

Introduciamo $b_{i}=b_{0}c^{i}$ nel modello: 
\begin{equation*}
y_{t}=a+b_{0}x_{t}+b_{0}cx_{t-1}+b_{0}c^{2}x_{t-2}+\ldots
\end{equation*}%
dipende da soli 3 parametri: $a$, $b_{0}$, $c$.%
\begin{eqnarray*}
y_{t-1} &=&a+b_{0}x_{t-1}+b_{0}cx_{t-2}+b_{0}c^{2}x_{t-3}+\ldots \ \ \ \ \ \
\ \ \mid \cdot c \\
y_{t}-cy_{t-1} &=&a(1-c)+b_{0}c
\end{eqnarray*}%
da cui ricaviamo la trasformazione denominata \emph{trasformazione di Koyck}%
\begin{equation*}
y_{t}=a^{\ast }+b_{0}x_{t}+cy_{t-1}
\end{equation*}%
In quest'esempio la distribuzione infinita di ritardi in $x$ ha
un'espressione equivalente in termini di due distribuzioni finite di ritardi:

\begin{itemize}
\item una in $y_{t}$ di grado 1 e

\item l'altra in $x_{t}$ di grado 0.
\end{itemize}

\begin{esempio}
Estensione dell'Esempio \ref{progressione geo}
\end{esempio}

\noindent \`{E} immaginabile che la progressione geometrica decrescente si
manifesti non gi\`{a} all'inizio ma soltanto a partire da un certo ritardo. 
\begin{equation*}
\text{Grafico}
\end{equation*}%
\vspace{5cm}

\begin{eqnarray*}
b_{0} &=&\text{libero} \\
b_{1} &=&\text{libero} \\
b_{2} &=&\text{libero} \\
b_{3} &=&b_{2}c \\
b_{4} &=&b_{2}c^{2} \\
&&\vdots
\end{eqnarray*}%
In una situazione del genere, applicando la tecnica utilizzata
precedentemente, si ottiene%
\begin{eqnarray*}
y_{t}
&=&a+b_{0}x_{t}+b_{1}x_{t-1}+b_{2}x_{t-2}+b_{2}cx_{t-3}+b_{2}c^{2}x_{t-4}+%
\ldots \\
y_{t-1} &=&a+\ \ \ \ \ \ \
b_{0}x_{t-1}+b_{1}x_{t-2}+b_{2}x_{t-3}+b_{2}cx_{t-4}+\ldots \ \ \ \ \mid
\cdot c \\
y_{t}-cy_{t-1} &=&a^{\ast
}+b_{0}x_{t}+(b_{1}-cb_{0})x_{t-1}+(b_{2}-cb_{1})x_{t-2}+0+\ldots
\end{eqnarray*}

\noindent Ci\`{o} che \`{e} importante notare in quest'ultimo esempio \`{e}
che la distribuzione infinita in $x$ ha un'espressione finita equivalente in
termini di due distribuzioni finite, una in $y$ di grado 1 ed una in $x$ di
grado 2.

\subsection{Distribuzione razionale dei ritardi:\ idee introduttive}

\noindent Nei due esempi precedenti la distribuzione dei ritardi infinita in 
$x$ ha un'espressione equivalente in termini di due distribuzioni finite,
una in $x$, l'altra in $y$. L'idea fondamentale di Jorgenson \`{e} che
qualsiasi distribuzione infinita in $x$ pu\`{o} essere approssimata a
piacere tramite due distribuzioni finite di ritardi, una in $x$, l'altra in $%
y$.

\subsection{L'operatore ritardo}

\begin{definizione}
L'operatore di ritardo, notato $L$, fa corrispondere a $x_{t}$ il valore
precedente:%
\begin{equation*}
Lx_{t}=x_{t-1}
\end{equation*}
\end{definizione}

\`{E} un operatore lineare, poich\'{e}

\begin{enumerate}
\item $L(x_{t}+z_{t})\underset{def}{=}x_{t-1}+z_{t-1}=Lx_{t}+Lz_{t}$

\item $\lambda $ scalare, $L(\lambda x_{t})=\lambda x_{t-1}=\lambda L(x_{t})$
\end{enumerate}

\noindent Definiamo ora l'elevamento a potenza:%
\begin{equation*}
L^{2}x_{t}=L\left( Lx_{t}\right) =L\left( x_{t-1}\right) =x_{t-2}
\end{equation*}

Con questo operatore, il modello generale si scrive:%
\begin{eqnarray*}
y_{t} &=&a+b_{0}x_{t}+b_{1}Lx_{t}+b_{2}L^{2}x_{t}+... \\
&=&a+\underset{B(L)_{\infty }}{\underbrace{\left(
b_{0}+b_{1}L+b_{2}L^{2}+...\right) }}x_{t}
\end{eqnarray*}%
$B(L)_{\infty }$ rappresenta il polinomio di grado infinito in $L$. Si
verifichi quanto segue:

\begin{itemize}
\item $B(0)=b_{0}$, l'effetto immediato.

\item $B(1)=b_{0}+b_{1}+b_{2}+\ldots =b$, l'effetto di lungo periodo.

\item $B^{\prime }(L)=b_{1}+2b_{2}L+3b_{3}L^{2}+\ldots $

\item $B^{\prime }(1)=b_{1}+2b_{2}+3b_{3}+\ldots $%
\begin{equation*}
\Rightarrow \overline{R}=\frac{B^{\prime }(1)}{B(1)}
\end{equation*}
\end{itemize}

\subsection{Il modello dinamico di Jorgenson (distribuzione razionale dei
ritardi)}

Riprendiamo il modello generale 
\begin{eqnarray}
y_{t} &=&a+b_{0}x_{t}+b_{1}x_{t-1}+b_{2}x_{t-2}+\ldots
\label{equazione di partenza} \\
&=&a+b_{0}x_{t}+b_{1}Lx_{t}+b_{2}L^{2}x_{t}+\ldots  \notag \\
&=&a+\left( b_{0}+b_{1}L+b_{2}L^{2}+\ldots \right) x_{t}  \notag \\
&=&a+B(L)_{\infty }x_{t}  \notag
\end{eqnarray}%
Nei due esempi trattati il polinomio infinito $B(L)_{\infty }$ aveva
un'espressione esatta equivalente in termini di due distribuzioni finite di
ritardi in $x$ e $y$. Jorgenson avvalendosi di un teorema fondamentale
dell'algebra afferma che il polinomio infinito $B(L)_{\infty }$ pu\`{o}
essere approssimato con precisione a piacere da un rapporto di due polinomi
finiti.%
\begin{equation*}
B(L)_{\infty }\simeq \frac{\beta (L)_{s}}{\alpha (L)_{r}},\ \ \ \ s\text{ e }%
r\text{ finiti}
\end{equation*}%
con%
\begin{eqnarray*}
\beta (L)_{s} &=&\beta _{0}+\beta _{1}L+\beta _{2}L^{2}+\ldots +\beta
_{s}L^{s} \\
\alpha (L)_{r} &=&1+\alpha _{1}L+\alpha _{2}L^{2}+\ldots +\alpha _{r}L^{r}
\end{eqnarray*}%
($\alpha _{0}=1$ \`{e} la cosiddetta normalizzazione).

\noindent Introduciamo l'approssimazione nel modello:%
\begin{eqnarray}
y_{t} &=&a+\frac{\beta (L)_{s}}{\alpha (L)_{r}}x_{t}  \notag \\
\alpha (L)_{r}y_{t} &=&\alpha (L)_{r}a+\beta (L)_{s}x_{t}
\label{formulazone partenza} \\
\left( 1+\alpha _{1}L+\ldots +\alpha _{r}L^{r}\right) y_{t} &=&a^{\ast
}+\left( \beta _{0}+\beta _{1}L+\ldots +\beta _{s}L^{s}\right) x_{t}
\label{forma canonica}
\end{eqnarray}%
L'equazione (\ref{forma canonica}) \`{e} anche chiamata \emph{la forma
canonica} del modello. $a^{\ast }$ rappresenta l'espressione $\alpha
(L)_{r}a $ ed \`{e} uguale ad $\alpha (1)_{r}a$ in quanto 
\begin{eqnarray*}
\alpha (L)_{r}a &=&\left( 1+\alpha _{1}L+\ldots +\alpha _{r}L^{r}\right) a\
\ \ \ \ \ \ \text{con }La=a \\
&=&\left( 1+\alpha _{1}+\ldots +\alpha _{r}\right) a \\
&=&\alpha (1)_{r}a\ .
\end{eqnarray*}%
L'\emph{equazione stimabile} di questo modello:%
\begin{eqnarray}
y_{t} &=&a^{\ast }+\beta _{0}x_{t}+\beta _{1}x_{t-1}+\ldots +\beta
_{s}x_{t-s}-\alpha _{1}y_{t-1}-\ldots -\alpha _{r}y_{t-r}+\varepsilon _{t} 
\notag \\
y_{t} &=&a^{\ast }+\beta _{0}x_{t}+\beta _{1}x_{t-1}+\ldots +\beta
_{s}x_{t-s}+\alpha _{1}^{\ast }y_{t-1}+\ldots +\alpha _{r}^{\ast
}y_{t-r}+\varepsilon _{t}  \label{equazione_stimabile_jorgenson}
\end{eqnarray}%
In pratica $s$ e $r$ non sono conosciuti, quindi si stima il modello per
svariate coppie $\left( s,r\right) $, ad esempio la coppia $%
(1,1);(1,2);\ldots $ Si sceglier\`{a} la coppia che d\`{a} migliori
risultati in termini di significativit\`{a} dei coefficienti e di valori del
criterio d'informazione di Akaike.

\noindent Avendo stimato i polinomi $\alpha (L)\ $e $\beta (L)$ (tralasciamo
i rispettivi indici $r$ ed $s$) si possono ottenere:

\begin{enumerate}
\item L'effetto immediato

\item L'effetto dinamico di lungo periodo

\item Il ritardo medio

\item Il profilo dei ritardi (ovvero i coefficienti del polinomio $%
B(L)_{\infty }$)
\end{enumerate}

\noindent Per le prime 3 domande utilizziamo la relazione%
\begin{equation*}
B(L)_{\infty }=\frac{\beta (L)}{\alpha (L)}
\end{equation*}%
per la quarta sfruttiamo la ricorrenza.

\begin{enumerate}
\item L'effetto immediato

\begin{equation*}
b_{0}=B(0)_{\infty }=\frac{\beta (0)}{\alpha (0)}=\frac{\beta _{0}}{1}=\beta
_{0}
\end{equation*}

\item L'effetto di lungo periodo

\begin{equation*}
b=b_{0}+b_{1}+b_{2}+\ldots =B(1)_{\infty }=\frac{\beta (1)}{\alpha (1)}=%
\frac{\beta _{0}+\beta _{1}+\ldots +\beta _{s}}{1+\alpha _{1}+\ldots +\alpha
_{r}}=\frac{\beta _{0}+\beta _{1}+\ldots +\beta _{s}}{1-\alpha _{1}^{\ast
}-\ldots -\alpha _{r}^{\ast }}
\end{equation*}

\item Il ritardo medio. Ricordiamo che%
\begin{equation*}
\overline{R}=\frac{B^{\prime }(1)}{B(1)}
\end{equation*}%
Ora, 
\begin{eqnarray*}
B^{\prime }(L) &=&\frac{\beta ^{\prime }(L)\alpha (L)-\beta (L)\alpha
^{\prime }(L)}{\alpha (L)^{2}} \\
&=&\frac{\beta ^{\prime }(L)}{\alpha (L)}-\frac{\beta (L)}{\alpha (L)^{2}}%
\alpha ^{\prime }(L) \\
&=&\frac{\beta (L)}{\alpha (L)}\left[ \frac{\beta ^{\prime }(L)}{\beta (L)}-%
\frac{\alpha ^{\prime }(L)}{\alpha (L)}\right]
\end{eqnarray*}%
e quindi essendo 
\begin{equation*}
B^{\prime }(1)=\frac{\beta (1)}{\alpha (1)}\left[ \frac{\beta ^{\prime }(1)}{%
\beta (1)}-\frac{\alpha ^{\prime }(1)}{\alpha (1)}\right]
\end{equation*}%
otteniamo che%
\begin{equation*}
\overline{R}=\frac{B^{\prime }(1)}{B(1)}=\frac{\beta ^{\prime }(1)}{\beta (1)%
}-\frac{\alpha ^{\prime }(1)}{\alpha (1)}
\end{equation*}
\end{enumerate}

\begin{esempio}
Con dati trimestrali abbiamo stimato la seguente funzione di consumo:%
\begin{equation*}
C_{t}=160+0.30R_{t}+0.20R_{t-1}+0.14R_{t-2}+0.2C_{t-1}
\end{equation*}
\end{esempio}

\begin{itemize}
\item La forma canonica%
\begin{eqnarray*}
C_{t}-0.2C_{t-1} &=&160+0.30R_{t}+0.20R_{t-1}+0.14R_{t-2} \\
\underset{\alpha (L)_{1}}{\underbrace{\left( 1-0.2L\right) }}C_{t} &=&160+%
\underset{\beta (L)_{2}}{\underbrace{\left( 0.30+0.20L+0.14L^{2}\right) }}%
R_{t}
\end{eqnarray*}

Calcoliamo:

\begin{itemize}
\item $\beta (1)=0.3+0.2+0.14=0.64$

\item $\beta ^{\prime }(L)=0.2+0.28L$

\item $\beta ^{\prime }(1)=0.48$

\item $\alpha (0)=1$ (sempre!), $\alpha (1)=0.8$

\item $\alpha ^{\prime }(L)=-0.2$

\item $\alpha ^{\prime }(1)=-0.2$
\end{itemize}

\item L'effetto immediato: $b_{0}=\beta (0)=0.3$

\item L'effetto dinamico di lungo periodo:

\begin{equation*}
b=B(1)_{\infty }=\frac{\beta (1)}{\alpha (1)}=\frac{0.48}{0.64}=0.8\text{
(propensione marginale di L.P.)}
\end{equation*}

\item Il ritardo medio%
\begin{eqnarray*}
\overline{R} &=&\frac{B^{\prime }(1)}{B(1)}=\frac{\beta ^{\prime }(1)}{\beta
(1)}-\frac{\alpha ^{\prime }(1)}{\alpha (1)} \\
&=&\frac{0.48}{0.64}+\frac{0.2}{0.8}=1
\end{eqnarray*}

\item Il profilo dei ritardi

Ammettiamo un punto di partenza d'equilibrio di lungo periodo con tutti gli
incrementi passati uguali a 0 per ogni $t<0$.%
\begin{equation*}
\Delta C_{t}=\Delta R_{t}=0\ \ \forall t<0
\end{equation*}%
Scriviamo la nostra equazione in termini di incrementi:%
\begin{equation*}
\Delta C_{t}=0.3\Delta R_{t}+0.2\Delta R_{t-1}+0.14\Delta R_{t-2}+0.2\Delta
C_{t-1}
\end{equation*}%
In $t=0$, consideriamo un aumento unitario del reddito%
\begin{equation*}
\Delta R_{0}=1
\end{equation*}%
e studiamo il suo influsso sul consumo presente e futuro%
\begin{eqnarray*}
\Delta C_{0} &=&0.3\underset{1}{\underbrace{\Delta R_{0}}}+0.2\underset{0}{%
\underbrace{\Delta R_{-1}}}+0.14\underset{0}{\underbrace{\Delta R_{-2}}}+0.2%
\underset{0}{\underbrace{\Delta C_{-1}}}=0.3=b_{0} \\
\Delta C_{1} &=&0.3\underset{0}{\underbrace{\Delta R_{1}}}+0.2\underset{1}{%
\underbrace{\Delta R_{0}}}+0.14\Delta R_{-1}+0.2\underset{0.3}{\underbrace{%
\Delta C_{0}}}=0.26=b_{1} \\
\Delta C_{2} &=&0.3\underset{0}{\underbrace{\Delta R_{2}}}+0.2\underset{0}{%
\underbrace{\Delta R_{1}}}+0.14\underset{1}{\underbrace{\Delta R_{0}}}+0.2%
\underset{0.26}{\underbrace{\Delta C_{1}}}=0.192=b_{2} \\
\Delta C_{3} &=&0.2\underset{0.192}{\underbrace{\Delta C_{2}}}=0.0384=b_{3}
\\
\Delta C_{4} &=&0.2\underset{0.0384}{\underbrace{\Delta C_{3}}}=\ldots
\end{eqnarray*}

Verifichiamo:%
\begin{eqnarray*}
b &=&b_{0}+b_{1}+b_{2}+\ldots \\
&=&0.3+0.26+0.192\left( 1+0.2+0.2^{2}+\ldots \right) \\
&=&0.56+\frac{0.192}{1-0.2}=0.8
\end{eqnarray*}
\end{itemize}

\begin{equation*}
\text{grafico}
\end{equation*}%
\vspace{5cm}

\subsubsection{L'equazione d'equilibrio di lungo periodo}

Poniamo 
\begin{equation*}
\left\{ 
\begin{array}{c}
y_{t}=y^{\ast } \\ 
x_{t}=x^{\ast }%
\end{array}%
\right. \ \ \ \forall t
\end{equation*}%
Dalla formulazione di partenza (\ref{formulazone partenza})%
\begin{eqnarray*}
\alpha (L)y_{t} &=&a^{\ast }+\beta (L)x_{t} \\
\alpha (L)y^{\ast } &=&a^{\ast }+\beta (L)x^{\ast } \\
\alpha (1)y^{\ast } &=&a^{\ast }+\beta (1)x^{\ast }
\end{eqnarray*}%
da cui otteniamo la relazione di lungo periodo%
\begin{eqnarray}
y^{\ast } &=&\frac{a^{\ast }}{\alpha (1)}+\frac{\beta (1)}{\alpha (1)}%
x^{\ast }  \notag \\
y^{\ast } &=&a+bx^{\ast }  \label{equazione di lungo periodo}
\end{eqnarray}%
dove $b$ rappresenta l'effetto di lungo periodo. L'equazione (\ref{equazione
di lungo periodo}) \`{e} derivabile direttamente dalla (\ref{equazione di
partenza}), ovvero%
\begin{eqnarray*}
y_{t} &=&a+b_{0}x_{t}+b_{1}x_{t-1}+b_{2}x_{t-2}+\ldots \\
y^{\ast } &=&a+b_{0}x^{\ast }+b_{1}x^{\ast }+b_{2}x^{\ast }+\ldots \\
&=&a+(b_{0}+b_{1}+b_{2}+\ldots )x^{\ast } \\
&=&a+bx^{\ast }
\end{eqnarray*}%
Nell'esempio:%
\begin{eqnarray*}
C^{\ast } &=&160+0.3R^{\ast }+0.2R^{\ast }+0.14R^{\ast }+0.2C^{\ast } \\
C^{\ast } &=&\frac{160}{0.8}+\frac{0.64}{0.8}R^{\ast } \\
C^{\ast } &=&200+0.8R^{\ast }
\end{eqnarray*}%
Se in $t-1$ ponessimo%
\begin{equation}
\varepsilon _{t-1}^{\ast }=C_{t-1}-200-0.8R_{t-1}
\label{scarto dall equilibrio}
\end{equation}%
otterremmo il cosiddetto scarto dall'equilibrio di lungo termine (o errore)
in $t-1$.

\chapter{Serie macroeconomiche non stazionarie}

\section{Stazionariet\`{a} e non stazionariet\`{a}}

\subsection{Stazionariet\`{a}}

Una serie storica \`{e} stazionaria se soddisfa tre principi:

\begin{enumerate}
\item Tende a ritornare alla media

\item La varianza \`{e} costante (non dipende dal tempo)

\item La covarianza fra due termini della serie \`{e} unicamente funzione
della distanza che separa i due termini e tende verso 0 quando questa
distanza aumenta.
\end{enumerate}

\begin{equation*}
\text{Grafico}
\end{equation*}%
\vspace{5cm}

\begin{esempio}
Rumore bianco (White Noise)
\end{esempio}

\begin{equation*}
\varepsilon _{t}\sim iid(0,\sigma ^{2})
\end{equation*}

\begin{esempio}
\label{esempioAR1}Il modello AR(1)
\end{esempio}

\begin{equation*}
\left\{ 
\begin{array}{c}
\varepsilon _{t}=\rho \varepsilon _{t-1}+u_{t} \\ 
u_{t}\sim iid(0,\sigma ^{2}) \\ 
|\rho| <1%
\end{array}%
\right.
\end{equation*}

\begin{itemize}
\item $E(\varepsilon _{t})=0.$

\item $V(\varepsilon _{t})=\frac{\sigma ^{2}}{1-\rho ^{2}}$ indipendente da $%
t$.

\item $Cov(\varepsilon _{t},\varepsilon _{t-s})=\frac{\sigma ^{2}}{1-\rho
^{2}}\rho ^{s}\underset{s\rightarrow \infty }{\longrightarrow }0.$
\end{itemize}

\begin{esempio}
Il modello $ARMA(p,q)$
\end{esempio}

\subsection{Non stazionariet\`{a}}

Una serie storica \`{e} non stazionaria se:

\begin{enumerate}
\item Si scosta indefinitivamente dalla media

\item La varianza \`{e} proporzionale al tempo

\item La correlazione fra due termini della serie tende verso uno quando la
distanza tra i due termini aumenta.
\end{enumerate}

\begin{equation*}
\text{Grafico}
\end{equation*}%
\vspace{5cm}

\begin{esempio}
\label{es RW}Cammino casuale o aleatorio (random walk). Nell'Esempio \ref%
{esempioAR1} poniamo $\rho =1$:
\end{esempio}

\begin{equation*}
\varepsilon _{t}=\varepsilon _{t-1}+u_{t}\sim White\ Noise
\end{equation*}%
\begin{eqnarray*}
\varepsilon _{0} &=&0 \\
\varepsilon _{1} &=&0+u_{1} \\
\varepsilon _{2} &=&\varepsilon _{1}+u_{2}=u_{1}+u_{2} \\
\varepsilon _{3} &=&\varepsilon _{2}+u_{3}=u_{1}+u_{2}+u_{3} \\
&&\vdots \\
\varepsilon _{t} &=&u_{1}+u_{2}+u_{3}+\ldots +u_{t}
\end{eqnarray*}%
Possiamo allora calcolare valore atteso e varianza:%
\begin{eqnarray*}
E(\varepsilon _{t}) &=&0 \\
V(\varepsilon _{t}) &=&t\sigma ^{2} \\
Cor(\varepsilon _{t},\varepsilon _{t-s}) &=&\frac{Cov(\varepsilon
_{t},\varepsilon _{t-s})}{\sqrt{V(\varepsilon _{t})V(\varepsilon _{t-s})}}=%
\frac{(t-s)}{\sqrt{t(t-s)}} \\
\lim_{t\rightarrow \infty }Cor(\varepsilon _{t},\varepsilon _{t-s}) &=&1
\end{eqnarray*}

\noindent Due autori sono arrivati alla conclusione che quasi tutte le
variabili macroeconomiche sono non stazionarie (Nelson e Plosser). Si cerca
quindi di rendere stazionarie le variabili prima di eseguire l'analisi
econometrica.

\noindent Una serie $\left\{ \varepsilon _{t}\right\} $ non stazionaria \`{e}
detta integrata di ordine 1, e si scrive%
\begin{equation*}
\varepsilon _{t}\sim I(1)
\end{equation*}%
se la differenza prima 
\begin{equation*}
d_{t}:=\varepsilon _{t}-\varepsilon _{t-1}
\end{equation*}%
\`{e} stazionaria (cio\'{e} $I(0)$). Nel nostro esempio \`{e} evidente che
un Random Walk \`{e} $I(1)$ in quanto 
\begin{equation*}
\varepsilon _{t}-\varepsilon _{t-1}=u_{t}\sim WN
\end{equation*}%
Per vedere se una serie \`{e} non stazionaria, stimo il modello AR(1): $%
\varepsilon _{t}=\rho \varepsilon _{t-1}+u_{t}$. Dopo aver stimato $\rho $
conduco il test d'ipotesi $H_{0}:\rho =1$ (test della radice unitaria). Se
accetto l'ipotesi, allora la serie \`{e} non stazionaria. C'\`{e} per\`{o}
un problema. Sotto l'ipotesi $H_{0}$ lo stimatore $\widehat{\rho }$ non ha
le propriet\`{a} statistiche che conosciamo. Torniamo al modello $AR(1)$: 
\begin{eqnarray*}
\varepsilon _{t} &=&\rho \varepsilon _{t-1}+u_{t} \\
\varepsilon _{t}-\rho \varepsilon _{t-1} &=&u_{t} \\
(1-\rho L)\varepsilon _{t} &=&u_{t}
\end{eqnarray*}%
una radice unitaria sarebbe quando trovo $\rho =1$. In tal caso scriviamo
anche $(1-L)=\Delta $.

\subsection{Una riparametrizzazione del modello dinamico classico (Jorgenson)%
}

\begin{equation}
y_{t}=\beta _{0}x_{t}+\beta _{1}x_{t-1}+\beta _{2}x_{t-2}+\alpha _{1}^{\ast
}y_{t-1}+\alpha _{2}^{\ast }y_{t-2}+\varepsilon _{t}
\label{modello Jorgenson}
\end{equation}%
5 parametri da stimare.

\begin{itemize}
\item Effetto immediato: $\beta _{0}$

\item Effetto di lungo termine: $b=\frac{\beta _{0}+\beta _{1}+\beta _{2}}{%
1-\alpha _{1}^{\ast }-\alpha _{2}^{\ast }}$

\item Relazione di lungo termine: $y_{t}=bx_{t}$.

Se in (\ref{modello Jorgenson}) adopero i valori stazionari:%
\begin{equation*}
\left\{ 
\begin{array}{c}
y_{t}=y^{\ast } \\ 
x_{t}=x^{\ast }%
\end{array}%
\right. \ \ \forall t
\end{equation*}%
\begin{eqnarray*}
y^{\ast } &=&\left( \beta _{0}+\beta _{1}+\beta _{2}\right) x^{\ast }+\left(
\alpha _{1}^{\ast }+\alpha _{2}^{\ast }\right) y^{\ast } \\
y^{\ast } &=&\frac{\beta _{0}+\beta _{1}+\beta _{2}}{1-\alpha _{1}^{\ast
}-\alpha _{2}^{\ast }}x^{\ast } \\
y^{\ast } &=&bx^{\ast }
\end{eqnarray*}%
da cui otteniamo lo scarto dall'equilibrio di lungo termine (errore): $%
y_{t}-bx_{t}$.

\item Prima di affrontare il caso particolare dato dall'equazione stimabile (%
\ref{modello Jorgenson}) consideriamo il caso generale dato dalla (\ref%
{equazione_stimabile_jorgenson}), ovvero

\begin{equation*}
y_{t}=a^{\ast }+\underset{p1}{\underbrace{\beta _{0}x_{t}+\beta
_{1}x_{t-1}+\ldots +\beta _{s}x_{t-s}}}+\underset{p2}{\underbrace{\alpha
_{1}^{\ast }y_{t-1}+\ldots +\alpha _{r}^{\ast }y_{t-r}}}+\varepsilon _{t}
\end{equation*}%
Consideriamo dapprima la riparametrizzazione della parte 1:%
\begin{eqnarray*}
p1 &=&\beta _{0}x_{t}+\left( \beta _{0}+\beta _{1}+\ldots +\beta _{s}-\beta
_{0}-\beta _{2}-\ldots -\beta _{s}\right) x_{t-1}+ \\
&&+\left( \beta _{2}+\beta _{3}+\ldots +\beta _{s}-\beta _{3}-\ldots -\beta
_{s}\right) x_{t-2} \\
&&+\left( \beta _{3}+\ldots +\beta _{s}-\beta _{4}-\ldots -\beta _{s}\right)
x_{t-3} \\
&&\vdots \\
&&+\left( \beta _{s-1}+\beta _{s}-\beta _{s}\right) x_{t-(s-1)} \\
&&+\beta _{s}x_{t-s} \\
&=&\beta _{0}\Delta x_{t}+\left( \beta _{0}+\beta _{1}+\ldots +\beta
_{s}\right) x_{t-1} \\
&&-\left( \beta _{2}+\ldots +\beta _{s}\right) \Delta x_{t-1} \\
&&-\left( \beta _{3}+\ldots +\beta _{s}\right) \Delta x_{t-2} \\
&&\vdots \\
&&-\left( \beta _{s-1}+\beta _{s}\right) \Delta x_{t-(s-2)} \\
&&-\beta _{s}\Delta x_{t-(s-1)}
\end{eqnarray*}%
e ridefinendo i coefficienti delle nuove variabili esplicative otteniamo per
questa prima parte%
\begin{equation*}
p1=\delta _{0}\Delta x_{t}+\gamma _{1}x_{t-1}+\delta _{1}\Delta
x_{t-1}+\ldots +\delta _{s-1}\Delta x_{t-(s-1)}.
\end{equation*}%
La riparametrizzazione della parte 2 \`{e} quasi identica alla precedente.
Abbiamo infatti%
\begin{eqnarray*}
p2 &=&\left( \alpha _{1}^{\ast }+\alpha _{2}^{\ast }+...+\alpha _{r}^{\ast
}-\alpha _{2}^{\ast }-\ldots -\alpha _{r}^{\ast }\right) y_{t-1} \\
&&+\left( \alpha _{2}^{\ast }+\alpha _{3}^{\ast }+...+\alpha _{r}^{\ast
}-\alpha _{3}^{\ast }-\ldots -\alpha _{r}^{\ast }\right) y_{t-2} \\
&&+\left( \alpha _{3}^{\ast }+\alpha _{4}^{\ast }+...+\alpha _{r}^{\ast
}-\alpha _{4}^{\ast }-\ldots -\alpha _{r}^{\ast }\right) y_{t-3} \\
&&\vdots \\
&&+\left( \alpha _{r-1}^{\ast }+\alpha _{r}^{\ast }-\alpha _{r}^{\ast
}\right) y_{t-(r-1)} \\
&&+\alpha _{r}^{\ast }y_{t-r} \\
&=&\left( \alpha _{1}^{\ast }+\alpha _{2}^{\ast }+...+\alpha _{r}^{\ast
}\right) y_{t-1} \\
&&-(\alpha _{2}^{\ast }+\ldots +\alpha _{r}^{\ast })\Delta y_{t-1} \\
&&-(\alpha _{3}^{\ast }+\ldots +\alpha _{r}^{\ast })\Delta y_{t-2} \\
&&\vdots \\
&&-\left( \alpha _{r-1}^{\ast }+\alpha _{r}^{\ast }\right) \Delta y_{t-(r-2)}
\\
&&-\alpha _{r}^{\ast }\Delta y_{t-(r-1)}
\end{eqnarray*}%
Anche in questo caso ridefiniamo i coefficienti dei $\Delta y$%
\begin{equation*}
p2=\left( \alpha _{1}^{\ast }+\ldots +\alpha _{r}^{\ast }\right)
y_{t-1}+q_{1}\Delta y_{t-1}+q_{2}\Delta y_{t-2}+\ldots +q_{r-1}\Delta
y_{t-(r-1)}
\end{equation*}%
Ricapitolando obbiamo che 
\begin{equation*}
y_{t}=a^{\ast }+\gamma _{1}x_{t-1}+\left( \alpha _{1}^{\ast }+\ldots +\alpha
_{r}^{\ast }\right) y_{t-1}+\sum_{i=0}^{s-1}\delta _{s}\Delta
x_{t-i}+\sum_{i=1}^{r-1}q_{s}\Delta y_{t-i}+\varepsilon _{t}\ 
\end{equation*}%
e sottraendo da entrambi i lati $y_{t-1}$ si ottiene la forma finale%
\begin{eqnarray*}
y_{t}-y_{t-1} &=&a^{\ast }+\gamma _{1}x_{t-1}+\underset{\gamma _{2}}{%
\underbrace{\left( \alpha _{1}^{\ast }+\ldots +\alpha _{r}^{\ast }-1\right) }%
}y_{t-1}+\sum_{i=0}^{s-1}\delta _{s}\Delta
x_{t-i}+\sum_{i=1}^{r-1}q_{s}\Delta y_{t-i}+\varepsilon _{t}\  \\
\Delta y_{t} &=&a^{\ast }+\gamma _{1}x_{t-1}+\gamma
_{2}y_{t-1}+\sum_{i=0}^{s-1}\delta _{s}\Delta
x_{t-i}+\sum_{i=1}^{r-1}q_{s}\Delta y_{t-i}+\varepsilon _{t}
\end{eqnarray*}

\item Dalla (\ref{modello Jorgenson}):%
\begin{eqnarray*}
y_{t} &=&\beta _{0}x_{t}+\left( \beta _{0}+\beta _{1}+\beta _{2}-\beta
_{0}-\beta _{2}\right) x_{t-1}+\beta _{2}x_{t-2}+ \\
&&\left( \alpha _{1}^{\ast }+\alpha _{2}^{\ast }-\alpha _{2}^{\ast }\right)
y_{t-1}+\alpha _{2}^{\ast }y_{t-2}+\varepsilon _{t} \\
y_{t} &=&\underset{\delta _{0}}{\underbrace{\beta _{0}}}\ \underset{\Delta
x_{t}}{\underbrace{\left( x_{t}-x_{t-1}\right) }\ +}\underset{=\gamma _{1}}{%
\underbrace{\left( \beta _{0}+\beta _{1}+\beta _{2}\right) }}x_{t-1}\underset%
{\delta _{1}}{\underbrace{-\beta _{2}}}\ \underset{\Delta x_{t-1}}{%
\underbrace{\left( x_{t-1}-x_{t-2}\right) }}+ \\
&&\left( \alpha _{1}^{\ast }+\alpha _{2}^{\ast }\right) y_{t-1}\underset{%
q_{1}}{\underbrace{-\alpha _{2}^{\ast }}}\underset{\Delta y_{t-1}}{%
\underbrace{\left( y_{t-1}-y_{t-2}\right) }}+\varepsilon _{t}
\end{eqnarray*}%
\begin{eqnarray}
\underset{\Delta y_{t}}{\underbrace{y_{t}-y_{t-1}}} &=&\gamma _{1}x_{t-1}+%
\underset{\gamma _{2}}{\underbrace{\left( \alpha _{1}^{\ast }+\alpha
_{2}^{\ast }-1\right) }}y_{t-1}+\delta _{0}\Delta x_{t}+\delta _{1}\Delta
x_{t-1}+q_{1}\Delta y_{t-1}+\varepsilon _{t}  \label{modello ridefinito} \\
\Delta y_{t} &=&\gamma _{1}x_{t-1}+\gamma _{2}y_{t-1}+\gamma _{3}\Delta
x_{t}+\gamma _{4}\Delta x_{t-1}+\gamma _{5}\Delta y_{t-1}+\varepsilon _{t}
\end{eqnarray}

Quest'equazione comporta 5 parametri come la (\ref{modello Jorgenson})
identificabili. Stimata con i $m.q.o.$ produce gli stessi risultati
(trasformati) dei coefficienti e lo stesso $SS$.

\begin{eqnarray*}
y &=&X\beta +\varepsilon \\
\beta &=&A\gamma +L
\end{eqnarray*}%
dove $A$ rappresenta una matrice nota non singolare ed $L$ un vettore noto.
E' possibile risalire ad $A$ ed $L$ confrontando le due formulazioni (\ref%
{modello Jorgenson}) e (\ref{modello ridefinito}) del modello. Infatti%
\begin{eqnarray*}
y &=&X\left( A\gamma +L\right) +\varepsilon \\
y &=&\underset{W}{\underbrace{XA}}\gamma +\underset{\text{conosciuto}}{%
\underbrace{XL}}+\varepsilon \\
\left( y-XL\right) &=&W\gamma +\varepsilon
\end{eqnarray*}%
confrontando i termini dell'equazione (\ref{modello ridefinito}) con
ques'ultima espressione \`{e} possibile evincere il contenuto della matrice $%
A$ e del vettore $L$.%
\begin{equation*}
L=\left[ 
\begin{array}{c}
0 \\ 
0 \\ 
0 \\ 
1 \\ 
0%
\end{array}%
\right] ;A=\left[ 
\begin{array}{ccccc}
0 & 1 & 0 & 0 & 0 \\ 
1 & -1 & 1 & 0 & 0 \\ 
0 & 0 & -1 & 0 & 0 \\ 
0 & 0 & 0 & 1 & 1 \\ 
0 & 0 & 0 & 0 & -1%
\end{array}%
\right]
\end{equation*}%
Inoltre dall'equazione (\ref{modello ridefinito}) \`{e} possibile derivare
direttamente il moltiplicatore di lungo termine come 
\begin{equation*}
b=\frac{\gamma _{1}}{-\gamma _{2}}=\frac{\beta _{0}+\beta _{1}+\beta _{2}}{%
1-\alpha _{1}^{\ast }-\alpha _{2}^{\ast }}=\frac{\beta (1)}{\alpha (1)}
\end{equation*}

\noindent Riscriviamo ora la (\ref{modello ridefinito}) nel seguente modo:%
\begin{eqnarray}
\Delta y_{t} &=&\gamma _{2}\left( y_{t-1}+\frac{\gamma _{1}}{\gamma _{2}}%
x_{t-1}\right) +\sum_{i=0}^{1}\delta _{i}\Delta
x_{t-i}+\sum_{i=1}^{1}r_{i}\Delta y_{t-i}  \notag \\
\Delta y_{t} &=&\gamma _{2}\left( y_{t-1}-bx_{t-1}\right)
+\sum_{i=0}^{1}\delta _{i}\Delta x_{t-i}+\sum_{i=1}^{1}r_{i}\Delta y_{t-i}
\label{error correction formulation}
\end{eqnarray}

\item $\gamma _{2}$: coefficiente di correzione dell'errore.

\item $y_{t-1}-bx_{t-1}$: scarto (errore) dall'equilibrio di lungo termine, $%
\varepsilon _{t-1}$

\item $\sum_{s=0}^{1}\delta _{s}\Delta x_{t-s}+\sum_{s=1}^{1}r_{s}\Delta
y_{t-s}$: dinamica di corto periodo: tanti $\Delta x_{t-i}$ quanti i numeri
di ritardi ($s$), tanti $\Delta y_{t-i}$ quanti i numeri di ritardi in $y$
meno $1$ ($r-1$).
\end{itemize}

\noindent L'equazione (\ref{error correction formulation}) \`{e} chiamata
modello a correzione d'errore: descrive come l'errore di equilibrio di lungo
termine \`{e} corretto ed esprime la dinamica di corto periodo. Questa
equazione \`{e} alla base dell'approccio delle serie storiche per il
concetto di cointegrazione. L'equazione (\ref{error correction formulation})
non \`{e} direttamente stimabile in quanto il parametro $b$ non \`{e}
conosciuto.

\noindent L'equazione (\ref{error correction formulation}) non \`{e}
direttamente stimabile in quanto il parametro $b$ non \`{e} conosciuto.

\subsection{Introduzione alla cointegrazione}

Due serie non stazionarie possono ammettere una relazione stabile di lungo
periodo, vale a dire che sebbene prese singolarmente le due serie tendano a
divergere indefinitamente, lo scarto dell'una rispetto all'altra \`{e}
qualcosa di stabile o pi\`{u} propriamente detto qualcosa di stazionario.%
\begin{equation*}
\text{Grafico}
\end{equation*}%
\vspace{5cm}

\begin{definition}
2 serie non stazionarie ed integrate d'ordine 1%
\begin{eqnarray*}
y_{t} &\sim &I(1) \\
x_{t} &\sim &I(1)
\end{eqnarray*}%
si dicono cointegrate se esiste una relazione lineare tra le due%
\begin{equation*}
y_{t}-bx_{t}
\end{equation*}%
che \`{e} stazionaria%
\begin{equation*}
y_{t}-bx_{t}\sim I(0)
\end{equation*}
\end{definition}

\subsubsection{La procedura di stima}

La procedura di stima di serie storiche cointegrate comprende le seguenti
tappe:

\begin{enumerate}
\item Verificare attraverso il test di radice unitaria che le due serie sono
effettivamente $I(1)$.

\item Stimare la relazione stazionaria di lungo termine%
\begin{equation*}
y_{t}=bx_{t}+\varepsilon _{t}
\end{equation*}%
e verificare che l'errore stimato $\widehat{\varepsilon }_{t}$ \`{e}
stazionario (che non abbia radice unitaria). Come visto precedentemente $%
\varepsilon _{t}$ rappresenta lo scarto dall'equilibrio di lungo periodo.
Generalmente l'errore $\varepsilon _{t}$ non \`{e} $i.i.d.$ ma pu\`{o} avere
una dinamica anche complessa. Ci\`{o} che \`{e} richiesto affinch\'{e} si
possa parlare di cointegrazione \`{e} che $\varepsilon _{t}$ sia un processo
stazionario: nel corto periodo $y_{t}$ e $bx_{t}$ possono divergere ma nel
lungo periodo devono nuovamente riavvicinarsi. Da qui la necessit\`{a} di
verificare (testare tramite test appropriati) la stazionariet\`{a}
dell'errore stimato $\widehat{\varepsilon }_{t}$.

\item Se l'ipotesi di stazionariet\`{a} del termine d'errore non \`{e}
rigettata si introduce $\widehat{\varepsilon }_{t-1}$ nel modello a
correzione d'errore (vedi equazione (\ref{error correction formulation})).
In questo modello, tutte le variabili esplicative sono stazionarie:%
\begin{eqnarray*}
\Delta x_{t-s} &\sim &I(0) \\
\Delta y_{t-r} &\sim &I(0) \\
\widehat{\varepsilon }_{t-1} &\sim &I(0)
\end{eqnarray*}%
e quindi pu\`{o} essere stimato con i $m.q.o.$.
\end{enumerate}

\begin{remark}
Questa procedura garantisce la convergenza in probabilit\`{a} degli
stimatori. Per ottenere l'efficienza asintotica occorre utilizzare il metodo
di massima verosimiglianza (metodo di Johanson).
\end{remark}

\end{document}
