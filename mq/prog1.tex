
\documentclass[a4paper,12pt]{article}
%%%%%%%%%%%%%%%%%%%%%%%%%%%%%%%%%%%%%%%%%%%%%%%%%%%%%%%%%%%%%%%%%%%%%%%%%%%%%%%%%%%%%%%%%%%%%%%%%%%%%%%%%%%%%%%%%%%%%%%%%%%%%%%%%%%%%%%%%%%%%%%%%%%%%%%%%%%%%%%%%%%%%%%%%%%%%%%%%%%%%%%%%%%%%%%%%%%%%%%%%%%%%%%%%%%%%%%%%%%%%%%%%%%%%%%%%%%%%%%%%%%%%%%%%%%%
%TCIDATA{OutputFilter=LATEX.DLL}
%TCIDATA{Version=5.50.0.2953}
%TCIDATA{<META NAME="SaveForMode" CONTENT="1">}
%TCIDATA{BibliographyScheme=Manual}
%TCIDATA{LastRevised=Wednesday, February 17, 2010 14:24:35}
%TCIDATA{<META NAME="GraphicsSave" CONTENT="32">}

\topmargin-0.7cm
\headheight0cm
\headsep0cm
\textheight26cm
\pagestyle{empty}
\parindent0cm

\begin{document}


{\small \textsc{universit\`{a} della svizzera italiana}}\hfill {\small 
\textsc{facolt\`{a} di scienze economiche}}\newline
{\small \textsc{dr. claudio ortelli}}\hfill {\small \textsc{semestre estivo
2010}}

\vspace{0.4cm}

\begin{center}
\textbf{\large METODI QUANTITATIVI}
\end{center}

\vspace{0.1cm}

\begin{itemize}
\item Parte comune ai profili economico e finanziario (le prime sette
settimane del semestre estivo)

\item Assistenti: Diego Ronchetti, Elisa Ossola
\end{itemize}

\begin{center}
\vspace{0.4cm}

\textbf{\large LA STIMA DEI MODELLI ECONOMETRICI} \\[0pt]
\textbf{{\large (al di l\`{a} della regressione classica)}}
\end{center}

\vspace{1cm}

\underline{Complementi alla regressione classica:}

\begin{enumerate}
\item[1] La regressione scomposta
\end{enumerate}

\noindent \underline{Parte prima: La regressione generalizzata}

\begin{enumerate}
\item[2] Il modello generale

\begin{enumerate}
\item[2.1] Introduzione

\begin{itemize}
\item Breve ritorno sulla regressione classica

\item Modelli econometrici che non soddisfano le ipotesi classiche
\end{itemize}

\item[2.2] Il modello e le ipotesi

\item[2.3] Quando la matrice delle varianze-covarianze \`{e} conosciuta: i
minimi quadrati generalizzati

\begin{itemize}
\item Gli stimatori e le loro propriet\`{a}

\item L'induzione statistica

\item La previsione
\end{itemize}

\item[2.4] Le conseguenze dell'applicazione dei minimi quadrati ordinari

\begin{itemize}
\item Correttezza e perdita di efficienza

\item Stima convergente della varianza dello stimatore dei minimi quadrati
ordinari

\item Casi in cui i minimi quadrati ordinari ed i minimi quadrati
generalizzati sono equivalenti
\end{itemize}

\item[2.5] Quando la matrice delle varianze-covarianze dipende da parametri
sconosciuti

\begin{itemize}
\item I minimi quadrati generalizzati a due tappe

\item Il massimo di verosimiglianza\newpage
\end{itemize}
\end{enumerate}

\item[3] L'autocorrelazione

\begin{enumerate}
\item[3.1] Il modello markoviano di primo ordine

\begin{itemize}
\item Ipotesi e propriet\`{a}

\item Il test dell'autocorrelazione

\item I metodi di stima

\item La previsione
\end{itemize}

\item[3.2] Altre forme d'autocorrelazione
\end{enumerate}
\end{enumerate}

\bigskip

\underline{Parte seconda: La regressione stocastica}

\begin{enumerate}
\item[4] Elementi di analisi asintotica

\begin{enumerate}
\item[4.1] Convergenza in probabilit\`{a}

\begin{itemize}
\item Definizione e propriet\`{a}

\item Convergenza in media quadratica

\item Convergenza dei momenti del campione
\end{itemize}

\item[4.2] Convergenza in distribuzione

\begin{itemize}
\item Un esempio

\item Teorema del limite centrale
\end{itemize}

\item[4.3] Analisi asintotica della regressione classica
\end{enumerate}

\item[5] Il modello di regressione stocastica

\begin{enumerate}
\item[5.1] Il modello e le ipotesi

\item[5.2] Quando i regressori e gli errori sono indipendenti

\item[5.3] Quando sono correlati

\begin{itemize}
\item La non convergenza degli stimatori dei minimi quadrati

\item Il metodo delle variabili strumentali

\item La scelta degli strumenti
\end{itemize}
\end{enumerate}
\end{enumerate}

\vspace{1.5cm}

\underline{Testi di riferimento}

\begin{itemize}
\item BALTAGI, Badi H., Econometrics, Springer, 1998

\item GREENE, W.H., Econometric Analysis, 3.rd Ed., Prentice Hall, 1997

\item HAYASHI, F., Econometrics, Princeton University Press, 2000

\item MADDALA, G.S., Introduction to Econometrics, 2.nd Ed., Prentice Hall,
1992
\end{itemize}

\end{document}
